\documentclass[12pt, titlepage]{article}

\usepackage{amsmath, mathtools}

\usepackage[round]{natbib}
\usepackage{amsfonts}
\usepackage{amssymb}
\usepackage{graphicx}
\usepackage{colortbl}
\usepackage{xr}
\usepackage{hyperref}
\usepackage{longtable}
\usepackage{xfrac}
\usepackage{tabularx}
\usepackage{float}
\usepackage{siunitx}
\usepackage{booktabs}
\usepackage{multirow}
\usepackage[section]{placeins}
\usepackage{caption}
\usepackage{fullpage}

\hypersetup{
bookmarks=true,     % show bookmarks bar?
colorlinks=true,       % false: boxed links; true: colored links
linkcolor=red,          % color of internal links (change box color with linkbordercolor)
citecolor=blue,      % color of links to bibliography
filecolor=magenta,  % color of file links
urlcolor=cyan          % color of external links
}

\usepackage{array}

\externaldocument{../../SRS/SRS}

%% Comments

\usepackage{color}

\newif\ifcomments\commentstrue %displays comments
%\newif\ifcomments\commentsfalse %so that comments do not display

\ifcomments
\newcommand{\authornote}[3]{\textcolor{#1}{[#3 ---#2]}}
\newcommand{\todo}[1]{\textcolor{red}{[TODO: #1]}}
\else
\newcommand{\authornote}[3]{}
\newcommand{\todo}[1]{}
\fi

\newcommand{\wss}[1]{\authornote{blue}{SS}{#1}} 
\newcommand{\plt}[1]{\authornote{magenta}{TPLT}{#1}} %For explanation of the template
\newcommand{\an}[1]{\authornote{cyan}{Author}{#1}}

%% Common Parts

\newcommand{\progname}{ProgName} % PUT YOUR PROGRAM NAME HERE
\newcommand{\authname}{Team \#, Team Name
\\ Student 1 name
\\ Student 2 name
\\ Student 3 name
\\ Student 4 name} % AUTHOR NAMES                  

\usepackage{hyperref}
    \hypersetup{colorlinks=true, linkcolor=blue, citecolor=blue, filecolor=blue,
                urlcolor=blue, unicode=false}
    \urlstyle{same}
                                


\begin{document}

\title{Module Interface Specification for \progname{}}

\author{\authname}

\date{\today}

\maketitle

\pagenumbering{roman}

\section{Revision History}

\begin{tabularx}{\textwidth}{p{3cm}p{2cm}X}
\toprule {\bf Date} & {\bf Version} & {\bf Notes}\\
\midrule
17/01/2024 & 1.0 & Initial version of MIS\\
\bottomrule
\end{tabularx}

~\newpage

\section{Symbols, Abbreviations and Acronyms}

See SRS Documentation at \url{https://github.com/r-yeh/grocery-spending-tracker/blob/master/docs/SRS/SRS.pdf}
\newline
\renewcommand{\arraystretch}{1.2}
\begin{tabular}{l l} 
  \toprule		
  \textbf{symbol} & \textbf{description}\\
  \midrule 
  SRS & Software Requirements Specification\\
  MIS & Module Interface Specification \\
  SKU & Stock Keeping Unit: ID of Product Used by Retailers \\
  UPC & Universal Product Code: widely used barcode symbology for tracking trade items in stores \\
  JWT & JSON Web Token \\
  \bottomrule
\end{tabular}\\

\newpage

\tableofcontents

\newpage

\pagenumbering{arabic}

\section{Introduction}

The following document details the Module Interface Specifications for
the Grocery Spending Tracker. The Grocery Spending Tracker is a mobile application
designed to help households face the ever rising cost of groceries in Canada.
To accomplish this, users can easily scan their grocery receipts with their device camera allowing the
application to record and track user spending. Analytics are then produced
with this data to help users better visualize their spending habits. They can also
set budgets and spending goals for themselves allowing the application to
suggest cheaper alternatives based on recently purchased grocery items and
location preferences. Overall, these features should help users stay more informed on their
grocery spending leading to smarter financial decisions and reduced spending over time.

Complementary documents include the System Requirement Specifications
and Module Guide.  The full documentation and implementation can be
found at \url{https://github.com/r-yeh/grocery-spending-tracker}.

\section{Notation}

The structure of the MIS for modules comes from \citet{HoffmanAndStrooper1995},
with the addition that template modules have been adapted from
\cite{GhezziEtAl2003}.  The mathematical notation comes from Chapter 3 of
\citet{HoffmanAndStrooper1995}.  For instance, the symbol := is used for a
multiple assignment statement and conditional rules follow the form $(c_1
\Rightarrow r_1 | c_2 \Rightarrow r_2 | ... | c_n \Rightarrow r_n )$.

The following table summarizes the primitive data types used by \progname. 

\begin{center}
\renewcommand{\arraystretch}{1.2}
\noindent 
\begin{tabular}{l l p{7.5cm}} 
\toprule 
\textbf{Data Type} & \textbf{Notation} & \textbf{Description}\\ 
\midrule
character & char & a single symbol or digit\\
integer & $\mathbb{Z}$ & a number without a fractional component in (-$\infty$, $\infty$) \\
natural number & $\mathbb{N}$ & a number without a fractional component in [1, $\infty$) \\
real & $\mathbb{R}$ & any number in (-$\infty$, $\infty$)\\
real positive & $\mathbb{R}+$ & any number in [0, $\infty$) \\
boolean & $\mathbb{B}$ & a truth value, resolving to true or false \\
\bottomrule
\end{tabular} 
\end{center}

\noindent
The specification of \progname \ uses some derived data types: lists, strings, maps
and tuples. Lists are data types that store an array of a set data type, formatted as List$<$Object$>$. Strings
are sequences of characters. Maps are a list of key value pairs of two identical or different data types
formatted as Map$<$Key, Value$>$. Tuples contain a series of values, potentially of
different types. Furthermore, some exported data types using a combination of the above
primitive and derived data types will be defined as required in certain
MIS sections. In addition, \progname \ uses functions, which
are defined by the data types of their inputs and outputs. Local functions are
described by giving their type signature followed by their specification.

\section{Module Decomposition}

The following table is taken directly from the Module Guide document for this project.

\begin{table}[h!]
\centering
\begin{tabular}{p{0.3\textwidth} p{0.6\textwidth}}
\toprule
\textbf{Level 1} & \textbf{Level 2}\\
\midrule

{Hardware-Hiding} & ~ \\
\midrule

\multirow{7}{0.3\textwidth}{Behaviour-Hiding} & Receipt Vision Module (OCR)\\
& Receipt Extraction Module\\
& Location Management Module\\
& User Analytics Module\\
& Users Module\\
& Authentication Module\\
\midrule

\multirow{3}{0.3\textwidth}{Software Decision} & Recommendation Engine\\
& Classification Engine\\
& Database Driver Module\\
\bottomrule

\end{tabular}
\caption{Module Hierarchy}
\label{TblMH}
\end{table}

\newpage

\section{MIS of Receipt Vision Module} \label{mVision}

The Receipt Vision Module is responsible for providing methods and logic related to image
capture of the receipt. This includes responsibilities like handling device permissions and camera access.

\subsection{Module}

receiptVision

\subsection{Uses}

None

\subsection{Syntax}

\subsubsection{Exported Constants}

None

\subsubsection{Exported Types}

None

\subsubsection{Exported Access Programs}

\begin{center}
  \begin{tabular}{p{5cm} p{4cm} p{3cm} p{4cm}}
  \hline
  \textbf{Name} & \textbf{In} & \textbf{Out} & \textbf{Exceptions} \\
  \hline
  requestCameraPermission & - & - & - \\
  didChangeAppLifeCycle & appLifecycle & - & - \\
  scanReceipt & - & String & noPermissionException \\
  \hline
  \end{tabular}
\end{center}

\subsection{Semantics}

\subsubsection{State Variables}

\textbf{App Lifecycle} \textit{appLifecycle}: Tracks module activity on the user's device. \\
\textbf{$\mathbb{B}$} \textit{isPermissionGranted}: Stores camera permission status. \\
\textbf{Device Camera} \textit{selectedCamera}: Stores data on which camera to use.

\subsubsection{Environment Variables}

\textbf{Device Camera} User Camera \\
\textbf{2D Sequence of Pixels} User Display

\subsubsection{Assumptions}

\begin{itemize}
  \item Application will always be used on a device with a camera.
\end{itemize}

\subsubsection{Access Routine Semantics}

\noindent requestCameraPermission():
\begin{itemize}
  \item transition: Request camera permission and update \textit{isPermissionGranted} accordingly.
  \item output: none
  \item exception: none
  \end{itemize}

\noindent didChangeAppLifeCycle(\textit{appLifecycle}):
\begin{itemize}
  \item transition: $(\neg isPermissionGranted) \Rightarrow$ no action is performed $|$\\
                    $(isPermissionGranted \land appLifecycle = active) \Rightarrow$ Turn on device camera $|$\\
                    $(isPermissionGranted \land appLifecycle = inactive) \Rightarrow$ Turn off device camera
  \item output: none
  \item exception: none
\end{itemize}

\noindent scanReceipt():
\begin{itemize}
  \item transition: Display camera preview and interface for image capture.
  \item output: $out := isPermissionGranted \Rightarrow$ take picture with camera and parse text
  \item exception: $exc := (\neg isPermissionGranted) \Rightarrow$ noPermissionException
\end{itemize}

\subsubsection{Local Functions}

\noindent initializeCamera(availableCameras):
\begin{itemize}
  \item transition: Let $findBackFacingCamera = \forall cameras (\exists camera \in cameras : camera.direction = back)$ \\
                    $(\neg isPermissionGranted) \Rightarrow$ no action is performed $|$\\
                    $isPermissionGranted \Rightarrow findBackFacingCamera(availableCameras) \Rightarrow $ \\
                    $selectedCamera = camera$
\end{itemize}

\newpage

\section{MIS of Receipt Extraction Module} \label{mExtraction}

The Receipt Extraction Module contains methods related to extracting relevant data such as grocery items,
prices, and date of transaction from a parsed receipt.

\subsection{Module}

receiptExtraction

\subsection{Uses}

Database Driver Module (Section \ref{mDBDriver})

\subsection{Syntax}

\subsubsection{Exported Constants}

None

\subsubsection{Exported Types}

\textbf{Item}

\begin{table}[H]
  \begin{tabular}{|p{0.25\textwidth}|p{0.25\textwidth}|p{0.40\textwidth}|}
    \hline
    \textbf{Output Name} & \textbf{Output Type} & \textbf{Description} \\
    \hline
    itemKey & String & SKU of the purchased item \\
    \hline
    itemDesc & String & Name of the product on the receipt \\
    \hline
    price & $\mathbb{R}$ & Purchase price of the product \\
    \hline
    taxed & $\mathbb{B}$ & Whether an item is taxed (Some groceries in Canada aren't taxed) \\
    \hline
  \end{tabular}
\end{table}

\textbf{UpdatedItem}

\begin{table}[H]
  \begin{tabular}{|p{0.25\textwidth}|p{0.25\textwidth}|p{0.40\textwidth}|}
    \hline
    \textbf{Output Name} & \textbf{Output Type} & \textbf{Description} \\
    \hline
    itemKey & Optional$<$String$>$ & SKU of the purchased item \\
    \hline
    itemDesc & Optional$<$String$>$ & Name of the product on the receipt \\
    \hline
    price & Optional$<\mathbb{R}>$ & Purchase price of the product \\
    \hline
    taxed & Optional$<\mathbb{B}>$ & Whether an item is taxed (Some groceries in Canada aren't taxed) \\
    \hline
  \end{tabular}
\end{table}

\textbf{GroceryTrip}

\begin{table}[H]
  \begin{tabular}{|p{0.25\textwidth}|p{0.25\textwidth}|p{0.40\textwidth}|}
    \hline
    \textbf{Output Name} & \textbf{Output Type} & \textbf{Description} \\
    \hline
    userId & String & ID of the user \\
    \hline
    dateTime & $\mathbb{N}$ & Date and time of transaction as Unix timestamp \\
    \hline
    location & String & Address of the grocery store location \\
    \hline
    items & List$<$Item$>$ & List of purchased grocery items from receipt \\
    \hline
    subtotal & $\mathbb{R}$ & Total spent on grocery trip before tax \\
    \hline
    total & $\mathbb{R}$ & Total spent on grocery trip \\
    \hline
    tripDesc & Optional$<$String$>$ & Notes made by the user \\
    \hline
  \end{tabular}
\end{table}

\textbf{UpdatedGroceryTrip}

\begin{table}[H]
  \begin{tabular}{|p{0.25\textwidth}|p{0.35\textwidth}|p{0.3\textwidth}|}
    \hline
    \textbf{Output Name} & \textbf{Output Type} & \textbf{Description} \\
    \hline
    dateTime & Optional$<\mathbb{N}>$ & Date and time of transaction as Unix timestamp \\
    \hline
    location & Optional$<$String$>$ & Address of the grocery store location \\
    \hline
    items & Optional$<$List$<$UpdatedItem$>$$>$ & List of purchased grocery items from receipt \\
    \hline
    subtotal & Optional$<\mathbb{R}>$ & Total spent on grocery trip before tax \\
    \hline
    total & Optional$<\mathbb{R}>$ & Total spent on grocery trip \\
    \hline
    tripDesc & Optional$<$String$>$ & Notes made by the user \\
    \hline
  \end{tabular}
\end{table}

\subsubsection{Exported Access Programs}

This module takes as input User as described in Section \ref{mUsers}.

\begin{center}
  \begin{tabular}{p{5cm} p{4cm} p{3cm} p{4cm}}
  \hline
  \textbf{Name} & \textbf{In} & \textbf{Out} & \textbf{Exceptions} \\
  \hline
  createGroceryTrip & User, String & GroceryTrip & -\\
  updateGroceryTrip & UpdatedGroceryTrip & GroceryTrip & -\\
  createItem & String & Item & -\\
  updateItem & UpdatedItem & Item & -\\
  \hline
  \end{tabular}
\end{center}

\subsection{Semantics}

\subsubsection{State Variables}

None

\subsubsection{Environment Variables}

\textbf{2D Sequence of Pixels} User Display

\subsubsection{Assumptions}
\begin{itemize}
  \item \textit{createItem} will always be working with strings of the format "\textit{itemKey itemDesc Optional$<$H$>$ price}"
\end{itemize}

\subsubsection{Access Routine Semantics}

\noindent createGroceryTrip(\textit{user, receiptData}):
\begin{itemize}
  \item transition: Extracts \textit{GroceryTrip} data, redirects user to new page where confirmation
of extracted data can be performed.
  \item output: $out :=$ new GroceryTrip(\textit{user.userId}, toUnixTimestamp(getDateTime(\textit{receiptData})), getLocation(\textit{receiptData}),
                        getItems(\textit{receiptData}), getSubtotal(\textit{receiptData}), getTotal(\textit{receiptData}), "")
  \item exception: none
\end{itemize}

\noindent updateGroceryTrip(\textit{updatedTrip}):
\begin{itemize}
  \item transition: Updates \textit{GroceryTrip}, reloads confirmation page.
  \item output: $out :=$ Update all fields of the \textit{GroceryTrip} object with non-empty fields of \textit{updatedTrip}.
  \item exception: none
\end{itemize}

\noindent createItem(\textit{itemString}):
\begin{itemize}
  \item transition: none
  \item output: Let itemComponents = \text{\textit{itemString}.split}(" ") \\
                $out := \exists taxed \in itemString : taxed \in L(\text{ H }) \Rightarrow$ \\
                new Item(itemComponents[0], itemComponents[1], itemComponents[3], True) $|$ \\
                $\neg(\exists taxed \in itemString : taxed \in L(\text{ H })) \Rightarrow$ \\
                new Item(itemComponents[0], itemComponents[1], itemComponents[2], False)
  \item exception: none
\end{itemize}

\noindent updateItem(\textit{updatedItem}):
\begin{itemize}
  \item transition: none
  \item output: $out :=$ Update all fields of the \textit{Item} object with non-empty fields of \textit{updatedItem}.
  \item exception: none
\end{itemize}

\subsubsection{Local Functions}

\noindent getDateTime(\textit{receiptData}):
\begin{itemize}
  \item output: $\text{Let } A=\{x : 0 \leq x \leq 9\}, B=\{x : 0 \leq x \leq 2\}, C=\{x : 0 \leq x \leq 5\}$ \\
                $\text{Let } containsDateTime=\exists data \in receiptData : data$\\ $\in L((A|AA)(/|-)(A|AA)(/|-)(AAAA)\text{ }(BA)\text{:}(CA)\text{:}(CA))$ \\
                $out := containsDateTime(receiptData) \Rightarrow data | \neg containsDateTime(receiptData) \Rightarrow null$
\end{itemize}

\noindent toUnixTimestamp(\textit{dateTime}):
\begin{itemize}
  \item output: takes String \textit{dateTime} and converts it to a Unix timestamp.
\end{itemize}

\noindent getLocation(\textit{receiptData}):
\begin{itemize}
  \item output: $\text{Let } A=\{x : 0 \leq x \leq 9\}, B=\{a,b,\dots,y,z\}\cup\{A,B,\dots,Y,Z\}\cup\{\text{ }\}$ \\
                $\text{Let } containsLocation=\exists data \in receiptData : data \in L(A{}^\ast \text{ } B{}^\ast)$ \\
                $out := containsLocation(receiptData) \Rightarrow data | \neg containsLocation(receiptData) \Rightarrow null$
\end{itemize}

\noindent getItems(\textit{receiptData}):
\begin{itemize}
  \item output: $\text{Let } A=\{x : 0 \leq x \leq 9\}, B=\{a,b,\dots,y,z\}\cup\{A,B,\dots,Y,Z\}\cup\{\text{ }\}$ \\
                $\text{Let } getItems=\exists data : data \in L(B^\ast \text{ } (\$|\epsilon)A{}^\ast.AA), itemList=[]$ \\
                $out := itemList \text{ s.t. } \forall text(getItems(text) \land text \in receiptData) : itemList.append(createItem(text))$
\end{itemize}

\noindent getSubtotal(\textit{receiptData}):
\begin{itemize}
  \item output: $\text{Let } A=\{x : 0 \leq x \leq 9\}$ \\
                $\text{Let } containsSubtotal=\exists data \in receiptData : data \in L((\text{SUBTOTAL}|\text{Subtotal})\text{ }\$A{}^\ast.AA)$ \\
                $out := containsSubtotal(receiptData) \Rightarrow data | \neg containsSubtotal(receiptData) \Rightarrow null$
\end{itemize}

\noindent getTotal(\textit{receiptData}):
\begin{itemize}
  \item output: $\text{Let } A=\{x : 0 \leq x \leq 9\}$ \\
                $\text{Let } containsTotal=\exists data \in receiptData : data \in L((\text{TOTAL}|\text{Total})\text{ }\$A{}^\ast.AA)$ \\
                $out := containsTotal(receiptData) \Rightarrow data | \neg containsTotal(receiptData) \Rightarrow null$
\end{itemize}

\newpage

\section{MIS of Location Management Module} \label{mLocation} 
The Location Management Module manages user location data and outputs relevant resources based the available user location data.

\subsection{Module}

locationManagement

\subsection{Uses}

Database Driver Module (Section \ref{mDBDriver})

\subsection{Syntax}

\subsubsection{Exported Constants}
None

\subsubsection{Exported Types}

None

\subsubsection{Exported Access Programs}

\begin{center}
\begin{tabular}{p{5cm} p{3cm} p{3cm} p{5cm}}
\hline
\textbf{Name} & \textbf{In} & \textbf{Out} & \textbf{Exceptions} \\
\hline
getLastLocation & - & String & noLocationSavedException \\
setLastLocation & String & - & invalidLocationException \\
getNearbyStores & - & List$<$String$>$ & - \\
setRadius & $\mathbb{N}$ & - & - \\
\hline
\end{tabular}
\end{center}

\subsection{Semantics}

\subsubsection{State Variables}
None

\subsubsection{Environment Variables}
None

\subsubsection{Assumptions}
None

\subsubsection{Access Routine Semantics}

\noindent getLastLocation():
\begin{itemize}
\item transition: none
\item output: \( \textit{out} \coloneqq \neg location = null \Rightarrow  location \), where \textit{location} is the user location stored in the database (Section \ref{mDBDriver})
\item exception: \( \textit{exc} \coloneqq location = null \Rightarrow  noLocationSavedException \), where \textit{location} is the user location stored in the database (Section \ref{mDBDriver})
\end{itemize}

\noindent setLastLocation(\textit{location}):
\begin{itemize}
\item transition: Set user location in the database (Section \ref{mDBDriver}) to \textit{location}
\item output: none
\item exception: \( \textit{exc} \coloneqq invalidLocationException \), if \textit{location} is not a valid location in the places dependency
\end{itemize}

\noindent getNearbyStores():
\begin{itemize}
\item transition: none
\item output: \( \textit{out} \coloneqq locations \), where \textit{locations} is a List of Strings representing relevant stores specified by the user's preferences stored in the database (Section \ref{mDBDriver})
\item exception: none
\end{itemize}

\noindent setRadius(\textit{R}):
\begin{itemize}
\item transition: Set user location radius in the database (Section \ref{mDBDriver}) to \textit{R}
\item output: none
\item exception: none
\end{itemize}


\subsubsection{Local Functions}
None

\newpage

\section{MIS of User Analytics Module} \label{mAnalytics} 
The User Analytics Module provides analytics and feedback on user purchasing behaviour, offering insight on recent and future purchasing for individual users.

\subsection{Module}

userAnalytics

\subsection{Uses}

Database Driver Module (Section \ref{mDBDriver})

\subsection{Syntax}

\subsubsection{Exported Constants}
None

\subsubsection{Exported Types}

\textbf{Item} as defined in Section \ref{mExtraction}

\subsubsection{Exported Access Programs}

\begin{center}

\begin{tabular}{p{5cm} p{3cm} p{4cm} p{4cm}}
\hline
\textbf{Name} & \textbf{In} & \textbf{Out} & \textbf{Exceptions} \\
\hline
getPurchaseHistoryByTrip & $\mathbb{N}$ & List$<$List$<$Item$>$$>$  & - \\
getSpendingByCategory & $\mathbb{N}^2$ & Map$<$String, List$<$Item$>$$>$  & - \\
getSpendingByTime & $\mathbb{N}^2$ & Map$<$String, $\mathbb{R}$+$>$ & - \\
createBudgetGoals & $\mathbb{N}^3$, String & - & - \\
updateBudgetGoals & $\mathbb{N}^4$, String & - & - \\
deleteBudgetGoals & $\mathbb{N}$, String & - & - \\
getBudgetGoals & - & List$<$($\mathbb{N}^3$, String)$>$ & - \\
\hline
\end{tabular}
\end{center}

\subsection{Semantics}

\subsubsection{State Variables}
\textbf{$\mathbb{N}$} \textit{pageSize}: number of trips in a page of shopping trips 

\subsubsection{Environment Variables}
\textbf{GroceryTrip} (Section \ref{mExtraction})


\subsubsection{Assumptions}
None

\subsubsection{Access Routine Semantics}

\noindent getPurchaseHistoryByTrip(\textit{page}):
\begin{itemize}
\item transition: none
\item output: $\textit{out} \coloneqq [GroceryTrip_{1}, GroceryTrip_{2},..., GroceryTrip_{pageSize}]$ s.t. \\ $GroceryTrip_{i}.dateTime > GroceryTrip_{i+1}.dateTime$, where GroceryTrip are extracted from (Section \ref{mDBDriver})
\item exception: none
\end{itemize}

\noindent getSpendingByCategory(\textit{startDate}, \textit{endDate}):
\begin{itemize}
\item transition: none
\item output: \( \textit{out} \coloneqq spendingList \), where \textit{spendingList} is a List of user purchases in the time period specified by \textit{startDate} and \textit{endDate} grouped by product category stored in the database (Section \ref{mDBDriver})
\item exception: none
\end{itemize}

\noindent setSpendingByTime(\textit{startDate}, \textit{endDate}):
\begin{itemize}
\item transition: none
\item output: \( \textit{out} \coloneqq spendingList \), where \textit{spendingList} is a List of user spending amount by day within the specified \textit{startDate} and \textit{endDate} stored in the database (Section \ref{mDBDriver})
\item exception: none
\end{itemize}

\noindent createBudgetGoals(\textit{startDate}, \textit{endDate}, \textit{budget}, \textit{category}):
\begin{itemize}
\item transition: Create budget goal for the user in the Goals table of the database (Section \ref{mDBDriver})
\item output: none
\item exception: none
\end{itemize}

\noindent updateBudgetGoals(\textit{budgetId}, \textit{startDate}, \textit{endDate}, \textit{budget}, \textit{category}):
\begin{itemize}
\item transition: Update budget goal for the user in the Goals table of the database (Section \ref{mDBDriver})
\item output: none
\item exception: none
\end{itemize}

\noindent deleteBudgetGoals(\textit{budgetId}):
\begin{itemize}
\item transition: Delete budget goal for the user by budget goal ID (\textit{budgetId}) in the Goals table of the database (Section \ref{mDBDriver})
\item output: none
\item exception: none
\end{itemize}

\noindent getBudgetGoals():
\begin{itemize}
\item transition: none
\item output: \( \textit{out} \coloneqq budgetGoals \) Where \textit{budgetGoals} is the List of budget goals of the user in the Goals table of the database (Section \ref{mDBDriver})
\item exception: none
\end{itemize}


\subsubsection{Local Functions}
None

\newpage

\section{MIS of Users Module} \label{mUsers} 
The Users Module exists to handle CRUD operations for operations related to users and their grocery trips.

\subsection{Module}

users

\subsection{Uses}

Database Driver Module (Section \ref{mDBDriver}) 

\subsection{Syntax}

\subsubsection{Exported Constants}
None

\subsubsection{Exported Types}
\textbf{Item} as defined in Section \ref{mExtraction} 

\noindent \textbf{GroceryTrip} as defined in Section \ref{mExtraction} \\

\noindent \textbf{User}

\begin{table}[H]
  \begin{tabular}{|p{0.25\textwidth}|p{0.25\textwidth}|p{0.40\textwidth}|}
    \hline
    \textbf{Output Name} & \textbf{Output Type} & \textbf{Description} \\
    \hline
    userId & String & Unique user id number \\
    \hline
    email & String & Email of user \\
    \hline
    password & String & Password of user (Encrypted) \\
    \hline
    firstName & String & First name of User \\
    \hline
    lastName & String & Last name of User \\
    \hline
    location & String & Location of user's home-base. (Coordinates) \\
    \hline
  \end{tabular}
\end{table}

\subsubsection{Exported Access Programs}

\begin{center}
\begin{tabular}{p{5cm} p{3cm} p{3cm} p{5cm}}
\hline
\textbf{Name} & \textbf{In} & \textbf{Out} & \textbf{Exceptions} \\
\hline
newUser & User & - & userExistsException \\
getUser & String & User & userDneException \\
updateUser & User & - & userDneException \\
deleteUser & String & - & userDneException \\
addTrip & GroceryTrip & - & userDneException \\
getTrip & $\mathbb{N}$ & GroceryTrip & - \\
\hline
\end{tabular}
\end{center}

\subsection{Semantics}

\subsubsection{State Variables}
None

\subsubsection{Environment Variables}
None

\subsubsection{Assumptions}
None

\subsubsection{Access Routine Semantics}

\noindent newUser(\textit{user}):
\begin{itemize}
\item transition: \( \neg userExists(\textit{user.userId}) \Rightarrow addUserInDB(\textit{user})\) 
\item output: none
\item exception \( \textit{exc} \coloneqq userExists(\textit{user.userId}) \Rightarrow \\ userExistsException\)
\end{itemize}

\noindent getUser(\textit{userId}):
\begin{itemize}
\item transition: none
\item output: \( \textit{out} \coloneqq userExists(\textit{user.userId}) \Rightarrow getUserInDB(userId)\)
\item exception \( \textit{exc} \coloneqq \neg userExists(\textit{user.userId}) \Rightarrow \\ userDneException\)
\end{itemize}

\noindent updateUser(\textit{user}):
\begin{itemize}
\item transition: \( userExists(\textit{user.userId}) \Rightarrow updateUser(\textit{user})\)
\item output: none
\item exception \( \textit{exc} \coloneqq \neg userExists(\textit{user.userId}) \Rightarrow \\ userDneException\)
\end{itemize}

\noindent deleteUser(\textit{user}):
\begin{itemize}
\item transition: \( userExists(\textit{user.userId}) \Rightarrow deleteUserInDB(\textit{user})\) 
\item output: none
\item exception \( \textit{exc} \coloneqq \neg userExists(\textit{user.userId}) \Rightarrow \\ userDneException\)
\end{itemize}

\noindent addTrip(\textit{groceryTrip}):
\begin{itemize}
\item transition: \( userExists(\textit{groceryTrip.userId}) \Rightarrow addTripInDB(\textit{groceryTrip})\) 
\item output: none
\item exception \( \textit{exc} \coloneqq \neg userExists(\textit{user.userId}) \Rightarrow \\ userDneException\)
\end{itemize}

\noindent getTrip(\textit{dateTime}):
\begin{itemize}
\item transition: none
\item output: \( \textit{out} \coloneqq getTripFromDateTime(\textit{dateTime}) \)
\end{itemize}

\subsubsection{Local Functions}
addUserInDB(\textit{user})
\begin{itemize}
\item transition: Add tuple to User database (Section \ref{mDBDriver}) using fields from \textit{user}.
\end{itemize}

\noindent getUserInDB(\textit{userId})
\begin{itemize}
\item output:  Query User database (Section \ref{mDBDriver}) where \textit{userId} and map to fields in \textit{user}. Output \textit{user}.
\end{itemize}

\noindent updateUserInDB(\textit{user})
\begin{itemize}
\item output:  Replace tuple with key \textit{user.userId} in User database (Section \ref{mDBDriver}) with \textit{user} by mapping fields fields.
\end{itemize}

\noindent deleteUserInDB(\textit{user})
\begin{itemize}
\item transition: Remove tuple to User database (Section \ref{mDBDriver}) using fields from primary key \textit{user.userId}.
\end{itemize}

\noindent userExists(\textit{groceryTrip.userId})
\begin{itemize}
\item output: Query User database (Section \ref{mDBDriver}) and output True if a tuple with \textit{userId} = \textit{groceryTrip.userId} exists in the table.
\end{itemize}

\noindent addTripInDB(\textit{groceryTrip})
\begin{itemize}
\item transition: Add tuple to Trip database (Section \ref{mDBDriver}) using fields from \textit{groceryTrip}.
\end{itemize}

\noindent getTripFromDateTime(\textit{dateTime})
\begin{itemize}
\item output: Query Trip database (Section \ref{mDBDriver}) and output the \textit{GroceryTrip} tuple that is closest to \textit{dateTime}.
\end{itemize}

\newpage

\section{MIS of Authentication Module} \label{mAuthentication} 
The Authentication module acts to "log-in" the user by providing the client with a JWT token.

\subsection{Module}

authentication

\subsection{Uses}

Database Driver Module (Section \ref{mDBDriver}) 

\subsection{Syntax}

\subsubsection{Exported Constants}

None

\subsubsection{Exported Types}

None

\subsubsection{Exported Access Programs}

\begin{center}
\begin{tabular}{p{5cm} p{3cm} p{3cm} p{5cm}}
\hline
\textbf{Name} & \textbf{In} & \textbf{Out} & \textbf{Exceptions} \\
\hline
getToken & String$^2$ & String & invalidCredentialsException \\
verifyToken & String & String & invalidTokenException \\
\hline
\end{tabular}
\end{center}

\subsection{Semantics}

\subsubsection{State Variables}
None

\subsubsection{Environment Variables}
\textbf{String} PrivateKey (for JWT) \\
\textbf{String} PublicKey (for JWT)

\subsubsection{Assumptions}
None

\subsubsection{Access Routine Semantics}

\noindent getToken(\textit{email}, \textit{passwordEncrypted}):
\begin{itemize}
\item transition: \textit{signedJwtToken} produced using \textit{PrivateKey}
\item output: \( \textit{out} \coloneqq verifyCredentials(\textit{email}, \textit{passwordEncrypted}) \Rightarrow  signedJwtToken\) 
\item exception \( \textit{exc} \coloneqq \neg verifyCredentials(\textit{email}, \textit{passwordEncrypted}) \Rightarrow \\ invalidCredentialsException\)
\end{itemize}

\noindent verifyJwtToken(\textit{jwtToken}):
\begin{itemize}
\item transition: none
\item output: \( \textit{out} \coloneqq verifyToken(\textit{jwtToken})\)
\item exception: \( \textit{exc} \coloneqq \neg verifyToken(\textit{jwtToken}) \Rightarrow invalidTokenException \), if the provided JWT token does not evaluate true in comparison to the PublicKey
\end{itemize}

\subsubsection{Local Functions}
verifyCredentials(\textit{email}, \textit{passwordEncrypted})
\begin{itemize}
\item output: \( \textit{out} \coloneqq \) True if the password is deemed correct for the given email using the database (Section \ref{mDBDriver}), else False
\end{itemize}

\noindent verifyToken(\textit{jwtToken})
\begin{itemize}
\item output: \( \textit{out} \coloneqq \) True if the JWT token is deemed corrrect using the PublicKey and JWTLibrary, else False
\end{itemize}

\newpage

\section{MIS of Recommendation Module} \label{mRecommendation}

Module for recommending user purchases.

\subsection{Module}

recommendationModule

\subsection{Uses}

Database Driver Module (Section \ref{mDBDriver})

\subsection{Syntax}

\subsubsection{Exported Constants}

None

\subsubsection{Exported Types}

\textbf{Item} as defined in Section \ref{mExtraction} \\

\subsubsection{Exported Access Programs}

\begin{center}
    \begin{tabular}{p{5cm} p{3cm} p{3cm} p{5cm}}
    \hline
    \textbf{Name} & \textbf{In} & \textbf{Out} & \textbf{Exceptions} \\
    \hline
    collectItemData & Item & - & InvalidDataFormatException \\
    getCheaperAlternatives & String & List$<$String$>$ & - \\
    getRecommendationDetails & String & Item & - \\
    \hline
    \end{tabular}
\end{center}

\subsection{Semantics}

\subsubsection{State Variables}

None

\subsubsection{Environment Variables}

None

\subsubsection{Assumptions}

None

\subsubsection{Access Routine Semantics}

\noindent collectItemData(\textit{Item}):
\begin{itemize}
  \item transition: Updates the module's internal data structure with the new \textit{Item} data.
  \item output: none
  \item exception: $exc := \textit{InvalidDataFormatException}$, if \textit{Item} does not adhere to the expected format.
\end{itemize}

\noindent getCheaperAlternatives(\textit{itemId}):
\begin{itemize}
  \item transition: none
  \item output: $out := \textit{getCheaperAlternatives}(\textit{itemId})$ 
  \item exception: none
\end{itemize}

\noindent getRecommendationDetails(\textit{itemId}):
\begin{itemize}
  \item transition: none
  \item output: $out := \textit{getItemDetails}(\textit{itemId})$
  \item exception: none
\end{itemize}

\subsubsection{Local Functions}

\noindent getCheaperAlternatives(\textit{itemId}):
\begin{itemize}
  \item output: $out :=$ list of item IDs that are cheaper alternatives to the item identified by, \textit{itemId}
\end{itemize}

\noindent getItemDetails(\textit{itemId}):
\begin{itemize}
  \item output: $out :=$ detailed information of the item identified by, \textit{itemId}
\end{itemize}

\newpage

\section{MIS of Classification Module} \label{mClassification}

The Classification Module is responsible for interpreting and identifying recognized items from receipt data.

\subsection{Module}

classificationModule

\subsection{Uses}

Database Driver Module (Section \ref{mDBDriver})

\subsection{Syntax}

\subsubsection{Exported Constants}

None

\subsubsection{Exported Types}

\textbf{ClassifiedItem}

\begin{table}[H]
  \begin{tabular}{|p{0.2\linewidth}|p{0.2\linewidth}|p{0.5\linewidth}|}
    \hline
    \textbf{Column Name} & \textbf{Type} & \textbf{Description} \\
    \hline
    itemId & String & Unique id of this instance of this item \\
    \hline
    productKey & String & Universal item identifier (UPC) \\
    \hline
    dateTime & $\mathbb{N}$ & Date and time of purchase \\
    \hline
    price & $\mathbb{R}$ & Price of item \\
    \hline
    location & String & Location of purchase \\
    \hline
    userId & String & ID of user\\
    \hline
    itemDesc & String & Item description\\
    \hline
    isTaxed & $\mathbb{B}$ & Describes if the item is taxed\\
    \hline
    category & String & Describes the product category the item belongs to\\
    \hline
  \end{tabular}
\end{table}

\subsubsection{Exported Access Programs}

This module uses Item as an input, as described in Section \ref{mExtraction}

\begin{center}
    \begin{tabular}{p{5cm} p{4cm} p{3cm} p{4cm}}
    \hline
    \textbf{Name} & \textbf{In} & \textbf{Out} & \textbf{Exceptions} \\
    \hline
    setItemData & Item & - & invalidDataException \\
    getClassifiedItem & String & ClassifiedItem & - \\
    \hline
    \end{tabular}
\end{center}

\subsection{Semantics}

\subsubsection{State Variables}

None

\subsubsection{Environment Variables}

None

\subsubsection{Assumptions}

None

\subsubsection{Access Routine Semantics}

\noindent setItemData(\textit{Item}):
\begin{itemize}
  \item transition: Set Item data in the database (Section \ref{mDBDriver}) to \textit{Classified Item}
  \item output: none
  \item exception: $exc := invalidDataException$, if \textit{Item} does not adhere to expected format.
  \end{itemize}

\noindent getClassifiedItem(\textit{itemId}):
\begin{itemize}
  \item transition: none
  \item output: $out := \textit{itemExists}(\textit{itemId}) \Rightarrow \textit{getClassifiedItemInDB}(\textit{itemId})$
  \item exception: none
\end{itemize}

\subsubsection{Local Functions}

\noindent itemExists(\textit{itemId}):
\begin{itemize}
  \item output: $out := (\textit{itemId} \text{ is present in the database})$
\end{itemize}

\noindent getClassifiedItemInDB(\textit{itemId}):
\begin{itemize}
  \item output: $out := \textit{itemExists}(\textit{itemId}) \Rightarrow \textit{retrieveClassifiedItem}(\textit{itemId})$
  \item exception: $\textit{exc} := \textit{itemNotFoundException}$, if $\textit{itemId}$ does not exist in the database.
\end{itemize}

\noindent identifyItemUsingModel(\textit{inputItemData}):
\begin{itemize}
  \item output: $out := \textit{classifyUsingModel}(\textit{inputItemData})$
  \item exception: $\textit{exc} := \textit{modelIdentificationException}$, if the classification model fails to identify the item.
\end{itemize}

\noindent classifyUsingModel(\textit{inputItemData}):
\begin{itemize}
  \item output: $out :=$ the classification result of the model for, \textit{inputItemData}
  \item exception: $\textit{exc} := \textit{classificationException}$, if the model encounters an error during the classification process.
\end{itemize}

\noindent findMatchingProduct(\textit{inputItemData}):
\begin{itemize}
  \item output: $out :=$ product details from the database that best match, \textit{inputItemData}
  \item exception: $\textit{exc} := \textit{noProductFoundException}$, if no product in the database matches \textit{inputItemData}.
\end{itemize}

\newpage

\section{MIS of Database Driver Module} \label{mDBDriver}
Module for hiding database implementation. Database schema and tables are detailed in Section \ref{dbAppendix}

\subsection{Module}
dbModule

\subsection{Uses}
None

\subsection{Syntax}

\subsubsection{Exported Constants}
None

\subsubsection{Exported Types}
None

\subsubsection{Exported Access Programs}

\begin{center}
\begin{tabular}{p{4cm} p{4cm} p{4cm} p{3cm}}
\hline
\textbf{Name} & \textbf{In} & \textbf{Out} & \textbf{Exceptions} \\
\hline
insert & String$^2$ & - & - \\
update & String$^3$ & - & - \\
delete & String$^2$ & - & - \\
select & String$^3$ & String & - \\
\hline
\end{tabular}
\end{center}

\subsection{Semantics}

\subsubsection{State Variables}
None

\subsubsection{Environment Variables}
\textbf{Relational Database} db

\subsubsection{Assumptions}
\begin{itemize}
  \item The schema is set up as described in Section \ref{dbAppendix}
  \item The connection to the database is established with necessary permissions.
\end{itemize}

\subsubsection{Access Routine Semantics}

\noindent insert(\textit{tableName, values}):
\begin{itemize}
\item transition: the target table in \textit{db} specified by \textit{tableName} adds \textit{values}.
\item output: none
\item exception: none
\end{itemize}

\noindent update(\textit{tableName, values, conditions}):
\begin{itemize}
\item transition: the target items in \textit{db} specified by \textit{tableName} and \textit{conditions} is updated with \textit{values}.
\item output: none
\item exception: none
\end{itemize}

\noindent delete(\textit{tableName, conditions}):
\begin{itemize}
\item transition: the target items in \textit{db} specified by \textit{tableName} and \textit{conditions} are deleted.
\item output: none
\item exception: none
\end{itemize}

\noindent select(\textit{fields, tableName, conditions}):
\begin{itemize}
\item transition: none
\item output: $out \coloneqq data$, where \textit{data} describes the fields specified by \textit{fields} of the target items specified by \textit{tableName} and \textit{conditions} in db.
\item exception: none
\end{itemize}

\subsubsection{Local Functions}
None

\newpage

\bibliographystyle {plainnat}
\bibliography {../../../refs/References}

\newpage

\section{Appendix} \label{Appendix}

\subsection{Database Schema and Tables} \label{dbAppendix}

\subsubsection{rawItems}
\begin{table}[H]
  \begin{tabular}{|p{0.2\linewidth}|p{0.1\linewidth}|p{0.6\linewidth}|}
    \hline
    \textbf{Column Name} & \textbf{Type} & \textbf{Description} \\
    \hline
    itemId & String & PRIMARY KEY unique id of this instance of this item \\
    \hline
    itemKey & String & Item identifier specific to the retailer (usually SKU) \\
    \hline
    dateTime & $\mathbb{N}$ & Date and time of purchase \\
    \hline
    price & $\mathbb{R}$ & Price of item \\
    \hline
    location & String & Location of purchase \\
    \hline
    userId & String & FOREIGN KEY ID of user\\
    \hline
    itemDesc & String & Item description\\
    \hline
    isTaxed & $\mathbb{B}$ & Describes if the item is taxed\\
    \hline
  \end{tabular}
\end{table}

\subsubsection{classifiedItems}
\begin{table}[H]
  \begin{tabular}{|p{0.2\linewidth}|p{0.1\linewidth}|p{0.6\linewidth}|}
    \hline
    \textbf{Column Name} & \textbf{Type} & \textbf{Description} \\
    \hline
    itemId & String & PRIMARY KEY unique id of this instance of this item \\
    \hline
    productKey & String & Universal item identifier (UPC) \\
    \hline
    dateTime & $\mathbb{N}$ & Date and time of purchase \\
    \hline
    price & $\mathbb{R}$ & Price of item \\
    \hline
    location & String & Location of purchase \\
    \hline
    userId & String & FOREIGN KEY ID of user\\
    \hline
    transactionId & String &  FOREIGN KEY ID of transaction\\
    \hline
    itemDesc & String & Item description\\
    \hline
    isTaxed & $\mathbb{B}$ & Describes if the item is taxed\\
    \hline
    category & String & Describes the product category the item belongs to\\
    \hline
  \end{tabular}
\end{table}

\subsubsection{transaction}
\begin{table}[H]
  \begin{tabular}{|p{0.2\linewidth}|p{0.1\linewidth}|p{0.6\linewidth}|}
    \hline
    \textbf{Column Name} & \textbf{Type} & \textbf{Description} \\
    \hline
    transactionId & String & PRIMARY KEY ID of transaction\\
    \hline
    userId & String & FOREIGN KEY ID of user\\
    \hline
    dateTime & $\mathbb{N}$ & Date and time of transaction \\
    \hline
    location & String & Location of purchase \\
    \hline
    transactionDesc & String & Transaction description\\
    \hline
  \end{tabular}
\end{table}

\subsubsection{users}
\begin{table}[H]
  \begin{tabular}{|p{0.2\linewidth}|p{0.1\linewidth}|p{0.6\linewidth}|}
    \hline
    \textbf{Column Name} & \textbf{Type} & \textbf{Description} \\
    \hline
    userId & String & PRIMARY KEY unique id assigned to the user \\
    \hline
    email & String & User email \\
    \hline
    password & String & Encrypted user password \\
    \hline
    firstName & String & User first name \\
    \hline
    lastName & String & User last name \\
    \hline
    location & String & Location of the user, usually describes the user's home\\
    \hline
  \end{tabular}
\end{table}

\subsubsection{goals}
\begin{table}[H]
  \begin{tabular}{|p{0.2\linewidth}|p{0.1\linewidth}|p{0.6\linewidth}|}
    \hline
    \textbf{Column Name} & \textbf{Type} & \textbf{Description} \\
    \hline
    goalId & $\mathbb{N}$ & PRIMARY KEY unique id assigned to the goal \\
    \hline
    userId & String & FOREIGN KEY ID of user \\
    \hline
    startDate & $\mathbb{N}$ & Start date of the goal \\
    \hline
    endDate & $\mathbb{N}$ & End date of the goal \\
    \hline
    budget & $\mathbb{Z}$ & Money allotted for the goal \\
    \hline
    category & String & The product category of relevant to the goal\\
    \hline
  \end{tabular}
\end{table}


\end{document}
