\documentclass{article}

\usepackage{float}
\restylefloat{table}

\usepackage{booktabs}

\title{Team Contributions: Rev 0\\\progname}

\author{\authname}

\date{}

%% Comments

\usepackage{color}

\newif\ifcomments\commentstrue %displays comments
%\newif\ifcomments\commentsfalse %so that comments do not display

\ifcomments
\newcommand{\authornote}[3]{\textcolor{#1}{[#3 ---#2]}}
\newcommand{\todo}[1]{\textcolor{red}{[TODO: #1]}}
\else
\newcommand{\authornote}[3]{}
\newcommand{\todo}[1]{}
\fi

\newcommand{\wss}[1]{\authornote{blue}{SS}{#1}} 
\newcommand{\plt}[1]{\authornote{magenta}{TPLT}{#1}} %For explanation of the template
\newcommand{\an}[1]{\authornote{cyan}{Author}{#1}}

%% Common Parts

\newcommand{\progname}{ProgName} % PUT YOUR PROGRAM NAME HERE
\newcommand{\authname}{Team \#, Team Name
\\ Student 1 name
\\ Student 2 name
\\ Student 3 name
\\ Student 4 name} % AUTHOR NAMES                  

\usepackage{hyperref}
    \hypersetup{colorlinks=true, linkcolor=blue, citecolor=blue, filecolor=blue,
                urlcolor=blue, unicode=false}
    \urlstyle{same}
                                


\begin{document}

\maketitle

\section{Demo Plans}

For the demo, we plan to walk through the main success case for a user using the application.
The demo will start with logging in using an existing account, before scanning a receipt. Once
the receipt is scanned, we will show how the confirmation screen looks. We will show how editing
data on the confirmation screen works before sending the data to the backend. Once everything has
been properly added to the database, the analytics and history screen will be accessed to display the account's
purchase history and data.

\section{Meeting Attendance}

\begin{table}[H]
\centering
\begin{tabular}{ll}
\toprule
\textbf{Student} & \textbf{Meetings}\\
\midrule
Total & 13\\
Ryan Yeh & 13\\
Allan Fang & 13\\
Sawyer Tang & 13\\
Jason Nam & 13\\
\bottomrule
\end{tabular}
\end{table}

\section{Lecture Attendance}

\begin{table}[H]
\centering
\begin{tabular}{ll}
\toprule
\textbf{Student} & \textbf{Lectures}\\
\midrule
Total & 2\\
Ryan Yeh & 2\\
Allan Fang & 1\\
Sawyer Tang & 0\\
Jason Nam & 1\\
\bottomrule
\end{tabular}
\end{table}

There has been 2 lectures in the date-range that we are recording data.

\section{Commits}

The below contains the number of commits made by team members across all our repositories. This includes
contributions to the main repository as well as sub-repositories that make up our actual implementation.

\begin{table}[H]
\centering
\begin{tabular}{lll}
\toprule
\textbf{Student} & \textbf{Commits} & \textbf{Percent}\\
\midrule
Total & 117 & 100\% \\
Ryan Yeh & 60 & 51.28\% \\
Allan Fang & 13 & 11.11\% \\
Sawyer Tang & 18 & 15.39\% \\
Jason Nam & 26 & 22.22\% \\
\bottomrule
\end{tabular}
\end{table}

\begin{itemize}
  \item \textbf{Ryan Yeh:} I have an additional 45 commits made to our frontend repository that has not yet been
  merged to the main branch since the feature is still under development. I would also like to note that I write
  documentation on a local \LaTeX environment rather than Overleaf so I am generally able to make much more commits.
  \item \textbf{Allan Fang:} I have an additional 5 pending commits that are not ready for the main branch as the feature is incomplete.
  \item \textbf{Sawyer Tang:} 13 of my 18 commits are on the master branch of our back-end repository \href{https://github.com/Tanger71/grocery-spending-tracker-backend}{here}. Additionally, I squash my commits to the main repo before merging into master. The total of these commits unsquashed go from 18 to 30 detailed \href{https://github.com/r-yeh/grocery-spending-tracker/commits?author=Tanger71&since=2023-11-13&until=2024-01-28}{here}.
  \item \textbf{Jason Nam:} 9 of my 26 commits are on the master branch of our repository \href{https://github.com/jason-nam/grocery-web-scraper}{here}. I have an additional 3 pending commits on the same repository.
\end{itemize}

\section{Issue Tracker}

\begin{table}[H]
\centering
\begin{tabular}{lll}
\toprule
\textbf{Student} & \textbf{Authored (O+C)} & \textbf{Assigned (C only)}\\
\midrule
Ryan Yeh & 54 & 36 \\
Allan Fang & 40 & 29 \\
Sawyer Tang & 36 & 36 \\
Jason Nam & 26 & 21 \\
\bottomrule
\end{tabular}
\end{table}


\section{CICD}

Currently, \textit{Github Actions} is being used to compile all of our \LaTeX documents in the primary repository.
Going forward, we aim to have unit tests being executed when merging branches as well in order to ensure code quality.
Once modules are more complete, these tests will be created and integrated in to the CICD pipeline.

\end{document}
