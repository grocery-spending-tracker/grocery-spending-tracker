\documentclass{article}

\usepackage{booktabs}
\usepackage{tabularx}
\usepackage{hyperref}
\usepackage{pdflscape}
\usepackage{float}
\usepackage{adjustbox}
\usepackage{multirow}
\usepackage{enumitem}
\usepackage{longtable}
\usepackage[round]{natbib}
\usepackage[autostyle]{csquotes}

\hypersetup{
    colorlinks=true,       % false: boxed links; true: colored links
    linkcolor=red,          % color of internal links (change box color with linkbordercolor)
    citecolor=green,        % color of links to bibliography
    filecolor=magenta,      % color of file links
    urlcolor=cyan           % color of external links
}

\title{Hazard Analysis\\\progname}

\author{\authname}

\date{}

\input{../Comments}
%% Common Parts

\newcommand{\progname}{Grocery Spending Tracker} % PUT YOUR PROGRAM NAME HERE
\newcommand{\authname}{Team 1, JARS
\\ Jason Nam
\\ Allan Fang
\\ Ryan Yeh
\\ Sawyer Tang} % AUTHOR NAMES                  

\usepackage{hyperref}
    \hypersetup{colorlinks=true, linkcolor=blue, citecolor=blue, filecolor=blue,
                urlcolor=blue, unicode=false}
    \urlstyle{same}
                                


\begin{document}

\maketitle
\thispagestyle{empty}

~\newpage

\pagenumbering{roman}

\begin{table}[hp]
\caption{Revision History} \label{TblRevisionHistory}
\begin{tabularx}{\textwidth}{llX}
\toprule
\textbf{Date} & \textbf{Developer(s)} & \textbf{Change}\\
\midrule
16/10/23 & Ryan Yeh & Added List of Tables, List of Figures\\
16/10/23 & Ryan Yeh & Added Failure Mode and Effect Analyis\\
... & ... & ...\\
\bottomrule
\end{tabularx}
\end{table}

~\newpage

\tableofcontents

~\newpage

\listoftables

\listoffigures

\newpage

\pagenumbering{arabic}

\wss{You are free to modify this template.}

\section{Introduction}

\wss{You can include your definition of what a hazard is here.}

\section{Scope and Purpose of Hazard Analysis}

\section{System Boundaries and Components}

\section{Critical Assumptions}

\wss{These assumptions that are made about the software or system.  You should
minimize the number of assumptions that remove potential hazards.  For instance,
you could assume a part will never fail, but it is generally better to include
this potential failure mode.}

\newpage
\begin{landscape}
    \section{Failure Mode and Effect Analysis}
    \begin{longtable}{|p{0.15\textwidth}|p{0.15\textwidth}|p{0.25\textwidth}|p{0.25\textwidth}|p{0.35\textwidth}|p{0.1\textwidth}|p{0.05\textwidth}|}
        \caption{Failure Mode and Effect Analysis Table}  \\
		\hline
        \multicolumn{1}{|c|}{\textbf{Component}}
		& \multicolumn{1}{|c|}{\textbf{Failure Modes}}
		& \multicolumn{1}{|c|}{\textbf{Effects of Failure}}
		& \multicolumn{1}{|c|}{\textbf{Causes of Failure}}
		& \multicolumn{1}{|c|}{\textbf{Recommended Action}}
		& \multicolumn{1}{|c|}{\textbf{SR}}
		& \multicolumn{1}{|c|}{\textbf{Ref.}}  \\
        \hline
        Authenti\-cation
        & User is unable to login
        & User cannot access their account and application features
        & \begin{enumerate}[label=\alph*., leftmargin=*]
            \item User entered incorrect credentials
        \end{enumerate}
        & \begin{enumerate}[label=\alph*., leftmargin=*]
            \item Allow users to reset their password
        \end{enumerate}
        & \begin{enumerate}[label=\alph*., leftmargin=*]
            \item SR1
        \end{enumerate}
        & HA1 \\
        \cline{2-7}
        & User's account is hacked by a third-party
        & User's data can be read and modified by third-party
        & \begin{enumerate}[label=\alph*., leftmargin=*]
            \item Third-party gains access to the  user's account without their permission
        \end{enumerate}
        & \begin{enumerate}[label=\alph*., leftmargin=*]
            \item Allow users to reset their password
            \item Limit the number of login attempts a person has for a given time period
        \end{enumerate}
        & \begin{enumerate}[label=\alph*., leftmargin=*]
            \item SR1
            \item SR2
        \end{enumerate}
        & HA2 \\
        \hline
    \end{longtable}
\end{landscape}    

\wss{Include your FMEA table here}

\section{Safety and Security Requirements}

\wss{Newly discovered requirements.  These should also be added to the SRS.  (A
rationale design process how and why to fake it.)}

\section{Roadmap}

\wss{Which safety requirements will be implemented as part of the capstone timeline?
Which requirements will be implemented in the future?}

\end{document}