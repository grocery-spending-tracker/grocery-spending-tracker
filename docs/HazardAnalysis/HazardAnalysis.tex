\documentclass{article}

\usepackage{booktabs}
\usepackage{tabularx}
\usepackage{hyperref}
\usepackage{pdflscape}
\usepackage{float}
\usepackage{adjustbox}
\usepackage{multirow}
\usepackage{enumitem}
\usepackage{longtable}
\usepackage[round]{natbib}
\usepackage[autostyle]{csquotes}

\hypersetup{
    colorlinks=true,       % false: boxed links; true: colored links
    linkcolor=red,          % color of internal links (change box color with linkbordercolor)
    citecolor=green,        % color of links to bibliography
    filecolor=magenta,      % color of file links
    urlcolor=cyan           % color of external links
}

\title{Hazard Analysis\\\progname}

\author{\authname}

\date{}

\input{../Comments}
%% Common Parts

\newcommand{\progname}{Grocery Spending Tracker} % PUT YOUR PROGRAM NAME HERE
\newcommand{\authname}{Team 1, JARS
\\ Jason Nam
\\ Allan Fang
\\ Ryan Yeh
\\ Sawyer Tang} % AUTHOR NAMES                  

\usepackage{hyperref}
    \hypersetup{colorlinks=true, linkcolor=blue, citecolor=blue, filecolor=blue,
                urlcolor=blue, unicode=false}
    \urlstyle{same}
                                


\begin{document}

\maketitle
\thispagestyle{empty}

~\newpage

\pagenumbering{roman}

\begin{table}[hp]
\caption{Revision History} \label{TblRevisionHistory}
\begin{tabularx}{\textwidth}{llX}
\toprule
\textbf{Date} & \textbf{Developer(s)} & \textbf{Change}\\
\midrule
16/10/23 & Allan Fang & Added sections 1, 2, 3\\
Date2 & Name(s) & Description of changes\\
... & ... & ...\\
\bottomrule
\end{tabularx}
\end{table}

~\newpage

\tableofcontents

~\newpage

\pagenumbering{arabic}

\wss{You are free to modify this template.}

\section{Introduction}
This document details the hazards and the associated hazard controls for the Grocery Spending Tracker system to mitigate and eliminate unsafe behaviour. The Grocery Spending Tracker will assist users in making smart grocery decisions to better manage household and grocery expenses.  A hazard is defined as some system condition(s) that, together with some environmental condition(s) has the potential to cause loss of some value to stakeholders. A hazard control is a measure to mitigate the hazard.

\section{Scope and Purpose of Hazard Analysis}
The scope of the document is to identify possible hazards within the system components, the steps to mitigate the hazards, and the resulting safety and security requirements.

\section{System Boundaries and Components}
The system that this hazard analysis will be conducted on consists of the following components:
\begin{itemize}
    \item The front-end and back-end components of the application:
        \begin{itemize}
            \item Authentication
            \item Purchase data input (including optical recognition and language models)
            \item Nearby alternatives recommendations
            \item Purchase history reports
            \item Spending objectives
        \end{itemize}
    \item User mobile device
    \item The database storing all application data
\end{itemize}

\section{Critical Assumptions}

\wss{These assumptions that are made about the software or system.  You should
minimize the number of assumptions that remove potential hazards.  For instance,
you could assume a part will never fail, but it is generally better to include
this potential failure mode.}

\newpage
\begin{landscape}
    \section{Failure Mode and Effect Analysis}
    \begin{longtable}{|p{0.15\textwidth}|p{0.15\textwidth}|p{0.25\textwidth}|p{0.25\textwidth}|p{0.35\textwidth}|p{0.1\textwidth}|p{0.05\textwidth}|}
        \caption{Failure Mode and Effect Analysis Table}  \\
		\hline
        \multicolumn{1}{|c|}{\textbf{Component}}
		& \multicolumn{1}{|c|}{\textbf{Failure Modes}}
		& \multicolumn{1}{|c|}{\textbf{Effects of Failure}}
		& \multicolumn{1}{|c|}{\textbf{Causes of Failure}}
		& \multicolumn{1}{|c|}{\textbf{Recommended Action}}
		& \multicolumn{1}{|c|}{\textbf{SR}}
		& \multicolumn{1}{|c|}{\textbf{Ref.}}  \\
		\hline
    \end{longtable}
\end{landscape}    

\wss{Include your FMEA table here}

\section{Safety and Security Requirements}

\wss{Newly discovered requirements.  These should also be added to the SRS.  (A
rationale design process how and why to fake it.)}

\section{Roadmap}

\wss{Which safety requirements will be implemented as part of the capstone timeline?
Which requirements will be implemented in the future?}

\end{document}