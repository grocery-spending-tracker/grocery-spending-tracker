\documentclass[12pt, titlepage]{article}

\usepackage{booktabs}
\usepackage{tabularx}
\usepackage{float}
\usepackage{hyperref}
\hypersetup{
    colorlinks,
    citecolor=blue,
    filecolor=black,
    linkcolor=red,
    urlcolor=blue
}
\usepackage[round]{natbib}

%% Comments

\usepackage{color}

\newif\ifcomments\commentstrue %displays comments
%\newif\ifcomments\commentsfalse %so that comments do not display

\ifcomments
\newcommand{\authornote}[3]{\textcolor{#1}{[#3 ---#2]}}
\newcommand{\todo}[1]{\textcolor{red}{[TODO: #1]}}
\else
\newcommand{\authornote}[3]{}
\newcommand{\todo}[1]{}
\fi

\newcommand{\wss}[1]{\authornote{blue}{SS}{#1}} 
\newcommand{\plt}[1]{\authornote{magenta}{TPLT}{#1}} %For explanation of the template
\newcommand{\an}[1]{\authornote{cyan}{Author}{#1}}

%% Common Parts

\newcommand{\progname}{ProgName} % PUT YOUR PROGRAM NAME HERE
\newcommand{\authname}{Team \#, Team Name
\\ Student 1 name
\\ Student 2 name
\\ Student 3 name
\\ Student 4 name} % AUTHOR NAMES                  

\usepackage{hyperref}
    \hypersetup{colorlinks=true, linkcolor=blue, citecolor=blue, filecolor=blue,
                urlcolor=blue, unicode=false}
    \urlstyle{same}
                                


\begin{document}

\title{Project Title: System Verification and Validation Plan for \progname{}} 
\author{\authname}
\date{\today}
	
\maketitle

\pagenumbering{roman}

\section*{Revision History}

\begin{tabularx}{\textwidth}{p{3cm}p{2cm}X}
\toprule {\bf Date} & {\bf Version} & {\bf Notes}\\
\midrule
Date 1 & 1.0 & Notes\\
Date 2 & 1.1 & Notes\\
\bottomrule
\end{tabularx}

~\\
\wss{The intention of the VnV plan is to increase confidence in the software.
However, this does not mean listing every verification and validation technique
that has ever been devised.  The VnV plan should also be a \textbf{feasible}
plan. Execution of the plan should be possible with the time and team available.
If the full plan cannot be completed during the time available, it can either be
modified to ``fake it'', or a better solution is to add a section describing
what work has been completed and what work is still planned for the future.}

\wss{The VnV plan is typically started after the requirements stage, but before
the design stage.  This means that the sections related to unit testing cannot
initially be completed.  The sections will be filled in after the design stage
is complete.  the final version of the VnV plan should have all sections filled
in.}

\newpage

\tableofcontents

\listoftables
\wss{Remove this section if it isn't needed}

\listoffigures
\wss{Remove this section if it isn't needed}

\newpage

\section{Symbols, Abbreviations, and Acronyms}

This section will include any symbols, abbreviations and acronyms
used in this document and their definitions. \\

\renewcommand{\arraystretch}{1.2}
\begin{tabular}{l l} 
  \toprule		
  \textbf{symbol} & \textbf{description}\\
  \midrule 
  SRS & Software Requirements Specification\\
  OCR & Optical Character Recognition\\
  HA & Hazard Analysis\\
  VnV & Verification and Validation\\
  MG & Module Guide\\
  MIS & Module Interface Specification\\
  DP & Development Plan\\
  \bottomrule
\end{tabular}\\

\newpage

\pagenumbering{arabic}

This document outlines the verification and validation plan for the Grocery Spending Tracker.
The document will cover some basic information about the application as well as go over the plans
the team has for testing its requirements and features throughout development. It will also cover
the different system tests and unit tests that the team will employ to go about validating
the software.

\section{General Information}

This section will give a general overview of the application as well as the objectives
the team has for performing validation and verification. This section will also include any
relevant document related to the overall testing plan.

\subsection{Summary}

The Grocery Spending Tracker is a mobile application being developed to help
users save money on groceries. The main feature is allowing users to input photos of receipts
which the application will then parse. Once parsed, the items and their corresponding prices will then be recorded.
This data will then be used to show analytics regarding their purchase history over time.
Additionally, the application will also allow users to get budgeting suggestions based on personal spending preferences
as well as recommendations for alternative nearby grocery stores where recently purchased items
can be obtained for cheaper.

\subsection{Objectives}

In regards to the application's primary functionality, testing will be done to build confidence
in the correctness and accuracy of all main features such as the OCR for receipt input as well as analytic
data being returned. This also holds for any budgeting suggestions and alternative grocery store suggestions.
Due to the nature of this application focusing on user finances, the accuracy and correctness of all
main features are paramount to this application's success.
Additionally, qualities that are important for an enjoyable user experience
will also be emphasized in this VnV Plan. Testing should verify that the application demonstrates
acceptable levels of usability and performance according to stakeholder expectations. \\

In general, a majority of the requirements set by the
\href{https://github.com/r-yeh/grocery-spending-tracker/blob/master/docs/SRS/SRS.pdf}{SRS} and
\href{https://github.com/r-yeh/grocery-spending-tracker/blob/master/docs/HazardAnalysis/HazardAnalysis.pdf}{HA} will be considered
during validation and verification. Maintainability, adaptability, and compliance tests however will be placed out of scope
due to them having less direct impact on the main user experience. Additionally, personalization settings
will be out of scope for testing as well due to them having less importance for the current build of the project. The primary focus during
this development period will be to build a prototype application with all primary features and qualities rather than a
full consumer product. With this in mind, these qualities seemed less relevant to this overall goal.
Additionally, several features will leverage external libraries as part of the overall workflow and these components will
be excluded from testing as well on a unit level. The team will assume that these libraries have already
been tested extensively by the original developers and other users.

\subsection{Relevant Documentation}

% \wss{Reference relevant documentation.  This will definitely include your SRS
%   and your other project documents (design documents, like MG, MIS, etc).  You
%   can include these even before they are written, since by the time the project
%   is done, they will be written.}

Listed below are the relevant documents referred to in this VnV Plan:

\begin{itemize}
  \item \href{https://github.com/r-yeh/grocery-spending-tracker/blob/master/docs/SRS/SRS.pdf}{Software Requirements Specification} \citet{GrocerySRS}\\
  The SRS contains the primary functional and non-functional requirements that are being used as the basis for many of the tests in this document.
  \item \href{https://github.com/r-yeh/grocery-spending-tracker/blob/master/docs/HazardAnalysis/HazardAnalysis.pdf}{Hazard Analysis} \citet{GroceryHA}\\
  The HA contains a number of safety requirements that were identified to help mitigate risk that were also considered for testing in this document.
  \item \href{https://github.com/r-yeh/grocery-spending-tracker/blob/master/docs/Design/SoftArchitecture/MG.pdf}{Module Guide}\\
  The MG references the different modules that make up the general overall application in which the VnV Plan tests.
  \item \href{https://github.com/r-yeh/grocery-spending-tracker/blob/master/docs/Design/SoftDetailedDes/MIS.pdf}{Module Interface Specification}\\
  The MIS contains detailed information on each of the application modules which form the basis for the unit tests.
  \item \href{https://github.com/r-yeh/grocery-spending-tracker/blob/master/docs/DevelopmentPlan/DevelopmentPlan.pdf}{Development Plan} \citet{GroceryDP}\\
  The DP describes a roadmap for the project including timelines, communication plans, and technical details.
\end{itemize}

\section{Plan}

The following will outline a series of plans and processes that will be used for the verification and validation of the software solution and documentation the Grocery Spending Tracker project. The sections are as follows:
\begin{itemize}
    \item Verification and Validation Team: the team members their involvement in the verification and validation efforts 
    \item SRS Verification Plan: the approaches to be used for SRS verification
    \item Design Verification Plan: plans for design verification
    \item Verification and Validation Plan Verification Plan: plans for VnV verification
    \item Implementation Verification Plan: dynamic and static tests for verification of the implementation
    \item Automated Testing and Verification Tools: tools and metrics to be used for testing and verification of the implementation
    \item Software Validation Plan: plans for verification of the software requirements
\end{itemize}
The stakeholders referred to in this document are individuals who are members of low-income households in Hamilton and students from McMaster University as described in Section 2 of the \href{https://github.com/r-yeh/grocery-spending-tracker/blob/master/docs/SRS/SRS.pdf}{SRS}. As the primary client, their involvement in the VnV ensures that the system aligns with the needs and goals of the intended users.

\subsection{Verification and Validation Team}
Developers are responsible for creating, executing, and documenting the tests for verifying and validating the project. This includes SRS verification, design verification, VnV verification, system tests, and unit tests. Task distribution will be flexible and distributed based on available resources. Individuals will also have areas of focus (listed below) where they will take leadership in to minimize fragmented results. 
\begin{table}[H]
    \centering
    \begin{tabularx}{\textwidth}{|>{\hsize=.36\hsize}X|>{\hsize=1.64\hsize}X|}
        \hline
        \textbf{Name} & \textbf{Role}\\
        \hline
        Allan Fang & Developer: SRS, VnV, Design Verification\\
        \hline
        Jason Nam & Developer: SRS, NFR System Tests, FR System Tests\\
        \hline
        Ryan Yeh & Developer: Unit Tests, Software Validation\\
        \hline
        Sawyer Tang & Developer: FR System Tests, Unit Tests\\
        \hline
    \end{tabularx}
    \caption{VnV Team}
    \label{tab:vnvteam}
\end{table}

\subsection{SRS Verification Plan}
The design documentation will be verified with an informal walk-through by stakeholders and peers of the SFWRENG 4G06 course who will provide ad-hoc review as well as an inspection by the development team. The inspection will address the following issues and be tracked by the checklist:
\begin{itemize}
    \item Correctness: the \href{https://github.com/r-yeh/grocery-spending-tracker/blob/master/docs/SRS/SRS.pdf}{SRS} accurately represents the software expectations
    \item Ambiguity: the requirements and contents convey an unambiguous meaning
    \item Completeness: the SRS covers all necessary functional and non-functional requirements
    \item Verifiability: the requirements are testable and quantifiable
    \item Traceability: the requirements have clear source references
    \item Clarity: the document is written in a clear, concise, and unambiguous manner
\end{itemize}

\subsection{Design Verification Plan}
The design documentation will be verified with an informal walk-through by stakeholders and peers of the SFWRENG 4G06 course who will provide ad-hoc review as well as an inspection by the development team. The review will address the following issues and be guided by the checklist:
\begin{itemize}
    \item Design adequacy and conformance to software requirements: assess whether the design adequately meets the specified requirements (\href{https://github.com/r-yeh/grocery-spending-tracker/blob/master/docs/SRS/SRS.pdf}{SRS})
    \item Internal consistency: different components of the design are coherent
    \item External consistency: the design aligns with external interfaces, standards, and environments
    \item Feasibility: whether the design can be practically realized within technical and resource constraints
    \item Maintenance: whether the design can be easily updated, modified, or repaired
\end{itemize}

\subsection{Verification and Validation Plan Verification Plan}
The VnV plan will be verified with an informal walk-through by stakeholders and peers of the SFWRENG 4G06 course who will provide ad-hoc review as well as an inspection by the development team. The reviews will address the following issues and be guided by the checklist:
\begin{itemize}
    \item Completeness: the VnV plan covers all necessary documentation, functional requirements, and non-functional requirements
    \item Feasibility: whether the actions that the VnV plan details can be practically realized within technical and resource constraints
    \item Clarity: the document is written in a clear, concise, and unambiguous manner
\end{itemize}

\subsection{Implementation Verification Plan}
The verification of the implementation will be done with static and dynamic tests as well as surveys and interviews with stakeholders. Dynamic tests will be done according to sections 4 and 5 of this document. For static tests, the development team will conduct a code walk-through to check for incorrect code along with using the linter and coding standards detailed in Section 3.6.

\subsection{Automated Testing and Verification Tools}
Refer to sections 6 and 7 of the \href{https://github.com/r-yeh/grocery-spending-tracker/blob/master/docs/DevelopmentPlan/DevelopmentPlan.pdf}{Development Plan} for the automated testing and verification tools the team will be using.

\subsection{Software Validation Plan}
A task-based inspection process with stakeholders will be used to validate the software. The inspection will be conducted according to the business events and user flows described in sections 6 and 8 of the \href{https://github.com/r-yeh/grocery-spending-tracker/blob/master/docs/SRS/SRS.pdf}{SRS}.

\section{System Test Description}
	
\subsection{Tests for Functional Requirements}

\wss{Subsets of the tests may be in related, so this section is divided into
  different areas.  If there are no identifiable subsets for the tests, this
  level of document structure can be removed.}

\wss{Include a blurb here to explain why the subsections below
  cover the requirements.  References to the SRS would be good here.}

\subsubsection{Area of Testing1}

\wss{It would be nice to have a blurb here to explain why the subsections below
  cover the requirements.  References to the SRS would be good here.  If a section
  covers tests for input constraints, you should reference the data constraints
  table in the SRS.}
		
\paragraph{Title for Test}

\begin{enumerate}

\item{test-id1\\}

Control: Manual versus Automatic
					
Initial State: 
					
Input: 
					
Output: \wss{The expected result for the given inputs}

Test Case Derivation: \wss{Justify the expected value given in the Output field}
					
How test will be performed: 
					
\item{test-id2\\}

Control: Manual versus Automatic
					
Initial State: 
					
Input: 
					
Output: \wss{The expected result for the given inputs}

Test Case Derivation: \wss{Justify the expected value given in the Output field}

How test will be performed: 

\end{enumerate}

\subsubsection{Area of Testing2}

...

\subsection{Tests for Nonfunctional Requirements}

\wss{The nonfunctional requirements for accuracy will likely just reference the
  appropriate functional tests from above.  The test cases should mention
  reporting the relative error for these tests.  Not all projects will
  necessarily have nonfunctional requirements related to accuracy}

\wss{Tests related to usability could include conducting a usability test and
  survey.  The survey will be in the Appendix.}

\wss{Static tests, review, inspections, and walkthroughs, will not follow the
format for the tests given below.}

\subsubsection{Area of Testing1}
		
\paragraph{Title for Test}

\begin{enumerate}

\item{test-id1\\}

Type: Functional, Dynamic, Manual, Static etc.
					
Initial State: 
					
Input/Condition: 
					
Output/Result: 
					
How test will be performed: 
					
\item{test-id2\\}

Type: Functional, Dynamic, Manual, Static etc.
					
Initial State: 
					
Input: 
					
Output: 
					
How test will be performed: 

\end{enumerate}

\subsubsection{Area of Testing2}

...

\subsection{Traceability Between Test Cases and Requirements}

\wss{Provide a table that shows which test cases are supporting which
  requirements.}

\section{Unit Test Description}

\wss{This section should not be filled in until after the MIS (detailed design
  document) has been completed.}

\wss{Reference your MIS (detailed design document) and explain your overall
philosophy for test case selection.}  

\wss{To save space and time, it may be an option to provide less detail in this section.  
For the unit tests you can potentially layout your testing strategy here.  That is, you 
can explain how tests will be selected for each module.  For instance, your test building 
approach could be test cases for each access program, including one test for normal behaviour 
and as many tests as needed for edge cases.  Rather than create the details of the input 
and output here, you could point to the unit testing code.  For this to work, you code 
needs to be well-documented, with meaningful names for all of the tests.}

\subsection{Unit Testing Scope}

\wss{What modules are outside of the scope.  If there are modules that are
  developed by someone else, then you would say here if you aren't planning on
  verifying them.  There may also be modules that are part of your software, but
  have a lower priority for verification than others.  If this is the case,
  explain your rationale for the ranking of module importance.}

\subsection{Tests for Functional Requirements}

\wss{Most of the verification will be through automated unit testing.  If
  appropriate specific modules can be verified by a non-testing based
  technique.  That can also be documented in this section.}

\subsubsection{Module 1}

\wss{Include a blurb here to explain why the subsections below cover the module.
  References to the MIS would be good.  You will want tests from a black box
  perspective and from a white box perspective.  Explain to the reader how the
  tests were selected.}

\begin{enumerate}

\item{test-id1\\}

Type: \wss{Functional, Dynamic, Manual, Automatic, Static etc. Most will
  be automatic}
					
Initial State: 
					
Input: 
					
Output: \wss{The expected result for the given inputs}

Test Case Derivation: \wss{Justify the expected value given in the Output field}

How test will be performed: 
					
\item{test-id2\\}

Type: \wss{Functional, Dynamic, Manual, Automatic, Static etc. Most will
  be automatic}
					
Initial State: 
					
Input: 
					
Output: \wss{The expected result for the given inputs}

Test Case Derivation: \wss{Justify the expected value given in the Output field}

How test will be performed: 

\item{...\\}
    
\end{enumerate}

\subsubsection{Module 2}

...

\subsection{Tests for Nonfunctional Requirements}

\wss{If there is a module that needs to be independently assessed for
  performance, those test cases can go here.  In some projects, planning for
  nonfunctional tests of units will not be that relevant.}

\wss{These tests may involve collecting performance data from previously
  mentioned functional tests.}

\subsubsection{Module ?}
		
\begin{enumerate}

\item{test-id1\\}

Type: \wss{Functional, Dynamic, Manual, Automatic, Static etc. Most will
  be automatic}
					
Initial State: 
					
Input/Condition: 
					
Output/Result: 
					
How test will be performed: 
					
\item{test-id2\\}

Type: Functional, Dynamic, Manual, Static etc.
					
Initial State: 
					
Input: 
					
Output: 
					
How test will be performed: 

\end{enumerate}

\subsubsection{Module ?}

...

\subsection{Traceability Between Test Cases and Modules}

\wss{Provide evidence that all of the modules have been considered.}
				
\bibliographystyle{plainnat}

\bibliography{../../refs/References}

\newpage

\section{Appendix}

This is where you can place additional information.

\subsection{Symbolic Parameters}

The definition of the test cases will call for SYMBOLIC\_CONSTANTS.
Their values are defined in this section for easy maintenance.

\subsection{Usability Survey Questions?}

\wss{This is a section that would be appropriate for some projects.}

\newpage{}
\section*{Appendix --- Reflection}

The information in this section will be used to evaluate the team members on the
graduate attribute of Lifelong Learning.  Please answer the following questions:

\newpage{}
\section*{Appendix --- Reflection}

The information in this section will be used to evaluate the team members on the
graduate attribute of Lifelong Learning.  Please answer the following questions:

\begin{enumerate}
  \item \textbf{What knowledge and skills will the team collectively need to acquire to
  successfully complete the verification and validation of your project?
  Examples of possible knowledge and skills include dynamic testing knowledge,
  static testing knowledge, specific tool usage etc.  You should look to
  identify at least one item for each team member.}

  For the verification and validation of this project, several skills will be required
  to ensure that this project is well-tested and meets all requirements. This will involve
  some general testing knowledge as well as some knowledge specific to the technologies we
  are using. By learning the following, the team will be able to produce a better project for this
  capstone in addition to more confidence in the testing process in our careers.
  \begin{enumerate}
    \item Knowledge of the different forms and approaches of dynamic testing
    \item Knowledge of the different forms and approaches of static testing
    \item Experience and knowledge working with \textit{Flutter's} internal unit testing framework and
    \textit{Flutter Performance Profiling} for frontend performance
    \item Experience and knowledge using \textit{Postman} for backend tests
    \item Communication skills to write surveys and elicit feedback from stakeholders/testers
  \end{enumerate}

  \item \textbf{For each of the knowledge areas and skills identified in the previous
  question, what are at least two approaches to acquiring the knowledge or
  mastering the skill?  Of the identified approaches, which will each team
  member pursue, and why did they make this choice?}

  For skills \textit{(a)} and \textit{(b)}, one approach that could be taken is to review notes from
  \textit{SFWRENG 3S03} where topics such as dynamic and static testing were covered. The different
  approaches for dynamic and static testing were also part of the lecture material which would be useful to go over again.
  Additionally, articles can also be found on the internet covering the basics of these
  testing approaches. \par

  Skills \textit{(c)} and \textit{(d)} involve more specific technologies, so one approach to learning these
  could be to read documentation. For both technologies, documentation is present to learn how the
  developers had intended them to be used. Additionally, \textit{YouTube} tutorials can be referenced to
  go over the basics and learn these technologies. Considering how many tutorials are made by users, it
  could provide more insight on how these applications can be used in a more practical setting.
  Lastly, third-party articles can also be used for additional learning. \par

  Finally, skill \textit{(e)} is more of a social skill rather than a technical skill compared to the other skills listed.
  One way to improve communication skills for testing could be with group collaboration and
  interaction. For instance, before using a survey, getting it reviewed by the team could result in flaws or issues
  being identified which can be noted for the future. Furthermore, \textit{SFWRENG 3S03} had some lectures on surveys
  (such as the kinds of questions to ask and not ask) and similar testing approaches that could also be reviewed to better
  learn and improve this skill.

  Below are the chosen skills from each team member, their rationale, and approaches they will use to learn or
  improve said skills.

  \textbf{Ryan Yeh}\\
  \textit{Chosen Skills: (c), (d), (e)} \\
  For verification and validation, I will be taking a leadership role in designing unit tests and software
  validation. Therefore, I want to ensure familiarity with \textit{Flutter's} testing technology and \textit{Postman}
  given that unit tests will primarily utilize these technologies. By ensuring proper learning of
  this software, I should be able to better guide the rest of the team during unit test development resulting
  in a stronger test suite. In addition, I also chose communication skills as a skill to focus on since this application
  is primarily user-focused. Therefore, ensuring proper user feedback and testing with stakeholders will be very important
  to this app's success. \par
  To learn and improve my skills with \textit{Flutter's} testing software and \textit{Postman}, I will refer to the
  official documentation and \textit{YouTube} tutorials. I feel this will give me a balance between original developer
  knowledge and practical user knowledge. In terms of communication skills, I will be employing group feedback
  and collaboration to ensure high-quality questions are being asked to maximize the feedback and information we receive
  from stakeholders and testers.

  \medskip
  \textbf{Sawyer Tang}\\
  \textit{Chosen Skills: } \\

  \medskip
  \textbf{Jason Nam}\\
  \textit{Chosen Skills: }\\

  \medskip
  \textbf{Allan Fang}\\
  \textit{Chosen Skills: } \\
\end{enumerate}

\end{document}