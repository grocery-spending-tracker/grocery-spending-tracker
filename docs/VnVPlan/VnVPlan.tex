\documentclass[12pt, titlepage]{article}

\usepackage{booktabs}
\usepackage{tabularx}
\usepackage{float}
\usepackage{hyperref}
\hypersetup{
    colorlinks,
    citecolor=blue,
    filecolor=black,
    linkcolor=red,
    urlcolor=blue
}
\usepackage[round]{natbib}
\usepackage{pdflscape}
\usepackage{longtable}
\usepackage{multirow}

%% Comments

\usepackage{color}

\newif\ifcomments\commentstrue %displays comments
%\newif\ifcomments\commentsfalse %so that comments do not display

\ifcomments
\newcommand{\authornote}[3]{\textcolor{#1}{[#3 ---#2]}}
\newcommand{\todo}[1]{\textcolor{red}{[TODO: #1]}}
\else
\newcommand{\authornote}[3]{}
\newcommand{\todo}[1]{}
\fi

\newcommand{\wss}[1]{\authornote{blue}{SS}{#1}} 
\newcommand{\plt}[1]{\authornote{magenta}{TPLT}{#1}} %For explanation of the template
\newcommand{\an}[1]{\authornote{cyan}{Author}{#1}}

%% Common Parts

\newcommand{\progname}{ProgName} % PUT YOUR PROGRAM NAME HERE
\newcommand{\authname}{Team \#, Team Name
\\ Student 1 name
\\ Student 2 name
\\ Student 3 name
\\ Student 4 name} % AUTHOR NAMES                  

\usepackage{hyperref}
    \hypersetup{colorlinks=true, linkcolor=blue, citecolor=blue, filecolor=blue,
                urlcolor=blue, unicode=false}
    \urlstyle{same}
                                


\begin{document}

\title{System Verification and Validation Plan for \progname{}} 
\author{\authname}
\date{\today}
	
\maketitle

\pagenumbering{roman}

\section*{Revision History}

\begin{tabularx}{\textwidth}{p{3cm}p{2cm}X}
\toprule {\bf Date} & {\bf Version} & {\bf Notes}\\
\midrule
03/11/23 & 1.0 & Created initial version of the VnV Plan\\
04/03/24 & 1.5 & Modified surveys to better algin with current version of the application\\
\bottomrule
\end{tabularx}

~\\

\newpage

\tableofcontents

\listoftables

\newpage

\section{Symbols, Abbreviations, and Acronyms}

This section will include any symbols, abbreviations and acronyms
used in this document and their definitions. \\

\renewcommand{\arraystretch}{1.2}
\begin{tabular}{l l} 
  \toprule		
  \textbf{symbol} & \textbf{description}\\
  \midrule 
  SRS & Software Requirements Specification\\
  OCR & Optical Character Recognition\\
  HA & Hazard Analysis\\
  VnV & Verification and Validation\\
  MG & Module Guide\\
  MIS & Module Interface Specification\\
  DP & Development Plan\\
  \bottomrule
\end{tabular}\\

\newpage

\pagenumbering{arabic}

This document outlines the verification and validation plan for the Grocery Spending Tracker.
The document will cover some basic information about the application as well as go over the plans
the team has for testing its requirements and features throughout development. It will also cover
the different system tests and unit tests that the team will employ to go about validating
the software.

\section{General Information}

This section will give a general overview of the application as well as the objectives
the team has for performing validation and verification. This section will also include any
relevant document related to the overall testing plan.

\subsection{Summary}

The Grocery Spending Tracker is a mobile application being developed to help
users save money on groceries. The main feature is allowing users to input photos of receipts
which the application will then parse. Once parsed, the items and their corresponding prices will then be recorded.
This data will then be used to show analytics regarding their purchase history over time.
Additionally, the application will also allow users to get budgeting suggestions based on personal spending preferences
as well as recommendations for alternative nearby grocery stores where recently purchased items
can be obtained for cheaper.

\subsection{Objectives}

In regards to the application's primary functionality, testing will be done to build confidence
in the correctness and accuracy of all main features such as the OCR for receipt input as well as analytic
data being returned. This also holds for any budgeting suggestions and alternative grocery store suggestions.
Due to the nature of this application focusing on user finances, the accuracy and correctness of all
main features are paramount to this application's success.
Additionally, qualities that are important for an enjoyable user experience
will also be emphasized in this VnV Plan. Testing should verify that the application demonstrates
acceptable levels of usability and performance according to stakeholder expectations. \\

In general, a majority of the requirements set by the
\href{https://github.com/r-yeh/grocery-spending-tracker/blob/master/docs/SRS/SRS.pdf}{SRS} and
\href{https://github.com/r-yeh/grocery-spending-tracker/blob/master/docs/HazardAnalysis/HazardAnalysis.pdf}{HA} will be considered
during validation and verification. Maintainability, adaptability, and compliance tests however will be placed out of scope
due to them having less direct impact on the main user experience. Additionally, personalization settings
will be out of scope for testing as well due to them having less importance for the current build of the project. The primary focus during
this development period will be to build a prototype application with all primary features and qualities rather than a
full consumer product. With this in mind, these qualities seemed less relevant to this overall goal.
Additionally, several features will leverage external libraries as part of the overall workflow and these components will
be excluded from testing as well on a unit level. The team will assume that these libraries have already
been tested extensively by the original developers and other users.

\subsection{Relevant Documentation}

% \wss{Reference relevant documentation.  This will definitely include your SRS
%   and your other project documents (design documents, like MG, MIS, etc).  You
%   can include these even before they are written, since by the time the project
%   is done, they will be written.}

Listed below are the relevant documents referred to in this VnV Plan:

\begin{itemize}
  \item \href{https://github.com/r-yeh/grocery-spending-tracker/blob/master/docs/SRS/SRS.pdf}{Software Requirements Specification} \citet{GrocerySRS}\\
  The SRS contains the primary functional and non-functional requirements that are being used as the basis for many of the tests in this document.
  \item \href{https://github.com/r-yeh/grocery-spending-tracker/blob/master/docs/HazardAnalysis/HazardAnalysis.pdf}{Hazard Analysis} \citet{GroceryHA}\\
  The HA contains a number of safety requirements that were identified to help mitigate risk that were also considered for testing in this document.
  \item \href{https://github.com/r-yeh/grocery-spending-tracker/blob/master/docs/Design/SoftArchitecture/MG.pdf}{Module Guide}\\
  The MG references the different modules that make up the general overall application in which the VnV Plan tests.
  \item \href{https://github.com/r-yeh/grocery-spending-tracker/blob/master/docs/Design/SoftDetailedDes/MIS.pdf}{Module Interface Specification}\\
  The MIS contains detailed information on each of the application modules which form the basis for the unit tests.
  \item \href{https://github.com/r-yeh/grocery-spending-tracker/blob/master/docs/DevelopmentPlan/DevelopmentPlan.pdf}{Development Plan} \citet{GroceryDP}\\
  The DP describes a roadmap for the project including timelines, communication plans, and technical details.
\end{itemize}

\section{Plan}

The following will outline a series of plans and processes that will be used for the verification and validation of the software solution and documentation the Grocery Spending Tracker project. The sections are as follows:
\begin{itemize}
    \item Verification and Validation Team: the team members their involvement in the verification and validation efforts 
    \item SRS Verification Plan: the approaches to be used for SRS verification
    \item Design Verification Plan: plans for design verification
    \item Verification and Validation Plan Verification Plan: plans for VnV verification
    \item Implementation Verification Plan: dynamic and static tests for verification of the implementation
    \item Automated Testing and Verification Tools: tools and metrics to be used for testing and verification of the implementation
    \item Software Validation Plan: plans for verification of the software requirements
\end{itemize}
The stakeholders referred to in this document are individuals who are members of low-income households in Hamilton and students from McMaster University as described in Section 2 of the \href{https://github.com/r-yeh/grocery-spending-tracker/blob/master/docs/SRS/SRS.pdf}{SRS}. As the primary client, their involvement in the VnV ensures that the system aligns with the needs and goals of the intended users.

\subsection{Verification and Validation Team}
Developers are responsible for creating, executing, and documenting the tests for verifying and validating the project. This includes SRS verification, design verification, VnV verification, system tests, and unit tests. Task distribution will be flexible and distributed based on available resources. Individuals will also have areas of focus (listed below) where they will take leadership in to minimize fragmented results. 
\begin{table}[H]
    \centering
    \begin{tabularx}{\textwidth}{|>{\hsize=.36\hsize}X|>{\hsize=1.64\hsize}X|}
        \hline
        \textbf{Name} & \textbf{Role}\\
        \hline
        Allan Fang & Developer: SRS, VnV, Design Verification\\
        \hline
        Jason Nam & Developer: SRS, NFR System Tests, FR System Tests\\
        \hline
        Ryan Yeh & Developer: Unit Tests, Software Validation\\
        \hline
        Sawyer Tang & Developer: FR System Tests, Unit Tests\\
        \hline
    \end{tabularx}
    \caption{VnV Team}
    \label{tab:vnvteam}
\end{table}

\subsection{SRS Verification Plan}
The design documentation will be verified with an informal walk-through by stakeholders and peers of the SFWRENG 4G06 course who will provide ad-hoc review as well as an inspection by the development team. The inspection will address the following issues and be tracked by the checklist:
\begin{itemize}
    \item Correctness: the \href{https://github.com/r-yeh/grocery-spending-tracker/blob/master/docs/SRS/SRS.pdf}{SRS} accurately represents the software expectations
    \item Ambiguity: the requirements and contents convey an unambiguous meaning
    \item Completeness: the SRS covers all necessary functional and non-functional requirements
    \item Verifiability: the requirements are testable and quantifiable
    \item Traceability: the requirements have clear source references
    \item Clarity: the document is written in a clear, concise, and unambiguous manner
\end{itemize}

\subsection{Design Verification Plan}
The design documentation will be verified with an informal walk-through by stakeholders and peers of the SFWRENG 4G06 course who will provide ad-hoc review as well as an inspection by the development team. The review will address the following issues and be guided by the checklist:
\begin{itemize}
    \item Design adequacy and conformance to software requirements: assess whether the design adequately meets the specified requirements (\href{https://github.com/r-yeh/grocery-spending-tracker/blob/master/docs/SRS/SRS.pdf}{SRS})
    \item Internal consistency: different components of the design are coherent
    \item External consistency: the design aligns with external interfaces, standards, and environments
    \item Feasibility: whether the design can be practically realized within technical and resource constraints
    \item Maintenance: whether the design can be easily updated, modified, or repaired
\end{itemize}

\subsection{Verification and Validation Plan Verification Plan}
The VnV plan will be verified with an informal walk-through by stakeholders and peers of the SFWRENG 4G06 course who will provide ad-hoc review as well as an inspection by the development team. The reviews will address the following issues and be guided by the checklist:
\begin{itemize}
    \item Completeness: the VnV plan covers all necessary documentation, functional requirements, and non-functional requirements
    \item Feasibility: whether the actions that the VnV plan details can be practically realized within technical and resource constraints
    \item Clarity: the document is written in a clear, concise, and unambiguous manner
\end{itemize}

\subsection{Implementation Verification Plan}
The verification of the implementation will be done with static and dynamic tests as well as surveys and interviews with stakeholders. Dynamic tests will be done according to Sections 4 and 5 of this document. For static tests, the development team will conduct a code walk-through to check for incorrect code along with using the linter and coding standards detailed in Section 3.6.

\subsection{Automated Testing and Verification Tools}
Refer to Sections 6 and 7 of the \href{https://github.com/r-yeh/grocery-spending-tracker/blob/master/docs/DevelopmentPlan/DevelopmentPlan.pdf}{Development Plan} for the automated testing and verification tools the team will be using.

\subsection{Software Validation Plan}
A task-based inspection process with stakeholders will be used to validate the software. The inspection will be conducted according to the business events and user flows described in Sections 6 and 8 of the \href{https://github.com/r-yeh/grocery-spending-tracker/blob/master/docs/SRS/SRS.pdf}{SRS}.

\section{System Test Description}

This section will provide an outline of the various system tests that will be performed
by the team during development. The system tests will primarily be based on the functional and nonfunctional requirements
recorded in the \href{https://github.com/r-yeh/grocery-spending-tracker/blob/master/docs/SRS/SRS.pdf}{SRS}.
	
\subsection{Tests for Functional Requirements}

\subsubsection{User Authentication}

The following sections pertain to all tests that have to do with the user logging into and gaining access to the system.

\paragraph{Authentication Interface}

These tests pertain to all the functional requirements related to the authentication interface.

\begin{enumerate}

\item{\textbf{FRT-FR1-1}}

Control: Manual
          
Initial State: User is not signed into the system.
          
Input: User attempts to sign into the system with the correct username and credentials.

Output: User is successfully signed in and on the application landing page in no more than three steps.

Test Case Derivation: If the login credentials are correct and match a user in the database, then login should be simple and successful.
          
How test will be performed: Tester will attempt to log into the application. Validate that it takes less than four steps to reach the landing page.

\item{\textbf{FRT-FR1-2}}

Control: Manual
          
How test will be performed: Survey is presented to the user for them to respond how easy it is to log into the system.
A success is determined by at least a 70\% positive response. The survey question is detailed in Section 6.2.1.

\end{enumerate}


\paragraph{Log-in and Authentication Credentials}

These tests pertain to all the functional requirements related to authentication credentials and the user logging in to the system.

\begin{enumerate}

\item{\textbf{FRT-FR2-1}}

Control: Manual
					
Initial State: User with no existing account on file.
					
Input: User signs up for account with valid name, email, phone-number, and password.
					
Output: User is directed to landing page. Account is successfully created and verification email is sent.

Test Case Derivation: As long as all fields required for sign up are filled with the appropriate data types and formats,
then sign up should be completed successfully.
					
How test will be performed: Tester will open the application and attempt to create a new account with the pertaining credentials and details. 
					
\item{\textbf{FRT-FR2-2}}

Control: Manual
					
Initial State: User with no existing account on file.
					
Input: User signs up for account with valid name, phone-number, and password however
an invalid email is used (i.e. not following the format of a prefix prior to the @ symbol and a domain afterwards).
					
Output: Error message should be displayed on the screen detailing an invalid email within 3 seconds.

Test Case Derivation: If incorrect/invalid data is used during sign up, the user should be notified and sign up should
not be allowed.

How test will be performed: Tester will open the application and attempt to create a new account with the pertaining credentials and details. 

\item{\textbf{FRT-FR2-3}}

Control: Manual
          
Initial State: User with no existing account on file.
          
Input: User signs up for account with valid name, email, and password however an
invalid phone number is used, for instance, a number less than 10 digits.
          
Output: Error message should be displayed on the screen detailing an invalid phone number within 3 seconds.

Test Case Derivation: If incorrect/invalid data is used during sign up, the user should be notified and sign up should
not be allowed.

How test will be performed: Tester will open the application and attempt to create a new account with the pertaining credentials and details. 

\item{\textbf{FRT-FR2-4}}

Control: Manual
          
Initial State: User with no existing account on file.
          
Input: User signs up for account with valid name, email, and phone-number however an invalid password
is used, such as an empty string.
          
Output: Error message should be displayed on the screen detailing an invalid password within 3 seconds.

Test Case Derivation: If incorrect/invalid data is used during sign up, the user should be notified and sign up should
not be allowed.

How test will be performed: Tester will open the application and attempt to create a new account with the pertaining credentials and details. 

\item{\textbf{FRT-FR2-5}}

Control: Manual
          
Initial State: User with existing account on file.
          
Input: User signs into their account with email and password combination that exists in the user database.
          
Output: User is directed to landing page.

Test Case Derivation: If a user signs in using credentials that match an existing user, sign in should be successful.
          
How test will be performed: Tester will open the application and attempt to sign into an account with the pertaining credentials.
          
\item{\textbf{FRT-FR2-6}}

Control: Manual
          
Initial State: User with existing account on file.
          
Input: User signs in using an email that is not stored in the user database.
          
Output: Error message displayed on the screen saying the email is not associated to an existing account. Prompt to create an account.

Test Case Derivation: The system should prevent login if the email is not associated with a user in the database.
          
How test will be performed: Tester will open the application and attempt to sign into an account with an email that does not have an account.

\item{\textbf{FRT-FR2-7}}

Control: Manual
          
Initial State: User with existing account on file.
          
Input: User signs into an account with an existing email and non-matching password to that user.
          
Output: Error message displayed on the screen saying the password is incorrect for the entered email. Prompt to reset password is highlighted within 3 seconds.

Test Case Derivation: If both email and password do not match the credentials set for a user in the database, sign in
should be rejected.
          
How test will be performed: Tester will open the application and attempt to sign into an account with the pertaining credentials.

\end{enumerate}

\subsubsection{Receipt Scanning}

These tests pertain to all the functional requirements related to the scanning of receipts and inputting grocery data.

\begin{enumerate}

\item{\textbf{FRT-FR3-1}}

Control: Manual
          
Initial State: User has physical receipt and the system has access to the device camera.
          
Input: A photo of a physical receipt. User takes photo of receipt with their device.
          
Output: User is provided with a list of all their grocery items and prices on the receipt as well as an option to manually enter or edit them.

Test Case Derivation: Assuming optimal conditions, the application should be able to capture the items and corresponding prices
of each. The user should then have the opportunity to review the data and make changes before submitting.
          
How test will be performed: In even lighting, tester will run the system and photograph a receipt. They will then validate that the systems interpretation of the photograph matches that of the physical receipt with above 90\% precision.

\item{\textbf{FRT-FR11-1}}

Control: Manual
          
Initial State: User has physical receipt and the system has access to the device camera.

Input: User scans a physical receipt in bad conditions.
          
Output: User is provided with a list of all the scanned grocery items and prices on the receipt with a strong recommendation to retake the photo in better conditions or manually enter or edit them.

Test Case Derivation: In poor lighting conditions where the receipt is not clearly visible by the camera, the application should still
attempt to scan the receipt but should prompt the user to retake the photo in better conditions to ensure accuracy of the data.
          
How test will be performed: In a dark environment, tester will run the system and photograph a receipt. They will then validate that the system prompts the user to retake the photo or manually review all scanned items.

\item{\textbf{FRT-FR11-2}}

Control: Manual
          
Initial State: User has no physical receipt.

Input: User prompts the system that they do not have a receipt to scan.
          
Output: User is prompted to manually enter their grocery items and prices.

Test Case Derivation: If the user does not have a usable receipt that they can scan, the system should allow users to manually
enter data if necessary.
          
How test will be performed: Tester will prompt the system interface that they have no receipt. Tester will validate they are able to manually enter their items.

\item{\textbf{FRT-FR11-3}}

Control: Manual
          
Initial State: The system does not have access to the device camera.

Input: User attempts to scan a receipt.
          
Output: User is prompted to give permissions for the application to access camera. If not given, user is asked to manually enter grocery items and prices.

Test Case Derivation: Without permissions, the application will not work properly. Therefore, the application should prompt the user to give
camera access and restrict users to manual receipt entry if not given.
          
How test will be performed: Tester will revoke application access to device camera and then attempt to scan a receipt. Verify that the system prompts the user for camera access and asks the user for manual input if not given.

\end{enumerate}

\subsubsection{Account Goals and Preferences}

These tests pertain to all the functional requirements related to the user's account preferences, goals, or details that affect the system's performance.

\paragraph{Location-based Recommendations}

These tests pertain to all the functional requirements related to the user's location and how it affects their grocery recommendations.

\begin{enumerate}

\item{\textbf{FRT-FR4-1}}

Control: Manual
          
Initial State: The system does not have access to the user's device location.

Input: User prevents location access for the application
          
Output: User is prompted that the system has no location access and given the option to manually enter their general home location and desired recommendation radius.

Test Case Derivation: If the application is not given access to the device's location, then the application should allow for
manual entry by the user to prevent feature disruption.
          
How test will be performed: Tester will revoke application access to location  and then attempt to receive location-based grocery recommendations. Validate that system provides prompt.

\item{\textbf{FRT-FR4-2}}

Control: Manual
          
Initial State: The system has access to the user's device location but user wants to manually set a different one.

Input: User changes preferences to use manually entered home location and recommendation radius rather than device location.
          
Output: System provides accurate location-based recommendations regarding the manually entered location.

Test Case Derivation: If the user wishes to use a manually set location rather than their device's location, then the
application should still function the same in both scenarios.
          
How test will be performed: Tester will go to preferences and opt to use manually provided home location and recommendation radius. Validate that the system is providing correct grocery recommendations for the manually inputted location and radius.

\item{\textbf{FRT-FR4-3}}

Control: Manual
          
Initial State: The system has access to the user's device location.

Input: System has location access on device.
          
Output: Application provides recommendations to the user for stored that are within the default radius of their location.

Test Case Derivation: Should the application have access to the device location, then all location preferences should be
relative to the device's active location.
          
How test will be performed: Tester will enable location services for the application and validate that the system is providing correct grocery recommendations for the current device location.

\end{enumerate}

\paragraph{User Objectives}

These tests pertain to all the functional requirements related to the user's goals and spending objectives.

\begin{enumerate}

\item{\textbf{FRT-FR5-1}}

Control: Manual
          
Initial State: The system has no spending goals and desired outcomes stored in the system.

Input: User navigates to the preferences of their application.
          
Output: The system provides a space for the user to input their spending goals.

Test Case Derivation: The system must provide an interface for the user to input their spending goals.
          
How test will be performed: Tester will navigate to the application preferences and validate that an interface exists for inputting spending goals.

\item{\textbf{FRT-FR5-2}}

Control: Manual
          
Initial State: The system has existing spending goals and desired outcomes stored in the system. The system displayed the current spending goals.

Input: User navigates to the preferences of their application and changes their spending goals.
          
Output: The system is updated to reflect the new spending goals.

Test Case Derivation: If spending goals are ever updated, then the application should take the new spending goals into account
for correctness.
          
How test will be performed: Tester will navigate to the application preferences and validate that the existing spending goals are displayed. They will then attempt to change the values of the spending goals and then save the changes.

\item{\textbf{FRT-FR6-1}}

Control: Manual
          
Initial State: The system has no budgeting constraints stored.

Input: User navigates to the preferences of their application.
          
Output: The system provides a space for the user to input their budgeting constraints.

Test Case Derivation: The system must provide an interface for the user to input their budgeting constraints.
          
How test will be performed: Tester will navigate to the application preferences and validate that an interface exists for inputting user budgeting constraints.

\item{\textbf{FRT-FR6-2}}

Control: Manual
          
Initial State: The system has existing budgeting constraints stored.

Input: User navigates to the preferences of their application and changes their budgeting constraints.
          
Output: The system displays the new budgeting constraints.

Test Case Derivation: If budgeting constraints are ever updated, then the application should take the new constraints into account
for correctness.

How test will be performed: Tester will navigate to the application preferences and validate that the existing goals are displayed. They will then attempt to change the budgeting constraints values and save the changes.

\end{enumerate}

\subsubsection{System Behaviour and Recommendations}

These tests pertain to all the functional requirements related to how the system responds to the data that is provided to it. This includes the recommendations provided to the user.

\paragraph{Receiving Grocery Data}

These tests pertain to all the functional requirements related to how the system shall treat and handle data it receives.

\begin{enumerate}

\item{\textbf{FRT-FR7-1}}

Control: Automatic
          
Initial State: Test database is empty.

Input: API calls from client (front-end) to back-end containing receipt data. Receipt data is comprised of an array of \textit{Item(string name, double price, string date, string purchaseLocation, int userID)}.
          
Output: Database contains correctly organized items. The user's database table should include all items they purchased.

Test Case Derivation: System back-end must accurately map and sort receipt entries provided, keeping a record of the user who purchased each item.
          
How test will be performed: Begin with an empty testing database. Test cases comprised of items and prices are sent through an API call to the system back-end by testing client. Then an additional API call will be called from the testing client against the system back-end to validate that the items and prices are correctly stored in the database. \textit{Postman} will be used to facilitate API calls. 

\item{\textbf{FRT-FR7-2}}

Control: Automatic
          
Initial State: Test database is populated.

Input: API calls from client (front-end) to back-end containing receipt data. Receipt data only contains items that are not already in the database. Receipt data is comprised of an array of \textit{Item(string name, double price, string date, string purchaseLocation, int userID)}.
          
Output: System back-end must accurately map and sort receipt entries provided, keeping a record of the user who purchased each item while preserving existing data.

Test Case Derivation: System back-end must accurately map and sort receipt entries provided, keeping a record of each item all users purchase.
          
How test will be performed: Begin with a populated testing database. Test cases comprised of items and prices are sent through an API call to the system back-end by testing client. Then an additional API call will be called from the testing client against the system back-end to validate that the items and prices are correctly stored in the database. \textit{Postman} will be used to facilitate API calls. 

\item{\textbf{FRT-FR7-3}}

Control: Automatic
          
Initial State: Test database is populated.

Input: API calls from client (front-end) to back-end containing receipt data. Data containing items that are the same as existing items in the database but with different name identifiers.
An example of this would be a difference of "Rtz Crackers" or "Ritz Cracker". Receipt data is comprised of an array of \textit{Item(string name, int cents, double price, string purchaseLocation, int userID)}.
          
Output: Database contains correctly organized items and understanding the relationship between like items. If like products are added to the database, the database correctly maps them to the existing items of the same type, even if the name string is different. The database also keeps track of which users purchase items.

Test Case Derivation: System back-end must accurately map and sort receipt entries provided, keeping a record of the user who purchased each item while preserving existing data.
          
How test will be performed: Begin with a populated testing database. Test cases comprised of items and prices are sent through an API call to the system back-end by testing client. Then an additional API call will be called from the testing client against the system back-end to validate that the items and prices are correctly stored in the database. \textit{Postman} will be used to facilitate API calls. 

\end{enumerate}

\paragraph{Grocery Recommendations}

These tests pertain to all the functional requirements related to the system's grocery recommendations based on grocery data received from user and their preferences.

\begin{enumerate}

\item{\textbf{FRT-FR8-1}}

Control: Manual
          
Initial State: User has set budget information.

Input: User submits a receipt containing commonly purchased items plus a new highly priced item that puts them over budget.
          
Output: The system should identify the newly purchased item as the cause for the user going over budget and will recommend that the user stop purchasing it. If available, the system will also recommend alternatives that are in-line with the user's budget.

Test Case Derivation: System must be aware of the users purchasing activity and budget. If the user ever goes over budget when entering
a receipt, the system should identify why they went over budget and offer some advice.
          
How test will be performed: Tester will input receipt meeting the above conditions. The tester will then validate that the new item is correctly identified as putting the user over budget and confirm if they get a recommendation for an alternative.

\item{\textbf{FRT-FR8-2}}

Control: Manual

How test will be performed: Survey is presented to the user for them to respond how useful the system recommendations are. A success is determined by at least a 70\% positive response. The survey question is detailed in Section 6.2.2.

\item{\textbf{FRT-FR9-1}}

Control: Manual
          
Initial State: The database has an entry for a grocery item that is purchased from a location X.

Input: User inputs a receipt containing the grocery item in the database but at a higher price from location Y.
          
Output: System provides recommendation to purchase item from location X.

Test Case Derivation: System must be aware of the users purchasing activity and recommend the user with cheaper alternative locations for
the same item.
          
How test will be performed: Tester will submit items purchased for a cheap price on one user account (A) and then submit the same item purchased at a greater price from a different store with another user account (B). Tester verifies that user B receives a recommendation from the system to go to the same place as user A.

\item{\textbf{FRT-FR9-2}}

Control: Manual
          
Initial State: The database has an entry for a grocery item that is purchased from a location X.

Input: User submits a receipt containing the grocery item in the database but at a cheaper price from location Y.
          
Output: The database should be updated such that the cheapest price, location, and purchase date for that item are equal to the input data. Location Y should then be recommended to users who purchase that item.

Test Case Derivation: System must be aware of the users purchasing activity and recommend the user with cheaper alternative locations for
the same item.
          
How test will be performed: Tester will submit an item purchased for some amount on one user account (A) and then submit the same item for a cheaper amount from a different store with another user account (B). Tester verifies that user A receives a recommendation from the system to go to the same place as user B.

\end{enumerate}

\paragraph{Grocery Items Interface}

These tests pertain to all the functional requirements related to the system's grocery item interface.

\begin{enumerate}

\item{\textbf{FRT-FR10-1}}

Control: Manual
          
Initial State: User has no spending data or submitted items.

Input: User submits a receipt.
          
Output: System provides an interface showing all purchased items along with price, location, and date-time.

Test Case Derivation: System must provide the user with all of the data that the user has provided the system such that the user can review their own purchasing data.
          
How test will be performed: Tester will submit a receipt and then verify to see that the items have been added in the purchases interface.

\item{\textbf{FRT-FR10-2}}

Control: Manual
          
Initial State: User has existing spending data and submitted items.

Input: User submits a receipt.
          
Output: System provides an interface showing all purchased items along with price, location, and date-time.

Test Case Derivation: System must provide the user with all of the data that the user has provided the system such that the user can review their own purchasing data.
          
How test will be performed: Tester will submit a receipt and then verify to see that the items have been added in the purchases interface along with all of the preexisting items.

\end{enumerate}

\subsection{Tests for Nonfunctional Requirements}

\subsubsection{Look and Feel Requirements}

\begin{enumerate}
  \item{NFRT-LF1\\}

  Type: Manual, Dynamic
            
  How test will be performed: Testers will be presented with the application and guided through navigation of various
  pages available and asked to take note of the aesthetics, appearance and feel of the application. A survey
  will then be provided to evaluate these qualities which can be found in Section 6.3 of this document.
            
  \item{NFRT-LF2\\}

  Type: Manual
            
  How test will be performed: Testers will be given a list of features for the application and a list of
  unlabelled icons corresponding to each of the features. They will be asked to match which feature
  they believe pairs with which icon.

  \item{NFRT-LF3\\}

  Type: Manual, Static

  How test will be performed: Walkthroughs will be performed on the code of each of the frontend pages
  using the \textit{Flutter Inspector} to check size of components and layouts. This will be to ensure no more than
  85\% of the page is occupied and a numerical balance is achieved in the layout.

  \item{NFRT-LF4\\}

  Type: Manual, Dynamic

  Initial State: The application is opened on a device and logged into a user's account with internet access available.

  Input: The user disconnects from the internet.

  Output: The application should notify the user that they do not have internet connection.

  How test will be performed: Testers will be asked to disconnect from the internet while the application is open.
\end{enumerate}

\subsubsection{Usability and Humanity Requirements}

\begin{enumerate}
  \item{NFRT-UH1\\}

  Type: Manual, Dynamic
            
  How test will be performed: Testers will be presented with the
  application and given a brief overview of its purpose. They will then be given 15 minutes to use the application
  under supervision by the team but without instruction. After that time period, the testers will be asked to input
  a receipt and navigate to the purchase analytics.

  \item{NFRT-UH2\\}

  Type: Manual, Dynamic
					
  Initial State: The application is opened on a device and logged into a user's account with internet access available.
					
  Input: The user attempts to input a receipt but cancels the process.
					
  Output: No changes should be made to the user's account data.
  
  How test will be performed: Testers will be given a receipt to input and asked to input it into the application.
  Testers will then be asked to cancel the input before the process completes.

  \item{NFRT-UH3\\}

  Type: Manual, Dynamic
					
  Initial State: The application is opened on a device and logged into a user's account with internet access available.
					
  Input: The user tries changing the application's font size.
					
  Output: The font size should be changed throughout the application without negatively impacting readability
  or general functionality.
  
  How test will be performed: Testers will be asked to change the font size to their desired size and
  navigate through the application to check usability.

  \item{NFRT-UH4\\}

  Type: Manual, Dynamic
					
  Initial State: The application is opened on a device and logged in to a user's account with internet access available.
					
  Input: The user changes the accessibility colour scheme.
					
  Output: The colour scheme should be changed throughout the application without negatively impacting readability
  or general functionality.
  
  How test will be performed: Testers will be asked to change the accessibility colour scheme to their desired setting and
  navigate through the application to check usability.

  \item{NFRT-UH5\\}

  Type: Manual, Dynamic
					
  Initial State: The application is opened on a device and logged in to a user's account with accessibility settings
  having been changed.
					
  Input: The user closes the application.
					
  Output: When re-opened, the application should still be logged into the user's account and accessibility
  settings should still be in place.
  
  How test will be performed: Following \textit{NFRT-UH4} and \textit{NFRT-UH5}, testers will be asked to
  close the application on their device while logged in and re-open it afterwards.

  \item{NFRT-UH6\\}

  Type: Manual

  How test will be performed: A survey will be presented to testers to evaluate the overall usability of the
  application after first going through \textit{NFRT-UH1}. Afterwards, the team will show them any remaining features
  they had not yet experienced and given the opportunity to freely experiment with the applications. A survey will then be
  provided to them. The questions asked will be as written in Section 6.4 of this document.
\end{enumerate}

\subsubsection{Performance Requirements}

\begin{enumerate}
  \item{NFRT-PR1\\}

  Type: Manual, Dynamic

  How test will be performed: All pages of the application will be navigated through while using the
  \textit{Flutter Performance Profiling} tools. Times for rendering and frontend loading will then be recorded
  to ensure that frontend performance is within expectation.

  \item{NFRT-PR2\\}

  Type: Automatic, Dynamic

  How test will be performed: Requests will be sent to all backend endpoints via \textit{Postman} and after
  all requests are sent, the times for each request will be recorded. The team will then confirm that
  the backend performs within expectation.

  \item{NFRT-PR3\\}

  Type: Automatic, Dynamic

  Initial State: The backend server is running and operational.

  Input: For all endpoints, incorrect data types and incorrect inputs (such as empty inputs) will be passed.

  Output: The server should have appropriate error handling where no invalid data is accepted and the server
  remains operational.

  How test will be performed: A test suite with invalid data for the major endpoints will be created with
  \textit{Postman}. The test suite will then be automatically executed by \textit{Postman} and the results
  will be analyzed after.

  \item{NFRT-PR4\\}
  
  Type: Manual, Dynamic

  How test will be performed: Based on tests performed for \textit{FRT-FR3-1}, the tester should have to
  modify and correct data scanned for less than 10\% of total scanned items.

  \item{NFRT-PR5\\}

  Type: Manual, Dynamic

  How test will be performed: For tests performed for \textit{FRT-FR10-1} and \textit{FRT-FR10-2}, the relative
  error of data such as average spending will be computed and should be less than 5\%.

  \item{NFRT-PR6\\}

  Type: Manual, Dynamic

  Initial State: A device does not have the application installed.

  Input: The application is installed onto the device.

  Output: The device storage should indicate that the application uses less than 1GB of storage.

  How test will be performed: The tester will install the application onto a device without the application currently
  installed.
\end{enumerate}

\subsubsection{Cultural Requirements}

\begin{enumerate}

\item{NFRT-CUR1\\}

Type: Manual, Dynamic
					
Initial State: The app is running and set to the default system of measurement (e.g., metric).
					
Input: The user selects their preferred units of measurement from the app settings.
					
Output: The app interface and displayed measurements change to the selected units of measurement as per the user's choice.
					
How test will be performed: The tester will change the units of measurement from the app settings and verify that the interface and displayed measurements switch to the selected units.
					
\item{NFRT-CUR2\\}

Type: Manual, Dynamic
					
Initial State: The app contains common purchase items in the product database.
					
Input: The user attempts to add a purchase item to purchase history that is not present in the database.
					
Output: The app successfully adds the new purchase item to the user's purchase history.
					
How test will be performed: The tester will attempt to add a product not present in the app product database to the purchase history and verify that the item is added without any issues.

\item{NFRT-CUR3\\}

Type: Manual, Dynamic
					
Initial State: The app is set to the default currency (e.g., CAD).
					
Input: The user selects the American currency in the app settings.
					
Output: The app logs all future purchases all in USD based on the user's selection and will not compare the prices of similar products nearby of a different currency.
					
How test will be performed: The tester will access the app's settings, change the currency to USD. The tester will then verify that all prices of future purchases are logged in USD and similar products priced with CAD are not compared.

\end{enumerate}


\subsubsection{Operational and Environmental Requirements}

\begin{enumerate}
\item{NFRT-OER1\\}

Type: Manual, Dynamic

Initial State: Application is not yet installed on the device.
                
Input: The user will install and run the application on various devices.
                
Output: The application will be successfully installed. The application is operational.
                
How test will be performed: The tester will install the application on various mobile devices running Android on version 13 (IASR2) and iOS on version 16 (IASR1). The tester will also test on various tablet devices running Android or iPadOS. The tester will confirm that all functionalities work as expected on each device by conducting variety of actions and system events (including logging in to an user account, input receipt data, analyzing spending analytics, etc.).

\item{NFRT-OER2\\}

Type: Manual, Dynamic

Initial State: The application is installed on a device that has an operable camera peripheral and has access to the internet.
                
Input: The user will attempt to use application features that require a camera and/or internet access.
                
Output: The application will successfully operate the induced features that require the use of camera and/or internet.
                
How test will be performed: The tester will test the application on devices with and without camera peripherals to validate (WER1). The tester will disconnect devices from the internet to test offline functionality. The tester will then reconnect the device to the internet to test online features (WER2). The tester will confirm that all functionalities work on devices that have access to camera peripherals and internet access.

\item{NFRT-OER3\\}

Type: Automatic, Dynamic, Static

Initial State: The application is ready for new releases or any updates.
                
Input: Test suites will be executed against new releases and updates.
                
Output: The application will identify defects. The team will resolve identified defects. In doing so, the application testing prior to new releases and updates will achieve 80\% code coverage.
                
How test will be performed: The tester will use automated testing tools (that can include \textit{Postman}, \textit{Flutter}, etc) to achieve 80\% code coverage (RER1). The tester will then perform issue tracking and resolution processes (RER2). The tester will organize a Final Demonstration (RER3) between March 18 and March 29 of 2024, involving stakeholders to validate the final product against requirements.

\end{enumerate}

\subsubsection{Security Requirements}

\begin{enumerate}
\item{NFRT-SR1\\}

Type: Manual, Dynamic

Initial State: User is not logged in or terms are not accepted.
                
Input: User tries to access the application's database.
                
Output: Access will be denied until the user accepts terms and services.
                
How test will be performed: Attempts will take place to access the database without accepting terms. The tester will confirm denial of access. Then, attempt access after accepting terms and services. The tester will confirm successful database access.

\item{NFRT-SR2\\}

Type: Manual, Dynamic

Initial State: Data is stored in the application database.
                
Input: User will attempt to modify the data directly from the application database.
                
Output: Stored data is encrypted. Therefore, modifications to the data are denied.
                
How test will be performed: The tester will attempt to use database tools to try to modify data within the application database. Then, the tester will confirm that the application backend database is secure and has not been modified.

\item{NFRT-SR3\\}

Type: Manual, Dynamic

Initial State: Application is in operation.
                
Input: User will attempt simulated attacks like SQL injection, cross-site scripting, etc.
                
Output: Application will remain secure without any data breaches.
                
How test will be performed: The tester will conduct a series of different types of simulated cyber attacks. The tester will then observe application response and monitor application database.

\item{NFRT-SR4\\}

Type: Manual, Dynamic

Initial State: Raw user data is stored in the application database.
                
Input: The raw user data is processed and then stored in the application database.
                
Output: The data is anonymized by data perturbation (inducing small amount of noise) and encryption. The data is also sanitized by data deletion or masking.
                
How test will be performed: The tester will inspect the stored user data for any personal identifiers. The tester will search for any presence of accessible sensitive user data.

\item{NFRT-SR5\\}

Type: Manual, Dynamic

Initial State: Application has yet to collect user data.
                
Input: User will start using the application.
                
Output: A consent form will pop up before any user data is collected. No user data will be collected without explicit consent given by user.
                
How test will be performed: The tester will check for prompts that ask users to consent to data collection. The tester will then test the application with and without user consent to ensure compliance. The tester will confirm no user data was collected and stored in the database without user consent.

\item{NFRT-SR6\\}

Type: Manual, Dynamic

Initial State: Application is in operation.
                
Input: User will attempt various actions to induce system events (including logging in to an user account, input receipt data, analyzing spending analytics, etc.).
                
Output: Activities and events will be accurately logged. The log can be reviewed by authorized administrators.
                
How test will be performed: The tester will perform a variety of user actions and try to trigger system events (including logging in to an user account, input receipt data, analyzing spending analytics, etc.). The tester will verify if these actions and events are accurately logged to system log. The tester will confirm that only authorized administrators can access these logs.

\end{enumerate}

\newpage
\begin{landscape}
  \subsection{Traceability Between Test Cases and Requirements}

  \begin{longtable}{|l|ccccccccccc|}
		\caption{Traceability Between Functional System Test Cases and Functional Requirements} \\
		\hline
    \textbf{Functional Requirement ID}   & \multicolumn{11}{c|}{\textbf{Test ID}} \\
    \hline
    ~ & \textbf{FR1} & \textbf{FR2} & \textbf{FR3} & \textbf{FR4} & \textbf{FR5} & \textbf{FR6} & \textbf{FR7} & \textbf{FR8} & \textbf{FR9} & \textbf{FR10} & \textbf{FR11} \\
    \hline
    FRT-FR1-1 & X & ~ & ~ & ~ & ~ & ~ & ~ & ~ & ~ & ~ & ~ \\
    FRT-FR1-2 & X & ~ & ~ & ~ & ~ & ~ & ~ & ~ & ~ & ~ & ~ \\
    FRT-FR2-1 & ~ & X & ~ & ~ & ~ & ~ & ~ & ~ & ~ & ~ & ~ \\
    FRT-FR2-2 & ~ & X & ~ & ~ & ~ & ~ & ~ & ~ & ~ & ~ & ~ \\
    FRT-FR2-3 & ~ & X & ~ & ~ & ~ & ~ & ~ & ~ & ~ & ~ & ~ \\
    FRT-FR2-4 & ~ & X & ~ & ~ & ~ & ~ & ~ & ~ & ~ & ~ & ~ \\
    FRT-FR2-5 & ~ & X & ~ & ~ & ~ & ~ & ~ & ~ & ~ & ~ & ~ \\
    FRT-FR2-6 & ~ & X & ~ & ~ & ~ & ~ & ~ & ~ & ~ & ~ & ~ \\
    FRT-FR2-7 & ~ & X & ~ & ~ & ~ & ~ & ~ & ~ & ~ & ~ & ~ \\
    FRT-FR3-1 & ~ & ~ & X & ~ & ~ & ~ & ~ & ~ & ~ & ~ & ~ \\
    FRT-FR11-1 & ~ & ~ & ~ & ~ & ~ & ~ & ~ & ~ & ~ & ~ & X \\
    FRT-FR11-2 & ~ & ~ & ~ & ~ & ~ & ~ & ~ & ~ & ~ & ~ & X \\
    FRT FR11-3 & ~ & ~ & ~ & ~ & ~ & ~ & ~ & ~ & ~ & ~ & X \\
    FRT-FR4-1 & ~ & ~ & ~ & X & ~ & ~ & ~ & ~ & ~ & ~ & ~ \\
    FRT-FR4-2 & ~ & ~ & ~ & X & ~ & ~ & ~ & ~ & ~ & ~ & ~ \\
    FRT-FR4-3 & ~ & ~ & ~ & X & ~ & ~ & ~ & ~ & ~ & ~ & ~ \\
    FRT-FR5-1 & ~ & ~ & ~ & ~ & X & ~ & ~ & ~ & ~ & ~ & ~ \\
    FRT-FR5-2 & ~ & ~ & ~ & ~ & X & ~ & ~ & ~ & ~ & ~ & ~ \\
    FRT-FR6-1 & ~ & ~ & ~ & ~ & ~ & X & ~ & ~ & ~ & ~ & ~ \\
    FRT-FR6-2 & ~ & ~ & ~ & ~ & ~ & X & ~ & ~ & ~ & ~ & ~ \\
    FRT-FR7-1 & ~ & ~ & ~ & ~ & ~ & ~ & X & ~ & ~ & ~ & ~ \\
    FRT-FR7-2 & ~ & ~ & ~ & ~ & ~ & ~ & X & ~ & ~ & ~ & ~ \\
    FRT-FR7-3 & ~ & ~ & ~ & ~ & ~ & ~ & X & ~ & ~ & ~ & ~ \\
    FRT-FR8-1 & ~ & ~ & ~ & ~ & ~ & ~ & ~ & X & X & ~ & ~ \\
    FRT-FR8-2 & ~ & ~ & ~ & ~ & ~ & ~ & ~ & X & ~ & ~ & ~ \\
    FRT-FR9-1 & ~ & ~ & ~ & ~ & ~ & ~ & ~ & ~ & X & ~ & ~ \\
    FRT-FR9-2 & ~ & ~ & ~ & ~ & ~ & ~ & ~ & ~ & X & ~ & ~ \\
    FRT-FR10-1 & ~ & ~ & ~ & ~ & ~ & ~ & ~ & ~ & ~ & X & ~ \\
    FRT-FR10-2 & ~ & ~ & ~ & ~ & ~ & ~ & ~ & ~ & ~ & X & ~ \\
    \hline
  \end{longtable}

  \newpage

  \begin{longtable}{|l|cccccc|}
		\caption{Traceability Between Nonfunctional System Test Cases and Look and Feel Requirements} \\
		\hline
    \textbf{NFR ID}   & \multicolumn{6}{c|}{\textbf{Test ID}} \\
    \hline
    ~ & \textbf{AR1} & \textbf{AR2} & \textbf{AR3} & \textbf{SR1} & \textbf{SR2} & \textbf{SR3} \\
    \hline
    NFRT-LF1 & X & ~ & X & X & X & X \\
    NFRT-LF2 & ~ & X & ~ & ~ & ~ & ~ \\
    NFRT-LF3 & ~ & ~ & X & ~ & ~ & ~ \\
    NFRT-LF4 & ~ & ~ & ~ & ~ & ~ & X \\
    \hline
  \end{longtable}

  \begin{longtable}{|l|ccccccccccc|}
		\caption{Traceability Between Nonfunctional System Test Cases and Usability and Humanity Requirements} \\
		\hline
    \textbf{NFR ID}   & \multicolumn{11}{c|}{\textbf{Test ID}} \\
    \hline
    ~ & \textbf{EUR1} & \textbf{EUR2} & \textbf{EUR3} & \textbf{PIR1} & \textbf{LR1} & \textbf{LR2} & \textbf{LR3} & \textbf{UPR1} & \textbf{UPR2} & \textbf{ACR1} & \textbf{ACR2} \\
    \hline
    NFRT-UH1 & ~ & ~ & ~ & ~ & ~ & ~ & X & ~ & ~ & ~ & ~ \\
    NFRT-UH2 & ~ & X & ~ & ~ & ~ & ~ & ~ & ~ & ~ & ~ & ~ \\
    NFRT-UH3 & ~ & ~ & ~ & ~ & ~ & ~ & ~ & ~ & ~ & X & ~ \\
    NFRT-UH4 & ~ & ~ & ~ & ~ & ~ & ~ & ~ & ~ & ~ & ~ & X \\
    NFRT-UH5 & X & ~ & ~ & ~ & ~ & ~ & ~ & ~ & ~ & ~ & ~ \\
    NFRT-UH6 & ~ & ~ & X & X & X & X & ~ & X & X & X & X \\
    \hline
  \end{longtable}

  \newpage

  \begin{longtable}{|l|cccccc|}
		\caption{Traceability Between Nonfunctional System Test Cases and Performance Requirements} \\
		\hline
    \textbf{NFR ID}   & \multicolumn{6}{c|}{\textbf{Test ID}} \\
    \hline
    ~ & \textbf{SLR1} & \textbf{SLR2} & \textbf{PAR1} & \textbf{PAR2} & \textbf{RFR1} & \textbf{CR1} \\
    \hline
    NFRT-PR1 & ~ & X & ~ & ~ & ~ & ~ \\
    NFRT-PR2 & X & ~ & ~ & ~ & ~ & ~ \\
    NFRT-PR3 & ~ & ~ & ~ & ~ & X & ~ \\
    NFRT-PR4 & ~ & ~ & X & ~ & ~ & ~ \\
    NFRT-PR5 & ~ & ~ & ~ & X & ~ & ~ \\
    NFRT-PR6 & ~ & ~ & ~ & ~ & ~ & X \\
    \hline
  \end{longtable}

  \begin{longtable}{|l|ccc|}
		\caption{Traceability Between Nonfunctional System Test Cases and Cultural Requirements} \\
		\hline
    \textbf{NFR ID}   & \multicolumn{3}{c|}{\textbf{Test ID}} \\
    \hline
    ~ & \textbf{CUR1} & \textbf{CUR2} & \textbf{CUR3} \\
    \hline
    NFRT-CUR1 & X & ~ & ~ \\
    NFRT-CUR2 & ~ & X & ~ \\
    NFRT-CUR3 & ~ & ~ & X \\
    \hline
  \end{longtable}

  \begin{longtable}{|l|cccccccccc|}
		\caption{Traceability Between Nonfunctional System Test Cases and Operational and Environmental Requirements} \\
		\hline
    \textbf{NFR ID}   & \multicolumn{10}{c|}{\textbf{Test ID}} \\
    \hline
    ~ & \textbf{EPER1} & \textbf{EPER2} & \textbf{EPER3} & \textbf{WER1} & \textbf{WER2} & \textbf{IASR1} & \textbf{IASR2} & \textbf{RER1} & \textbf{RER2} & \textbf{RER3} \\
    \hline
    NFRT-OER1 & X & X & X & ~ & ~ & X & X & ~ & ~ & ~ \\
    NFRT-OER2 & ~ & ~ & ~ & X & X & ~ & ~ & ~ & ~ & ~ \\
    NFRT-OER3 & ~ & ~ & ~ & ~ & ~ & ~ & ~ & X & X & X \\
    \hline
  \end{longtable}

  \begin{longtable}{|l|cccccc|}
		\caption{Traceability Between Nonfunctional System Test Cases and Security Requirements} \\
		\hline
    \textbf{NFR ID}   & \multicolumn{6}{c|}{\textbf{Test ID}} \\
    \hline
    ~ & \textbf{ACR1} & \textbf{INR1} & \textbf{INR2} & \textbf{PRR1} & \textbf{PRR2} & \textbf{AUR1} \\
    \hline
    NFRT-SR1 & X & ~ & ~ & ~ & ~ & ~ \\
    NFRT-SR2 & ~ & X & ~ & ~ & ~ & ~ \\
    NFRT-SR3 & ~ & ~ & X & ~ & ~ & ~ \\
    NFRT-SR4 & ~ & ~ & ~ & X & ~ & ~ \\
    NFRT-SR5 & ~ & ~ & ~ & ~ & X & ~ \\
    NFRT-SR6 & ~ & ~ & ~ & ~ & ~ & X \\
    \hline
  \end{longtable}
\end{landscape}

\section{Unit Test Description}

This section details the unit tests that will be performed by the team during development. The
unit tests will be focused on the modules of the application as written in the
\href{https://github.com/r-yeh/grocery-spending-tracker/blob/master/docs/Design/SoftDetailedDes/MIS.pdf}{MIS}.

\subsection{Unit Testing Scope}

  All modules described in the \href{https://github.com/r-yeh/grocery-spending-tracker/blob/master/docs/Design/SoftDetailedDes/MIS.pdf}{MIS}
  will be within the scope of unit testing except for the Receipt Vision Module, Database
  Driver Module and Locations Module. The Receipt Vision Module is excluded due to it largely being implemented via a third-party
  package called Google ML Kit. It is assumed this is a well-tested package and is thus excluded. Other functionality
  such as camera operation is already being tested via System Tests. Similarly, the Database Driver Module is being implemented
  using PostgreSQL which is also assumed to be properly tested. Additionally, the database will be tested through
  System Tests rather than individual unit tests. Finally, the Locations Module is being excluded because the team
  has decided it is out of scope for our current implementation. As a result, implementation and testing related to
  the module is being pushed to a future version.

\subsection{Tests for Functional Requirements}

% \wss{Most of the verification will be through automated unit testing.  If
%   appropriate specific modules can be verified by a non-testing based
%   technique.  That can also be documented in this section.}

  \subsubsection{Receipt Extraction Module}

  All the unit tests for the Receipt Extraction Module can be found in the following file: 
  \href{https://github.com/allanfang1/grocery_spending_tracker_app/blob/main/test/receipt_extraction_test.dart}{Receipt Extraction
  Unit Tests}. The tests focus on the individual aspects/tasks of the Receipt Extraction process and try to verify
  the "happy paths" and edge cases such as empty inputs for each. The tests are all being performed automatically
  and dynamically through Flutter's testing package.

\subsubsection{User Analytics Module}

...

\subsubsection{Users Module}

...

\subsubsection{Authentication Module}

...

\subsubsection{Recommendation Engine}

...

\subsubsection{Classification Engine}

...

\subsection{Tests for Nonfunctional Requirements}

The team does not plan to use unit testing for nonfunctional requirements. All testing for
nonfunctional requirements will be done via System Tests.

\subsection{Traceability Between Test Cases and Modules}

\wss{Provide evidence that all of the modules have been considered.}
				
\bibliographystyle{plainnat}

\bibliography{../../refs/References}

\newpage

\section{Appendix}

This is where you can place additional information.

\subsection{Symbolic Parameters}

The definition of the test cases will call for SYMBOLIC\_CONSTANTS.
Their values are defined in this section for easy maintenance.

\subsection{Functional Requirement Survey Questions}
\subsubsection{Survey questions for \textbf{FRT-FR1-2}.}

\noindent \textbf{Question 1:} How easy was it to log into the application?
Responses: \{(not easy) 1, 2, 3, 4, 5 (very easy)\} \\

\subsubsection{Survey questions for \textbf{FRT-FR8-2}.}

\noindent \textbf{Question 1:} How useful would you say that the application recommendations are to you?
Responses: \{(not useful) 1, 2, 3, 4, 5 (very useful)\} \\

\noindent \textbf{Question 2:} How likely are you to act the recommendation provided to you by the application?
Responses: \{(not likely) 1, 2, 3, 4, 5 (very likely)\}

\subsection{Look and Feel Survey Questions}

\begin{enumerate}
  \item Did you feel the colour scheme was cohesive and consistent throughout the application?
  \item On a scale of 1-10, how would you rate the general layouts of each of the pages?
  \item Did the general layouts of the application pages appear consistent throughout navigation?
  \item On a scale of 1-10, how would you rate how features were organized?
  \item Were there any places where you expected loading indicators but there weren't any or vice versa?
  \item Do you have any feedback on the overall look and feel of the application?
\end{enumerate}

\subsection{Usability Survey Questions}

\begin{enumerate}
  \item What parts of the application were easiest for you to understand/learn?
  \item What parts of the application do you feel additional explanation or clarity could be provided?
  \item Was the language and wording used in the application understandable and inoffensive? If not, please
  provide where and how the wording could be improved?
  \item Were there any issues you encountered while using the application?
  \item Did you have any issues with navigating the application? If you made a mistake, were you able to navigate
  back to where you started?
  \item On a scale of 1-10, how would you rate the usability and intuitiveness of the application?
\end{enumerate}

\newpage{}
\section*{Appendix --- Reflection}

The information in this section will be used to evaluate the team members on the
graduate attribute of Lifelong Learning.  Please answer the following questions:

\begin{enumerate}
  \item \textbf{What knowledge and skills will the team collectively need to acquire to
  successfully complete the verification and validation of your project?
  Examples of possible knowledge and skills include dynamic testing knowledge,
  static testing knowledge, specific tool usage etc.  You should look to
  identify at least one item for each team member.}

  For the verification and validation of this project, several skills will be required
  to ensure that this project is well-tested and meets all requirements. This will involve
  some general testing knowledge as well as some knowledge specific to the technologies we
  are using. By learning the following, the team will be able to produce a better project for this
  capstone in addition to more confidence in the testing process in our careers.
  \begin{enumerate}
    \item Knowledge of the different forms and approaches of dynamic testing
    \item Knowledge of the different forms and approaches of static testing
    \item Experience and knowledge working with \textit{Flutter's} internal unit testing framework and
    \textit{Flutter Performance Profiling} for frontend performance
    \item Experience and knowledge using \textit{Postman} for backend tests
    \item Communication skills to write surveys and elicit feedback from stakeholders/testers
  \end{enumerate}

  \item \textbf{For each of the knowledge areas and skills identified in the previous
  question, what are at least two approaches to acquiring the knowledge or
  mastering the skill?  Of the identified approaches, which will each team
  member pursue, and why did they make this choice?}

  For skills \textit{(a)} and \textit{(b)}, one approach that could be taken is to review notes from
  \textit{SFWRENG 3S03} where topics such as dynamic and static testing were covered. The different
  approaches for dynamic and static testing were also part of the lecture material which would be useful to go over again.
  A second approach is to conduct research online for tutorials, articles and documents covering the static and dynamic testing
  approaches. Hands-on experience could also be employed with different technologies that
  support static and dynamic testing.\par

  Skills \textit{(c)} and \textit{(d)} involve more specific technologies, so one approach to learning these
  could be to read documentation. For both technologies, documentation is present to learn how the
  developers had intended them to be used. For a second approach, \textit{YouTube} tutorials can be referenced to
  go over the basics and learn these technologies. Since many tutorials are made by users, it
  can provide more insight on how these applications can be used in a more practical setting.
  Lastly, third-party articles can also be used for additional learning. \par

  Finally, skill \textit{(e)} is more of a social skill rather than a technical skill compared to the other skills listed.
  One approach to improve communication skills for testing is to have our team actively engage in group collaboration and seek feedback from fellow team members when preparing for feedback elicitation. For instance, reviewing a proposed survey with the team will identify the good and bad aspects present that team members can internalize. For a second approach, \textit{SFWRENG 3S03} has lectures on surveys
  (such as the kinds of questions to ask and not ask) and similar testing approaches that can also be reviewed to better
  learn and improve this skill. Additionally, coursework in \textit{SFWRENG 4HC3} could also be used as survey formats and guidelines
  are covered in the material. Last, familiarization with software that supports surveys and elicitation could also be used as
  another approach.

  Below are the chosen skills from each team member, their rationale, and approaches they will use to learn or
  improve said skills.

  \textbf{Ryan Yeh}\\
  \textit{Chosen Skills: (c), (d), (e)} \\
  For verification and validation, I will be taking a leadership role in designing unit tests and software
  validation. Therefore, I want to ensure familiarity with \textit{Flutter's} testing technology and \textit{Postman}
  given that unit tests will primarily utilize these technologies. By ensuring proper learning of
  this software, I should be able to better guide the rest of the team during unit test development resulting
  in a stronger test suite. In addition, I also chose communication skills as a skill to focus on since this application
  is primarily user-focused. Therefore, ensuring proper user feedback and testing with stakeholders will be very important
  to this app's success. \par
  To learn and improve my skills with \textit{Flutter's} testing software and \textit{Postman}, I will refer to the
  official documentation and \textit{YouTube} tutorials. I feel this will give me a balance between original developer
  knowledge and practical user knowledge. In terms of communication skills, I will be employing group feedback
  and collaboration to ensure high-quality questions are being asked to maximize the feedback and information we receive
  from stakeholders and testers.

  \medskip
  \textbf{Sawyer Tang}\\
  \textit{Chosen Skills: (a), (b), (d)} \\
  I am going to be responsible for the FR system tests and the unit tests for the system. Therefore, I believe it is important for me to have a deep understanding of dynamic testing, static testing, and the use of \textit{Postman} for API testing. By having a deep understanding of these concepts and skills, I can contribute to the project's correctness and testing depth.
  To develop my Dynamic testing and Static testing skills, I plan to reference my \textit{SFWRENG 3S03} notes and resources, as well as put into practice skills from online articles and forums. With regards to improving my abilities with \textit{Postman}, I plan to reference \textit{YouTube}, blogs, and \textit{Postman's} own documentation.

  \medskip
  \textbf{Jason Nam}\\
  \textit{Chosen Skills: (a), (b), (e)}\\
  I will be responsible for SRS Verification, NFR system tests, FR system tests. These roles will require relevant skills and knowledge for dynamic testing, static testing, and writing surveys and eliciting feedback from stakeholders and/or testers. Through thorough understanding of the the listed skills and knowledge, I believe that I can be of use and contribute to the verification and validation of the application.\par
  I plan to develop my dynamic testing and static testing skills by referring back to \textit{SFWRENG 3S03} practices. I also plan to understand the basics of each testing methods first by partaking in online courses, articles, and tutorials. I plan to try hands-on testing and practice cases. I will also take advantage of code review practices and analysis tools like \textit{flutter\_lints} and \textit{Pylint} to enhance static testing skills. For survey and eliciting feedback related skills, I will study different survey designs. I will reference back to \textit{SFWRENG 4HC3} for understanding survey procedures and guidelines as well as eliciting feedback. I plan to familiarize myself with \textit{Google Forms} and other tools to optimize surveys and eliciting feedback.

  \medskip
  \textbf{Allan Fang}\\
  \textit{Chosen Skills: (b), (e) } \\
  I will be taking the lead on SRS Verification, VnV Verification, and Design Verification which will primarily be the verification of documentation. Therefore, I will focus on acquiring the relevant skills and knowledge for documentation verification according to the VnV plan (guided walkthroughs with stakeholders and inspections with checklists) which will be static testing and communication for eliciting feedback. I will be using both proposed approaches for acquiring knowledge on static testing and both proposed approaches for acquiring communication skills to elicit feedback. \\
  I have chosen the first two approaches for understanding static testing because I can gain both theoretical knowledge (course notes) and practical insights (online research) into the various aspects of static testing. This comprehensive approach will help me be more effective in my verification.\\
  I have also chosen the first two approaches for communication skills for eliciting feedback. The first approach of group collaboration and feedback will enable me to practice and refine this skill while the second approach will provide a solid theoretical foundation to start with. Using both approaches will enhance my ability to elicit valuable feedback from stakeholders during the verification process, ensuring well-validated documentation.

  
\end{enumerate}

\end{document}