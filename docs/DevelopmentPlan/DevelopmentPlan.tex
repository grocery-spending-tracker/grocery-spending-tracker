\documentclass{article}

\usepackage{booktabs}
\usepackage{tabularx}

\def\code#1{\texttt{#1}}

\title{Development Plan\\\progname}

\author{\authname}

\date{}

%% Comments

\usepackage{color}

\newif\ifcomments\commentstrue %displays comments
%\newif\ifcomments\commentsfalse %so that comments do not display

\ifcomments
\newcommand{\authornote}[3]{\textcolor{#1}{[#3 ---#2]}}
\newcommand{\todo}[1]{\textcolor{red}{[TODO: #1]}}
\else
\newcommand{\authornote}[3]{}
\newcommand{\todo}[1]{}
\fi

\newcommand{\wss}[1]{\authornote{blue}{SS}{#1}} 
\newcommand{\plt}[1]{\authornote{magenta}{TPLT}{#1}} %For explanation of the template
\newcommand{\an}[1]{\authornote{cyan}{Author}{#1}}

%% Common Parts

\newcommand{\progname}{ProgName} % PUT YOUR PROGRAM NAME HERE
\newcommand{\authname}{Team \#, Team Name
\\ Student 1 name
\\ Student 2 name
\\ Student 3 name
\\ Student 4 name} % AUTHOR NAMES                  

\usepackage{hyperref}
    \hypersetup{colorlinks=true, linkcolor=blue, citecolor=blue, filecolor=blue,
                urlcolor=blue, unicode=false}
    \urlstyle{same}
                                


\begin{document}

\maketitle

\begin{table}[hp]
\caption{Revision History} \label{TblRevisionHistory}
\begin{tabularx}{\textwidth}{llX}
\toprule
\textbf{Date} & \textbf{Developer(s)} & \textbf{Change}\\
\midrule
Sept. 19, 2023 & Ryan Yeh & Wrote initial Team Communication Plan\\
Sept. 20, 2023 & Sawyer Tang & Workflow Plan\\
... & ... & ...\\
\bottomrule
\end{tabularx}
\end{table}

\wss{Put your introductory blurb here.}

\section{Team Meeting Plan}

\section{Team Communication Plan}

The primary means for communication will be the team \emph{Discord} server. In an effort to 
ensure team members are up-to-date with messages and discussion, everyone is expected to check \emph{Discord}
at least once per day and respond to direct messages within 24 hours.
Additionally, the server has been organized in such a way that different topics are separated into different channels. 
For general  discussion and planning, the \emph{general} channel will be used. When requesting reviewers on pull requests, 
the \emph{pull-request} channel should be used. Any bugs and issues that require group consultation will go 
into the \emph{bugs-issues} channel. Discussion of progress updates for upcoming deliverables will go into the
\emph{status-updates} channel and lastly, any requests for help regarding work items will go into the \emph{help} channel.
If urgent contact with group members is required, \emph{SMS} should be used.

\section{Team Member Roles}

\section{Workflow Plan}

We will be using git as a version control software with GitHub as our remote repository. The root branch will be \code{master} and development branch mased on \code{master} will be \code{dev}. The developer will branch off from \code{dev} to work on issues. There will be a new branch for each issue and will be named comtaining the issue number and name detailed below. Continuous Inntegration is set up between all branches \code{dev}, \code{master}, and Issues.

Issues are used to track tasks and meetings related to the project. Meeting issues will be created before each meeting according to the \textbf{MeetingIssueTemplate.md} template. These include addentance, minutes, and action items from the meeting. These notes can be referenced from other issues and PRs for tracability. 
Feature issues are used to track tasks. They can be referenced in PRs and meetings and outline the tesk that the developer must fullfill in their PR. 
Issues can be assigned to a Milestone which tracks issues related to deliverables. These are useful for filtering issues and also tracking due dates.
Issues can be assigned labels which give information of which part of the project the issue related to. These can denote issue types as well as issue severity or complexity. Useful for filtering.

\subsection{Version Control Developer Step-By-Step}

\begin{enumerate}
	\item Granted Issue ticket with issue number.
	\item Open branch with name \{IssueNumber\}/\{issue-name\}. (eg. \code{135/add-user-endpoint})
	\item Commit with format: \{IssueNumber\}:\{commit message\}. (eg. \code{"135: updated yaml for add-user-endpoint"})
	\item Open Pull Request on GitHub following the \textbf{PRTemplate.md} template, thus mapping the PR to the Issue.
	\item Await 2 reviewers.
	\item Fix reviewer comments.
	\item Merge to dev.
\end{enumerate}

\section{Proof of Concept Demonstration Plan}

What is the main risk, or risks, for the success of your project?  What will you
demonstrate during your proof of concept demonstration to convince yourself that
you will be able to overcome this risk?

\section{Technology}

\begin{itemize}
\item Specific programming language
\item Specific linter tool (if appropriate)
\item Specific unit testing framework
\item Investigation of code coverage measuring tools
\item Specific plans for Continuous Integration (CI), or an explanation that CI
  is not being done
\item Specific performance measuring tools (like Valgrind), if
  appropriate
\item Libraries you will likely be using?
\item Tools you will likely be using?
\end{itemize}

\section{Coding Standard}

\section{Project Scheduling}

\wss{How will the project be scheduled?}

\end{document}