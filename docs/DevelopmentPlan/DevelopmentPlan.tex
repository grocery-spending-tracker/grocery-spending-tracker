\documentclass{article}

\usepackage{booktabs}
\usepackage{tabularx}

\title{Development Plan\\\progname}

\author{\authname}

\date{}

%% Comments

\usepackage{color}

\newif\ifcomments\commentstrue %displays comments
%\newif\ifcomments\commentsfalse %so that comments do not display

\ifcomments
\newcommand{\authornote}[3]{\textcolor{#1}{[#3 ---#2]}}
\newcommand{\todo}[1]{\textcolor{red}{[TODO: #1]}}
\else
\newcommand{\authornote}[3]{}
\newcommand{\todo}[1]{}
\fi

\newcommand{\wss}[1]{\authornote{blue}{SS}{#1}} 
\newcommand{\plt}[1]{\authornote{magenta}{TPLT}{#1}} %For explanation of the template
\newcommand{\an}[1]{\authornote{cyan}{Author}{#1}}

%% Common Parts

\newcommand{\progname}{ProgName} % PUT YOUR PROGRAM NAME HERE
\newcommand{\authname}{Team \#, Team Name
\\ Student 1 name
\\ Student 2 name
\\ Student 3 name
\\ Student 4 name} % AUTHOR NAMES                  

\usepackage{hyperref}
    \hypersetup{colorlinks=true, linkcolor=blue, citecolor=blue, filecolor=blue,
                urlcolor=blue, unicode=false}
    \urlstyle{same}
                                


\begin{document}

\maketitle

\begin{table}[hp]
\caption{Revision History} \label{TblRevisionHistory}
\begin{tabularx}{\textwidth}{llX}
\toprule
\textbf{Date} & \textbf{Developer(s)} & \textbf{Change}\\
\midrule
19/09/23 & Ryan Yeh & Wrote initial Team Communication Plan\\
20/09/23 & Ryan Yeh & Revised Team Communication Plan based on examples\\
20/09/23 & Jason Nam & Added Team Meeting Plan Section for Development Plan Doc\\
21/09/23 & Ryan Yeh & Wrote initial Project Scheduling\\
... & ... & ...\\
\bottomrule
\end{tabularx}
\end{table}

\wss{Put your introductory blurb here.}

\section{Team Meeting Plan}

\begin{itemize}
	\item Standup meetings are scheduled weekly from \textbf{12:30 PM to 1:30 PM every Wednesday}. The primary meeting location is the \emph{H.G. Thode Library of Science \& Engineering}. However, in specific circumstances and within certain limitations, the meeting location may be shifted to \emph{Discord}.
    	\item Sprint review meetings are scheduled weekly from \textbf{5:00 PM to 6:00 PM every Sunday}. The meeting location is \emph{Discord}.
    	\item The roles and agendas for the next meeting will be decided in the final minutes of each meeting.
    	\item Requests for additional meetings outside of regular scheduled meetings must be made and communicated at least \textbf{1 day in advance}.
   	\item The Chair of the meeting is responsible for preparing the task backlog and identifying any relevant issues prior to the meeting.
\end{itemize}

\section{Team Communication Plan}

\subsection{Discord}

The primary means for communication will be the team \emph{Discord} server. In an effort to
ensure team members are up-to-date with messages and discussion, everyone is expected to check \emph{Discord}
at least once per day and respond to direct messages within 24 hours.
Additionally, the server has been organized in such a way that different topics are separated into different channels.
For general  discussion and planning, the \emph{general} channel will be used. When requesting reviewers on pull requests,
the \emph{pull-request} channel should be used. Discussion surrounding bugs and issues will be brought up in the
\emph{bugs-issues} channel. Discussion of progress updates for upcoming deliverables will go into the
\emph{status-updates} channel and lastly, any requests for help regarding work items will go into the \emph{help} channel.

\subsection{SMS}

Should the team require urgent or time sensitive contact with any members, \emph{SMS} will be used as the method
of communication. \emph{SMS} should facilitate faster and more efficient communication compared to \emph{Discord}
and will be used in events such as unforeseen issues with meeting availability or last-minute changes for deadlines.

\subsection{Github}

Github will be used to facilitate more technical code-related discussions via pull requests in addition to
keeping track of work items using the issue tracker. The issue tracker will also be used to store notes from
all group meetings.

\section{Team Member Roles}

\section{Workflow Plan}

\begin{itemize}
	\item How will you be using git, including branches, pull request, etc.?
	\item How will you be managing issues, including template issues, issue
	classificaiton, etc.?
\end{itemize}

\section{Proof of Concept Demonstration Plan}

What is the main risk, or risks, for the success of your project?  What will you
demonstrate during your proof of concept demonstration to convince yourself that
you will be able to overcome this risk?

\section{Technology}

\begin{itemize}
\item Specific programming language
\item Specific linter tool (if appropriate)
\item Specific unit testing framework
\item Investigation of code coverage measuring tools
\item Specific plans for Continuous Integration (CI), or an explanation that CI
  is not being done
\item Specific performance measuring tools (like Valgrind), if
  appropriate
\item Libraries you will likely be using?
\item Tools you will likely be using?
\end{itemize}

\section{Coding Standard}

\section{Project Scheduling}

The general scheduling of the project will follow a weekly sprint format. At the beginning
of each week, a plan will be made based on the deliverables due and progress made from previous sprints. The
work will then be distributed afterwards such that all members have tasks they have
to work on. This structure will be done to ensure feasibility regarding weekly
objectives and maintaining focus on certain goals rather than scope creeping over
a longer development period. This will also allow more flexibility among the team to adjust workload
based on other responsibilities. \par

Regarding course deliverables specifically, scheduling will follow closely to the deadlines written in the
course outline. The goal for each deliverable will be to have them finished a couple days
prior to the actual deadline in order to account for any problems or changes that may
need to be made. As requirements are created and planning gets done for different project features,
more technical tasks will be added via the \emph{Github} Issue Tracker and these too will be
integrated into the weekly sprints as they become available. This will especially be true during
extended deadlines such as the period of time between the Proof of Concept Demonstration and
the Design Document Revision 0 deadline. \par

Overall, this scheduling structure should promote a steady output of work allowing the team
to meet all deadlines and produce a better end product.

\end{document}