\documentclass{article}

\usepackage{booktabs}
\usepackage{tabularx}

\title{Development Plan\\\progname}

\author{\authname}

\date{}

%% Comments

\usepackage{color}

\newif\ifcomments\commentstrue %displays comments
%\newif\ifcomments\commentsfalse %so that comments do not display

\ifcomments
\newcommand{\authornote}[3]{\textcolor{#1}{[#3 ---#2]}}
\newcommand{\todo}[1]{\textcolor{red}{[TODO: #1]}}
\else
\newcommand{\authornote}[3]{}
\newcommand{\todo}[1]{}
\fi

\newcommand{\wss}[1]{\authornote{blue}{SS}{#1}} 
\newcommand{\plt}[1]{\authornote{magenta}{TPLT}{#1}} %For explanation of the template
\newcommand{\an}[1]{\authornote{cyan}{Author}{#1}}

%% Common Parts

\newcommand{\progname}{ProgName} % PUT YOUR PROGRAM NAME HERE
\newcommand{\authname}{Team \#, Team Name
\\ Student 1 name
\\ Student 2 name
\\ Student 3 name
\\ Student 4 name} % AUTHOR NAMES                  

\usepackage{hyperref}
    \hypersetup{colorlinks=true, linkcolor=blue, citecolor=blue, filecolor=blue,
                urlcolor=blue, unicode=false}
    \urlstyle{same}
                                


\begin{document}

\maketitle

\begin{table}[hp]
\caption{Revision History} \label{TblRevisionHistory}
\begin{tabularx}{\textwidth}{llX}
\toprule
\textbf{Date} & \textbf{Developer(s)} & \textbf{Change}\\
\midrule
19/09/23 & Ryan Yeh & Wrote initial Team Communication Plan\\
20/09/23 & Ryan Yeh & Revised Team Communication Plan based on examples\\
20/09/23 & Jason Nam & Added Team Meeting Plan Section for Development Plan Doc\\
24/09/23 & Jason Nam & Added Technology section of Development Plan Doc\\
\bottomrule
\end{tabularx}
\end{table}

\wss{Put your introductory blurb here.}

\section{Team Meeting Plan}

\begin{itemize}
	\item Standup meetings are scheduled weekly from \textbf{12:30 PM to 1:30 PM every Wednesday}. The primary meeting location is the \emph{H.G. Thode Library of Science \& Engineering}. However, in specific circumstances and within certain limitations, the meeting location may be shifted to \emph{Discord}.
    	\item Sprint review meetings are scheduled weekly from \textbf{5:00 PM to 6:00 PM every Sunday}. The meeting location is \emph{Discord}.
    	\item The roles and agendas for the next meeting will be decided in the final minutes of each meeting.
    	\item Requests for additional meetings outside of regular scheduled meetings must be made and communicated at least \textbf{1 day in advance}.
   	\item The Chair of the meeting is responsible for preparing the task backlog and identifying any relevant issues prior to the meeting.
\end{itemize}

\section{Team Communication Plan}

\subsection{Discord}

The primary means for communication will be the team \emph{Discord} server. In an effort to
ensure team members are up-to-date with messages and discussion, everyone is expected to check \emph{Discord}
at least once per day and respond to direct messages within 24 hours.
Additionally, the server has been organized in such a way that different topics are separated into different channels.
For general  discussion and planning, the \emph{general} channel will be used. When requesting reviewers on pull requests,
the \emph{pull-request} channel should be used. Discussion surrounding bugs and issues will be brought up in the
\emph{bugs-issues} channel. Discussion of progress updates for upcoming deliverables will go into the
\emph{status-updates} channel and lastly, any requests for help regarding work items will go into the \emph{help} channel.

\subsection{SMS}

Should the team require urgent or time sensitive contact with any members, \emph{SMS} will be used as the method
of communication. \emph{SMS} should facilitate faster and more efficient communication compared to \emph{Discord}
and will be used in events such as unforeseen issues with meeting availability or last-minute changes for deadlines.

\subsection{Github}

Github will be used to facilitate more technical code-related discussions via pull requests in addition to
keeping track of work items using the issue tracker. The issue tracker will also be used to store notes from
all group meetings.

\section{Team Member Roles}

\section{Workflow Plan}

\begin{itemize}
	\item How will you be using git, including branches, pull request, etc.?
	\item How will you be managing issues, including template issues, issue
	classificaiton, etc.?
\end{itemize}

\section{Proof of Concept Demonstration Plan}

What is the main risk, or risks, for the success of your project?  What will you
demonstrate during your proof of concept demonstration to convince yourself that
you will be able to overcome this risk?

\section{Technology}

\subsection{Programming Language}
    \begin{itemize}
        \item The team will utilize \textbf{VSCode} source-code editor.
        \begin{itemize}
            \item VSCode is very flexible and well documented.
            \item The team has many experiences working with VSCode.
        \end{itemize}
        \item The team will utilize \textbf{NodeJS} as their back-end programming language.
        \begin{itemize}
            \item NodeJS is very popular and well documented.
            \item NodeJS has a reliable and flexible coding environment.
        \end{itemize}
        \item The team will utilize \textbf{Flutter} (Dart) as their front-end programming language.
        \begin{itemize}
            \item Flutter is a mainstream and modern open-source UI software development kit created by Google.
            \item Flutter ports well with web, Android, and IOS operating systems.
        \end{itemize}
    \end{itemize}
\subsection{Linter Tool}
    \begin{itemize}
        \item The team will utilize \textbf{flutter\_lints} for front-end programming language which is built on top of Dart's recommended sets of lints from Official Dart lint rules.
        \item The team will utilize \textbf{ESLint} tool for identifying and reporting on patterns found in ECMAScript/JavaScript code.
    \end{itemize}
\subsection{Unit Testing Framework}
    \begin{itemize}
        \item The team will use Flutter's integrated unit testing framework for front-end unit testing purposes.
        \item The team will use \textbf{Postman} and \textbf{Mocha} frameworks for back-end unit testing purposes.
        \begin{itemize}
            \item The team has extensive experiences working with Postman.
            \item Postman is a popular choice for testing back-end database queries.
        \end{itemize}
    \end{itemize}
\subsection{Code Coverage Measuring Tools}
    \begin{itemize}
        \item The team will utilize \textbf{coverage-node} to run and report code coverage for back-end NodeJS code.
        \item The team will utilize Flutter's coverage extension in VSCode to measure front-end code coverage.
    \end{itemize}
\subsection{Plans for Continuous Integration (CI)}
    The team will utilize \textbf{Linter}, \textbf{Github}, and general automated testing on merge to implement Continuous Integration (CI).
\subsection{Performance Measuring Tools}
    \begin{itemize}
        \item The team will utilize \textbf{Flutter Performance Profiling}, part  of  Flutter DevTools, for debugging front-end performance issues.
    \end{itemize}
\subsection{Libraries and Other Tools}
    \begin{itemize}
        \item The team will utilize \textbf{node-tesseract-ocr} to handle OCR on back-end.
        \begin{itemize}
            \item node-tesseract-ocr compiles well with node.
            \item node-tesseract-ocr is popular and well documented.
        \end{itemize}
        \item The team will utilize \textbf{PostgreSQL} relational database to persist and aggregate data on demand.
        \begin{itemize}
            \item PostgreSQL is a flexible database witth high levels of control.
        \end{itemize}
        \item The team will utilize \textbf{node-rsa} to encrypt data and provide security for the system.
    \end{itemize}

\section{Coding Standard}

\section{Project Scheduling}

\wss{How will the project be scheduled?}

\end{document}