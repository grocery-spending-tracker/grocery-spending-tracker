\documentclass{article}

\usepackage{booktabs}
\usepackage{tabularx}

\def\code#1{\texttt{#1}}

\title{Development Plan\\\progname}

\author{\authname}

\date{}

%% Comments

\usepackage{color}

\newif\ifcomments\commentstrue %displays comments
%\newif\ifcomments\commentsfalse %so that comments do not display

\ifcomments
\newcommand{\authornote}[3]{\textcolor{#1}{[#3 ---#2]}}
\newcommand{\todo}[1]{\textcolor{red}{[TODO: #1]}}
\else
\newcommand{\authornote}[3]{}
\newcommand{\todo}[1]{}
\fi

\newcommand{\wss}[1]{\authornote{blue}{SS}{#1}} 
\newcommand{\plt}[1]{\authornote{magenta}{TPLT}{#1}} %For explanation of the template
\newcommand{\an}[1]{\authornote{cyan}{Author}{#1}}

%% Common Parts

\newcommand{\progname}{ProgName} % PUT YOUR PROGRAM NAME HERE
\newcommand{\authname}{Team \#, Team Name
\\ Student 1 name
\\ Student 2 name
\\ Student 3 name
\\ Student 4 name} % AUTHOR NAMES                  

\usepackage{hyperref}
    \hypersetup{colorlinks=true, linkcolor=blue, citecolor=blue, filecolor=blue,
                urlcolor=blue, unicode=false}
    \urlstyle{same}
                                


\begin{document}

\maketitle

\begin{table}[hp]
\caption{Revision History} \label{TblRevisionHistory}
\begin{tabularx}{\textwidth}{llX}
\toprule
\textbf{Date} & \textbf{Developer(s)} & \textbf{Change}\\
\midrule
19/09/23 & Ryan Yeh & Wrote initial Team Communication Plan\\
20/09/23 & Allan Fang & Added team member roles\\
20/09/23 & Ryan Yeh & Revised Team Communication Plan based on examples\\
20/09/23 & Sawyer Tang & Workflow Plan\\
20/09/23 & Jason Nam & Added Team Meeting Plan Section for Development Plan Doc\\
20/09/23 & Allan Fang & Added document introduction\\
21/09/23 & Ryan Yeh & Wrote initial Project Scheduling\\
24/09/23 & Ryan Yeh & Wrote Proof of Concept Demo Plan\\
24/09/23 & Sawyer Tang & Code Style Standards\\
24/09/23 & Jason Nam & Added Technology section of Development Plan Doc\\
\bottomrule
\end{tabularx}
\end{table}

This document describes the necessary aspects for the effective execution of the 'Grocery Spending Tracker' project including team organization, communication, technical details, and project timelines.

\section{Team Meeting Plan}

\begin{itemize}
	\item Standup meetings are scheduled weekly from \textbf{12:30 PM to 1:30 PM every Wednesday}. The primary meeting location is the \emph{H.G. Thode Library of Science \& Engineering}. However, in specific circumstances and within certain limitations, the meeting location may be shifted to \emph{Discord}.
    	\item Sprint review meetings are scheduled weekly from \textbf{5:00 PM to 6:00 PM every Sunday}. The meeting location is \emph{Discord}.
    	\item The roles and agendas for the next meeting will be decided in the final minutes of each meeting.
    	\item Requests for additional meetings outside of regular scheduled meetings must be made and communicated at least \textbf{1 day in advance}.
   	\item The Chair of the meeting is responsible for preparing the task backlog and identifying any relevant issues prior to the meeting.
\end{itemize}

\section{Team Communication Plan}

\subsection{Discord}

The primary means for communication will be the team \emph{Discord} server. In an effort to
ensure team members are up-to-date with messages and discussion, everyone is expected to check \emph{Discord}
at least once per day and respond to direct messages within 24 hours.
Additionally, the server has been organized in such a way that different topics are separated into different channels.
For general  discussion and planning, the \emph{general} channel will be used. When requesting reviewers on pull requests,
the \emph{pull-request} channel should be used. Discussion surrounding bugs and issues will be brought up in the
\emph{bugs-issues} channel. Discussion of progress updates for upcoming deliverables will go into the
\emph{status-updates} channel and lastly, any requests for help regarding work items will go into the \emph{help} channel.

\subsection{SMS}

Should the team require urgent or time sensitive contact with any members, \emph{SMS} will be used as the method
of communication. \emph{SMS} should facilitate faster and more efficient communication compared to \emph{Discord}
and will be used in events such as unforeseen issues with meeting availability or last-minute changes for deadlines.

\subsection{Github}

Github will be used to facilitate more technical code-related discussions via pull requests in addition to
keeping track of work items using the issue tracker. The issue tracker will also be used to store notes from
all group meetings.

\section{Team Member Roles}
Individuals will take leadership in their respective assigned areas. In development, all members are responsible for adjusting and filling roles as the project evolves. All team meetings will have a scrum master and a scribe, rotated on a per meeting basis.
	\subsection{Allan Fang - Developer}
        \begin{itemize}
            \item Lead on mobile and user interface development
            \item Lead on optical character recognition research and development
            \item Graphics design and creation specialist
        \end{itemize}
    \subsection{Jason Nam - Developer}
        \begin{itemize}
            \item Lead on research and development concerning language models
            \item Lead on backend development
            \item Lead data engineer
        \end{itemize}
    \subsection{Sawyer Tang - Developer}
        \begin{itemize}
            \item Responsible for server setup
            \item Responsible for DevOps and version control
            \item Graphics design and creation specialist
            \item Lead on backend development
        \end{itemize}
    \subsection{Ryan Yeh - Developer}
        \begin{itemize}
            \item Lead on project documentation
            \item Lead on testing and quality assurance
            \item Primary contact for external communications
        \end{itemize}

\section{Workflow Plan}

We will be using git as a version control software with GitHub as our remote repository. The root branch will be \code{master} and development branch based on \code{master} will be \code{dev}. The developer will branch off from \code{dev} to work on issues. There will be a new branch for each issue and will be named containing the issue number and name detailed below. Continuous Integration is set up between all branches \code{dev}, \code{master}, and Issues. Once the milestone is finished to the degree we are happy to be graded, the \code{dev} branch will be merged into \code{master}.

Issues are used to track tasks and meetings related to the project. Meeting issues will be created before each meeting according to the \textbf{MeetingIssueTemplate.md} template. These include attendance, minutes, and action items from the meeting. These notes can be referenced from other issues and PRs for traceability.
Feature issues are used to track tasks. They can be referenced in PRs and meetings to outline the task that the developer must fulfill in their PR.
Issues can be assigned to a Milestone which tracks issues related to deliverables. These are useful for filtering issues and also tracking due dates.
Issues can be assigned labels which give information on which part of the project the issue related to. These can denote issue types as well as issue severity or complexity and allows filtering.

\subsection{Some signifigant labels:}

\begin{itemize}
	\item \textbf{documentation}: used for all issues related to building or maintaining the documentation for our project (including, but not limited to, LaTeX related issues).
	\item \textbf{dev-ops}: used for all issues related to code production, integration, or conventions.
	\item \textbf{meeting}: used to denote an issue is for a meeting.
	\item \textbf{lecture}: used to denote an issue is for a lecture.
	\item \textbf{bug}: used for issues that describe a bug that needs to be fixed.
	\item \textbf{feature}: used for issues that describe a new code feature to be implemented.
\end{itemize}

\subsection{Version Control Developer Step-By-Step}

\begin{enumerate}
	\item Granted Issue ticket with issue number.
	\item Open branch with name \{IssueNumber\}/\{issue-name\}. (eg. \code{135/add-user-endpoint})
	\item Commit with format: \{IssueNumber\}:\{commit message\}. (eg. \code{"135: updated yaml for add-user-endpoint"})
	\item Open Pull Request on GitHub following the \textbf{PRTemplate.md} template, thus mapping the PR to the Issue.
	\item Await 2 reviewers.
	\item Fix reviewer comments.
	\item Developer will merge their \emph{own} branch to dev.
\end{enumerate}

\section{Proof of Concept Demonstration Plan}

The two main functionalities necessary for this application to succeed are the OCR of receipts
and the ability to map what is read to database entries. The OCR portion is the primary method of inputting data
and without it, the application would not work. Therefore, it is a major risk to consider. Similarly, another
function of the app is to recommend other grocery stores where recently purchased items can be bought at a lower price.
The basis of this feature will involve accurately mapping what is scanned from the OCR to the appropriate database entry and
updating it with the new purchase price and location. The primary risk comes from how different receipts can present
the same item differently. As such, a major task will be coming up with a way to ensure identical items on different receipts
map to the same database entry. To show that these risks can be overcome, the Proof of Concept Demo will involve a very
simple application with just the OCR functionality implemented. Then, two receipts with the same items will be scanned
and the database will be shown as it updates with the new information. Ideally, the results of the demonstration
will show that the OCR can accurately read the receipts and items on both receipts map appropriately in the database.

\section{Technology}

\subsection{Programming Language}
    \begin{itemize}
        \item The team will utilize \textbf{VSCode} source-code editor.
        \begin{itemize}
            \item VSCode is very flexible and well documented.
            \item The team has plenty of experience working with VSCode.
        \end{itemize}
        \item The team will utilize \textbf{NodeJS} as their back-end programming language.
        \begin{itemize}
            \item NodeJS is very popular and well documented.
            \item NodeJS has a reliable and flexible coding environment.
        \end{itemize}
        \item The team will utilize \textbf{Flutter} (Dart) as their front-end programming language.
        \begin{itemize}
            \item Flutter is a mainstream and modern open-source UI software development kit created by Google.
            \item Flutter ports well with web, Android, and IOS operating systems.
        \end{itemize}
    \end{itemize}
\subsection{Linter Tool}
    \begin{itemize}
        \item The team will utilize \textbf{flutter\_lints} for front-end programming language which is built on top of Dart's recommended sets of lints from Official Dart lint rules.
        \item The team will utilize \textbf{ESLint} tool for identifying and reporting on patterns found in ECMAScript/JavaScript code.
    \end{itemize}
\subsection{Unit Testing Framework}
    \begin{itemize}
        \item The team will use Flutter's integrated unit testing framework for front-end unit testing purposes.
        \item The team will use \textbf{Postman} and \textbf{Mocha} frameworks for back-end unit testing purposes.
        \begin{itemize}
            \item The team has extensive experiences working with Postman.
            \item Postman is a popular choice for testing back-end database queries.
        \end{itemize}
    \end{itemize}
\subsection{Code Coverage Measuring Tools}
    \begin{itemize}
        \item The team will utilize \textbf{coverage-node} to run and report code coverage for back-end NodeJS code.
        \begin{itemize}
            \item coverage-node allows code coverage report with zero configurations.
            \item Package is lean and simple to use.
        \end{itemize}
        \item The team will utilize Flutter's coverage extension in VSCode to measure front-end code coverage.
    \end{itemize}
\subsection{Plans for Continuous Integration (CI)}
    The team will utilize \textbf{Github} for Continuous Integration with automated testing and linting being integrated into the pipeline.
\subsection{Performance Measuring Tools}
    \begin{itemize}
        \item The team will utilize \textbf{Flutter Performance Profiling}, part  of  Flutter DevTools, for debugging front-end performance issues.
    \end{itemize}
\subsection{Libraries and Other Tools}
    \begin{itemize}
        \item The team will utilize \textbf{node-tesseract-ocr} to handle OCR on back-end.
        \begin{itemize}
            \item node-tesseract-ocr compiles well with node.
            \item node-tesseract-ocr is popular and well documented.
        \end{itemize}
        \item The team will utilize \textbf{PostgreSQL} relational database to persist and aggregate data on demand.
        \begin{itemize}
            \item PostgreSQL is a flexible database witth high levels of control.
        \end{itemize}
        \item The team will utilize \textbf{node-rsa} to encrypt data and provide security for the system.
        \begin{itemize}
            \item node-rsa is widely used for encrypting back-end database using node.js.
        \end{itemize}
    \end{itemize}

\section{Coding Standards}

\begin{itemize}
\item NodeJS code will follow the \href{https://www.ecma-international.org/publications-and-standards/standards/ecma-262/}{ECMAScript} scripting conventions. \href{https://www.npmjs.com/package/eslint}{eslint} package will be used to keep code in adherence with ECMAScript.
\item Flutter code will follow the \href{https://dart.dev/effective-dart/style}{Dart Style Guide}. \href{https://pub.dev/packages/flutter_lints}{flutter\_lints} package will be used to keep code in adherence with the Dart Style Guide.
\end{itemize}

\section{Project Scheduling}

The general scheduling of the project will follow a weekly sprint format. At the beginning
of each week, a plan will be made based on the deliverables due and progress made from previous sprints. The
work will then be distributed afterwards such that all members have tasks they have
to work on. This structure will ensure feasibility regarding weekly
objectives and maintaining focus on certain goals rather than scope creeping over
a longer development period. This will also allow more flexibility among the team to adjust workload
based on other responsibilities. \par

Regarding course deliverables specifically, scheduling will follow closely to the deadlines written in the
course outline. The goal for each deliverable will be to have them finished a couple days
prior to the actual deadline in order to account for any problems or changes that may
need to be made. As requirements are created and planning gets done for different project features,
more technical tasks will be added via the \emph{Github} Issue Tracker and these too will be
integrated into the weekly sprints as they become available. This will especially be true during
extended deadlines such as the period of time between the Proof of Concept Demonstration and
the Design Document Revision 0 deadline. \par

Overall, this scheduling structure should promote a steady output of work allowing the team
to meet all deadlines and produce a better end product.

\end{document}