\documentclass{article}

\usepackage{booktabs}
\usepackage{tabularx}

\title{Development Plan\\\progname}

\author{\authname}

\date{}

%% Comments

\usepackage{color}

\newif\ifcomments\commentstrue %displays comments
%\newif\ifcomments\commentsfalse %so that comments do not display

\ifcomments
\newcommand{\authornote}[3]{\textcolor{#1}{[#3 ---#2]}}
\newcommand{\todo}[1]{\textcolor{red}{[TODO: #1]}}
\else
\newcommand{\authornote}[3]{}
\newcommand{\todo}[1]{}
\fi

\newcommand{\wss}[1]{\authornote{blue}{SS}{#1}} 
\newcommand{\plt}[1]{\authornote{magenta}{TPLT}{#1}} %For explanation of the template
\newcommand{\an}[1]{\authornote{cyan}{Author}{#1}}

%% Common Parts

\newcommand{\progname}{ProgName} % PUT YOUR PROGRAM NAME HERE
\newcommand{\authname}{Team \#, Team Name
\\ Student 1 name
\\ Student 2 name
\\ Student 3 name
\\ Student 4 name} % AUTHOR NAMES                  

\usepackage{hyperref}
    \hypersetup{colorlinks=true, linkcolor=blue, citecolor=blue, filecolor=blue,
                urlcolor=blue, unicode=false}
    \urlstyle{same}
                                


\begin{document}

\maketitle

\begin{table}[hp]
\caption{Revision History} \label{TblRevisionHistory}
\begin{tabularx}{\textwidth}{llX}
\toprule
\textbf{Date} & \textbf{Developer(s)} & \textbf{Change}\\
\midrule
Sept. 19, 2023 & Ryan Yeh & Wrote initial Team Communication Plan\\
20/09/23 & Allan Fang & Added team member roles\\
\bottomrule
\end{tabularx}
\end{table}

\wss{Put your introductory blurb here.}

\section{Team Meeting Plan}

\section{Team Communication Plan}

The primary means for communication will be the team \emph{Discord} server. In an effort to 
ensure team members are up-to-date with messages and discussion, everyone is expected to check \emph{Discord}
at least once per day and respond to direct messages within 24 hours.
Additionally, the server has been organized in such a way that different topics are separated into different channels. 
For general  discussion and planning, the \emph{general} channel will be used. When requesting reviewers on pull requests, 
the \emph{pull-request} channel should be used. Any bugs and issues that require group consultation will go 
into the \emph{bugs-issues} channel. Discussion of progress updates for upcoming deliverables will go into the
\emph{status-updates} channel and lastly, any requests for help regarding work items will go into the \emph{help} channel.
If urgent contact with group members is required, \emph{SMS} should be used.

\section{Team Member Roles}
Individuals will take leadership in their respective assigned areas. In development, all members are responsible for adjusting and filling roles as the project evolves. All team meetings will have a scrum master and a scribe, rotated on a per meeting basis.
	\subsection{Allan Fang - Developer}
        \begin{itemize}
            \item Lead on mobile and user interface development
            \item Lead on optical character recognition research and development
            \item Graphics design and creation specialist
        \end{itemize}
    \subsection{Jason Nam - Developer}
        \begin{itemize}
            \item Lead on research and development concerning language models
            \item Lead on backend development
            \item Lead data engineer
        \end{itemize}
    \subsection{Sawyer Tang - Developer}
        \begin{itemize}
            \item Responsible for server setup
            \item Responsible for DevOps and version control
            \item Graphics design and creation specialist
            \item Lead on backend development
        \end{itemize}
    \subsection{Ryan Yeh - Developer}
        \begin{itemize}
            \item Lead on project documentation
            \item Lead on testing and quality assurance
            \item Primary contact for external communications
        \end{itemize}

\section{Workflow Plan}

\begin{itemize}
	\item How will you be using git, including branches, pull request, etc.?
	\item How will you be managing issues, including template issues, issue
	classificaiton, etc.?
\end{itemize}

\section{Proof of Concept Demonstration Plan}

What is the main risk, or risks, for the success of your project?  What will you
demonstrate during your proof of concept demonstration to convince yourself that
you will be able to overcome this risk?

\section{Technology}

\begin{itemize}
\item Specific programming language
\item Specific linter tool (if appropriate)
\item Specific unit testing framework
\item Investigation of code coverage measuring tools
\item Specific plans for Continuous Integration (CI), or an explanation that CI
  is not being done
\item Specific performance measuring tools (like Valgrind), if
  appropriate
\item Libraries you will likely be using?
\item Tools you will likely be using?
\end{itemize}

\section{Coding Standard}

\section{Project Scheduling}

\wss{How will the project be scheduled?}

\end{document}