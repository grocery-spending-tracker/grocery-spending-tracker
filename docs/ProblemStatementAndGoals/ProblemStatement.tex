\documentclass{article}

\usepackage{tabularx}
\usepackage{booktabs}

\title{Problem Statement and Goals\\\progname}

\author{\authname}

\date{}

%% Comments

\usepackage{color}

\newif\ifcomments\commentstrue %displays comments
%\newif\ifcomments\commentsfalse %so that comments do not display

\ifcomments
\newcommand{\authornote}[3]{\textcolor{#1}{[#3 ---#2]}}
\newcommand{\todo}[1]{\textcolor{red}{[TODO: #1]}}
\else
\newcommand{\authornote}[3]{}
\newcommand{\todo}[1]{}
\fi

\newcommand{\wss}[1]{\authornote{blue}{SS}{#1}} 
\newcommand{\plt}[1]{\authornote{magenta}{TPLT}{#1}} %For explanation of the template
\newcommand{\an}[1]{\authornote{cyan}{Author}{#1}}

%% Common Parts

\newcommand{\progname}{ProgName} % PUT YOUR PROGRAM NAME HERE
\newcommand{\authname}{Team \#, Team Name
\\ Student 1 name
\\ Student 2 name
\\ Student 3 name
\\ Student 4 name} % AUTHOR NAMES                  

\usepackage{hyperref}
    \hypersetup{colorlinks=true, linkcolor=blue, citecolor=blue, filecolor=blue,
                urlcolor=blue, unicode=false}
    \urlstyle{same}
                                


\begin{document}

\maketitle

\begin{table}[hp]
\caption{Revision History} \label{TblRevisionHistory}
\begin{tabularx}{\textwidth}{llX}
\toprule
\textbf{Date} & \textbf{Developer(s)} & \textbf{Change}\\
\midrule
20/09/23 & Sawyer Tang & Add Problem Statement.\\
20/09/23 & Ryan Yeh & Added section introduction for Problem Statement\\
20/09/23 & Jason Nam & Added Inputs and Outputs for Problem Statement Doc\\
22/09/23 & Jason Nam & Completed Stakeholders Section\\
23/09/23 & Ryan Yeh & Added initial Goals and Stretch Goals\\
24/09/23 & Jason Nam & Added Environment section for Problem Statement and Goals Doc\\
25/09/23 & Jason Nam & Removed team members as Stakeholders\\
\bottomrule
\end{tabularx}
\end{table}

\section{Problem Statement}

%\wss{You should check your problem statement with the
%\href{https://github.com/smiths/capTemplate/blob/main/docs/Checklists/ProbState-Checklist.pdf}
%{problem statement checklist}.}
%\wss{You can change the section headings, as long as you include the required information.}

This section will outline the problem being solved by the Grocery Spending Tracker. It will go
over the general problem, expected inputs and outputs of the application, stakeholders, and the environment
the application will be used in.

\subsection{Problem}

With the rising cost of living in Hamilton as well as across Canada, many households are searching for ways to cut back on costs of their living expenses. Grocery prices, most notably, are quickly on the rise and are making it difficult for households to handle/manage their budgets. \textbf{Grocery Spend Tracker} will strive to make it easier for users to make \textit{smart} grocery decisions to lower their spending on food and household needs.

\subsection{Inputs and Outputs}

\subsubsection{Inputs}
    \begin{itemize}
        \item Personal consumer spending data
        \item Location data
        \item Historical Shopping Lists
        \item Shopping History
    \end{itemize}

\subsubsection{Outputs}
    \begin{itemize}
        \item Analysis of spending data
        \item Purchase suggestions based on spending data
        \item Budgeting Plans
        \item Local Price Comparison
        \item Shopping Trip Summaries
    \end{itemize}

\subsection{Stakeholders}

\begin{itemize}
    \item Students of McMaster University
    \item Single income households
    \item Dr. Spencer Smith and TAs
\end{itemize}

\subsection{Environment}

\subsubsection{Hardware} 
The system will utilize any apparatus equipped with optical sensors for image capture and visual display components for presenting graphical information.
\subsubsection{Software}
The system will employ a centralized data repository managed by a database server to facilitate data storage and aggregation.

\section{Goals}

\begin{itemize}
  \item Create an intuitive interface for the application
  \begin{itemize}
    \item The application is intended for a broad range of users of different age groups
    and demographics. Therefore, the application should be easy to use such that anyone
    is able to navigate and use its features without guidance.
  \end{itemize}  
  \item Provide cost savings to users
  \begin{itemize}
    \item Considering the application is being created to help lessen the burden of increasing
    grocery prices and lower food spending, there should be a tangible decrease in grocery spending
    over time with its use.
  \end{itemize}
  \item Ensure reasonable performance for user convenience
  \begin{itemize}
    \item The application's performance should not provide any disruption to the user's life.
    It should be a tool that  can be seamlessly integrated without requiring user
    accommodation.
  \end{itemize} 
  \item Ensure high level of correctness on application features
  \begin{itemize}
    \item Since the application is targeted towards reducing spending of users, mistakes
    on the application side should be minimal and low risk if any. The possibility of negatively
    impacting a user's financial state due to bugs would be unacceptable for this kind of
    project.
  \end{itemize}  
\end{itemize}

\section{Stretch Goals}

\begin{itemize}
  \item Create a heatmap for locations with cheaper spending
  \begin{itemize}
    \item Using mapping and location data, a visual representation for locations with
    cheaper grocery spending could be created to better assist users in making smarter
    financial decisions.
  \end{itemize}  
  \item Add in-app budgeting suggestions for users
  \begin{itemize}
    \item Through further analysis of spending data and user data, it may be possible to provide 
    automatic in-app budgeting suggestions to further help reduce user grocery spending.
  \end{itemize}  
\end{itemize}  

\end{document}
