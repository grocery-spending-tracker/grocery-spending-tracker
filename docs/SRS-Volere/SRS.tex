% THIS DOCUMENT IS FOLLOWS THE VOLERE TEMPLATE BY Suzanne Robertson and James Robertson
% ONLY THE SECTION HEADINGS ARE PROVIDED
%
% Initial draft from https://github.com/Dieblich/volere
%
% Risks are removed because they are covered by the Hazard Analysis
\documentclass[12pt]{article}

\usepackage{booktabs}
\usepackage{tabularx}
\usepackage{hyperref}
\hypersetup{
    bookmarks=true,         % show bookmarks bar?
      colorlinks=true,      % false: boxed links; true: colored links
    linkcolor=red,          % color of internal links (change box color with linkbordercolor)
    citecolor=green,        % color of links to bibliography
    filecolor=magenta,      % color of file links
    urlcolor=cyan           % color of external links
}

\newcommand{\lips}{\textit{Insert your content here.}}

%% Comments

\usepackage{color}

\newif\ifcomments\commentstrue %displays comments
%\newif\ifcomments\commentsfalse %so that comments do not display

\ifcomments
\newcommand{\authornote}[3]{\textcolor{#1}{[#3 ---#2]}}
\newcommand{\todo}[1]{\textcolor{red}{[TODO: #1]}}
\else
\newcommand{\authornote}[3]{}
\newcommand{\todo}[1]{}
\fi

\newcommand{\wss}[1]{\authornote{blue}{SS}{#1}} 
\newcommand{\plt}[1]{\authornote{magenta}{TPLT}{#1}} %For explanation of the template
\newcommand{\an}[1]{\authornote{cyan}{Author}{#1}}

%% Common Parts

\newcommand{\progname}{ProgName} % PUT YOUR PROGRAM NAME HERE
\newcommand{\authname}{Team \#, Team Name
\\ Student 1 name
\\ Student 2 name
\\ Student 3 name
\\ Student 4 name} % AUTHOR NAMES                  

\usepackage{hyperref}
    \hypersetup{colorlinks=true, linkcolor=blue, citecolor=blue, filecolor=blue,
                urlcolor=blue, unicode=false}
    \urlstyle{same}
                                


\begin{document}

\title{Software Requirements Specification for \progname: subtitle describing software} 
\author{\authname}
\date{\today}
	
\maketitle

~\newpage

\pagenumbering{roman}

\tableofcontents

~\newpage

\section*{Revision History}

\begin{tabularx}{\textwidth}{p{3cm}p{2cm}X}
\toprule {\textbf{Date}} & {\textbf{Version}} & {\textbf{Notes}}\\
\midrule
Date 1 & 1.0 & Notes\\
Date 2 & 1.1 & Notes\\
\bottomrule
\end{tabularx}

~\\

~\newpage
\section{Purpose of the Project}
\subsection{User Business}

With the world currently facing record high inflation, cost-of-living is at the highest
it has ever been. This affects all daily necessities but is especially true for food and groceries.
As a whole, all households are affected but there is particular financial strain on those with lower-incomes.
As a result, interest in personal finance has grown and become more important in peoples' everyday lives.
To assist these individuals, we are developing an application that can help users better understand their
spending habits and make smarter financial decisions. This application will allow users to
take photos of grocery receipts and track their overall spending, analyze spending trends, and receive
suggestions on cheaper alternatives for purchased grocery items. Overall, we believe this application
will help users stay more informed and reduce grocery spending in the long-term.

\subsection{Goals of the Project}
\begin{itemize}
  \item The created application will help users save money on groceries over time.
  \item The application will provide accurate spending data and suggestions to end users.
\end{itemize}  
\section{Stakeholders}
\subsection{Client}
\lips
\subsection{Customer}
\lips
\subsection{Other Stakeholders}
\lips
\subsection{Hands-On Users of the Project}
\lips
\subsection{Personas}
\lips
\subsection{Priorities Assigned to Users}
\lips
\subsection{User Participation}
\lips
\subsection{Maintenance Users and Service Technicians}
\lips

\section{Mandated Constraints}
\subsection{Solution Constraints}
\lips
\subsection{Implementation Environment of the Current System}
\lips
\subsection{Partner or Collaborative Applications}
\lips
\subsection{Off-the-Shelf Software}
\lips
\subsection{Anticipated Workplace Environment}
\lips
\subsection{Schedule Constraints}
\lips
\subsection{Budget Constraints}
\lips
\subsection{Enterprise Constraints}
\lips

\section{Naming Conventions and Terminology}
\subsection{Glossary of All Terms, Including Acronyms, Used by Stakeholders
involved in the Project}
\lips

\section{Relevant Facts And Assumptions}
\subsection{Relevant Facts}
\lips
\subsection{Business Rules}
\lips
\subsection{Assumptions}
\lips

\section{The Scope of the Work}
\subsection{The Current Situation}
\lips
\subsection{The Context of the Work}
\lips
\subsection{Work Partitioning}
\lips
\subsection{Specifying a Business Use Case (BUC)}
\lips

\section{Business Data Model and Data Dictionary}
\subsection{Business Data Model}
\lips
\subsection{Data Dictionary}
\lips

\section{The Scope of the Product}
\subsection{Product Boundary}
\lips
\subsection{Product Use Case Table}
\lips
\subsection{Individual Product Use Cases (PUC's)}
\lips

\section{Functional Requirements}
\subsection{Functional Requirements}
\lips

\section{Look and Feel Requirements}
\subsection{Appearance Requirements}
\begin{itemize}
  \item \textbf{AR1}: The app should have a consistent colour scheme throughout.
  \item \textbf{AR2}: The app should have self-descriptive icons to convey to users what the related feature is.
  \item \textbf{AR3}: The user interface should avoid clutter and balance features with available space.
\end{itemize}
\subsection{Style Requirements}
\begin{itemize}
  \item \textbf{SR1}: All app features and options should be in consistent locations.
  \item \textbf{SR2}: Navigation options should always be available regardless of the application page.
  \item \textbf{SR3}: Features and options should be grouped based on relation to each other to improve intuitiveness.
  \item \textbf{SR4}: The number of clicks to access different features should be minimized.
  \item \textbf{SR5}: The app should utilize indicators and messages to communicate information to users.
\end{itemize}

\section{Usability and Humanity Requirements}
\subsection{Ease of Use Requirements}
\lips
\subsection{Personalization and Internationalization Requirements}
\lips
\subsection{Learning Requirements}
\lips
\subsection{Understandability and Politeness Requirements}
\lips
\subsection{Accessibility Requirements}
\lips

\section{Performance Requirements}
\subsection{Speed and Latency Requirements}
\lips
\subsection{Safety-Critical Requirements}
\lips
\subsection{Precision or Accuracy Requirements}
\lips
\subsection{Robustness or Fault-Tolerance Requirements}
\lips
\subsection{Capacity Requirements}
\lips
\subsection{Scalability or Extensibility Requirements}
\lips
\subsection{Longevity Requirements}
\lips

\section{Operational and Environmental Requirements}
\subsection{Expected Physical Environment}
\lips
\subsection{Wider Environment Requirements}
\lips
\subsection{Requirements for Interfacing with Adjacent Systems}
\lips
\subsection{Productization Requirements}
\lips
\subsection{Release Requirements}
\lips

\section{Maintainability and Support Requirements}
\subsection{Maintenance Requirements}
\lips
\subsection{Supportability Requirements}
\lips
\subsection{Adaptability Requirements}
\lips

\section{Security Requirements}
\subsection{Access Requirements}
\lips
\subsection{Integrity Requirements}
\lips
\subsection{Privacy Requirements}
\lips
\subsection{Audit Requirements}
\lips
\subsection{Immunity Requirements}
\lips

\section{Cultural Requirements}
\subsection{Cultural Requirements}
\lips

\section{Compliance Requirements}
\subsection{Legal Requirements}
\lips
\subsection{Standards Compliance Requirements}
\lips

\section{Open Issues}
\lips

\section{Off-the-Shelf Solutions}
\subsection{Ready-Made Products}
\lips
\subsection{Reusable Components}
\lips
\subsection{Products That Can Be Copied}
\lips

\section{New Problems}
\subsection{Effects on the Current Environment}
\lips
\subsection{Effects on the Installed Systems}
\lips
\subsection{Potential User Problems}
\lips
\subsection{Limitations in the Anticipated Implementation Environment That May
Inhibit the New Product}
\lips
\subsection{Follow-Up Problems}
\lips

\section{Tasks}
\subsection{Project Planning}
\lips
\subsection{Planning of the Development Phases}
\lips

\section{Migration to the New Product}
\subsection{Requirements for Migration to the New Product}
\lips
\subsection{Data That Has to be Modified or Translated for the New System}
\lips

\section{Costs}
\lips
\section{User Documentation and Training}
\subsection{User Documentation Requirements}
\lips
\subsection{Training Requirements}
\lips

\section{Waiting Room}
\lips

\section{Ideas for Solution}
\lips

\newpage{}
\section*{Appendix --- Reflection}

The information in this section will be used to evaluate the team members on the
graduate attribute of Lifelong Learning.  Please answer the following questions:

\begin{enumerate}
  \item What knowledge and skills will the team collectively need to acquire to
  successfully complete this capstone project?  Examples of possible knowledge
  to acquire include domain specific knowledge from the domain of your
  application, or software engineering knowledge, mechatronics knowledge or
  computer science knowledge.  Skills may be related to technology, or writing,
  or presentation, or team management, etc.  You should look to identify at
  least one item for each team member.
  \item For each of the knowledge areas and skills identified in the previous
  question, what are at least two approaches to acquiring the knowledge or
  mastering the skill?  Of the identified approaches, which will each team
  member pursue, and why did they make this choice?
\end{enumerate}

\section*{Added Section - Appendix - Stakeholder Requirements Elicitation Notes}

\begin{itemize}
  \item Consider your favourite or most used mobile apps, what elements of the UI
  stand out? What in your opinion makes an appealing user interface?
    \begin{itemize}
      \item Minimize options to create a cleaner UI.
      \item Utilization of easily accessible menu bars (like Instagram) to improve navigation UX.
      \item Consistent and cohesive colour theme throughout.
      \item Use of icons that clearly convey what the option is meant for.
      \item Put options in consistent locations to minimize hand movements (easy to tap interface).
    \end{itemize}
  \item When you think of intuitive applications, what about the application makes
  the UI or UX intuitive?
    \begin{itemize}
      \item Many apps borrow UI layouts from each other, helps with familiarity in terms of navigation.
      \item Always accessible navigation options.
      \item Proper organizing of features (unrelated features should be separated on the UI).
      \item Swipe navigation and minimal long presses.
      \item Saving state when closed to minimize user input on future uses.
    \end{itemize}
  \item How important do you consider application feedback and what feedback would you
  expect from an application? (i.e. loading indicators)
    \begin{itemize}
      \item High importance to communicate errors and loading to users.
      \item Should always convey when issues happen or something is being done.
      \item When fetching information or other longer duration tasks, loading bar should be utilized.
      \item Utilization of messages or pop ups if resources are unavailable or something goes wrong.
    \end{itemize}
  \item When learning a new app, how long would you consider a reasonable amount of time
  to consider an app easy-to-use?
    \begin{itemize}
      \item Average acceptable time to consider an app intuitive would be to be able to figure
      out basic functionality within 15 minutes.
    \end{itemize}
  \item What features would you want to help with the learning experience of a new app?
    \begin{itemize}
      \item The use of a tutorial on first launch to take users through the UI and different features.
      \item The use of tooltips for certain options and features.
      \item A button that users can press to access a Help menu or FAQ.
    \end{itemize}
  \item How do you usually go about learning a new application?
    \begin{itemize}
      \item All participants said they tend to brute force applications in order to learn
      (i.e. clicking around and seeing how the application responds).
      \item As an addition to this brute force approach, participants explained how useful an
      undo feature would be as well as always accessible navigation.
    \end{itemize}
  \item If you were using an application, what personalizations/accessibility settings
  would you consider a must?
    \begin{itemize}
      \item Dark/Light mode.
      \item Ability to  change text sizes.
      \item Multiple language support.
      \item Allowing users to enable or disable notifications if used.
      \item Alternate color palettes to help with those with visual impairment like color blindness.
    \end{itemize}
  \item Consider an app you use often, generally, how long would you say you wait for operations
  to complete? How long would you be willing to wait for an application before closing the app?
    \begin{itemize}
      \item Ideally, regular operations should take a second or less.
      \item At most, participants would wait an average of 12 seconds before closing the app or
      thinking something went wrong.
    \end{itemize}
  \item What kind of phone do you use? (Apple/Android/Other)
    \begin{itemize}
      \item All participants used Apple devices.
    \end{itemize}
  \item On a scale of 1-10, for an app of this nature, how important would you consider accuracy
  of the system and data? How important would you consider responsiveness on the same scale?
    \begin{itemize}
      \item All participants considered accuracy of the features and data to be the most important (10) due
      to the financial aspect of the app.
      \item Responsiveness was considered important (8) in order to ensure a positive user experience. If the
      app does not feel good to use, most participants said they would find another app instead.
    \end{itemize}
  \item How much do you consider the size of an application you install? How big of an app would you
  consider reasonable?
    \begin{itemize}
      \item Most participants said they did not pay much attention to the size of applications when installing
      them.
      \item Participants said they would consider anything over 1GB unreasonable for an app of this nature.
    \end{itemize}
\end{itemize}

\end{document}