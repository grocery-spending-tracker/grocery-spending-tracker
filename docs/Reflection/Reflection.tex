\documentclass{article}

\usepackage{tabularx}
\usepackage{booktabs}

\title{Reflection Report on \progname}

\author{\authname}

\date{}

%% Comments

\usepackage{color}

\newif\ifcomments\commentstrue %displays comments
%\newif\ifcomments\commentsfalse %so that comments do not display

\ifcomments
\newcommand{\authornote}[3]{\textcolor{#1}{[#3 ---#2]}}
\newcommand{\todo}[1]{\textcolor{red}{[TODO: #1]}}
\else
\newcommand{\authornote}[3]{}
\newcommand{\todo}[1]{}
\fi

\newcommand{\wss}[1]{\authornote{blue}{SS}{#1}} 
\newcommand{\plt}[1]{\authornote{magenta}{TPLT}{#1}} %For explanation of the template
\newcommand{\an}[1]{\authornote{cyan}{Author}{#1}}

%% Common Parts

\newcommand{\progname}{ProgName} % PUT YOUR PROGRAM NAME HERE
\newcommand{\authname}{Team \#, Team Name
\\ Student 1 name
\\ Student 2 name
\\ Student 3 name
\\ Student 4 name} % AUTHOR NAMES                  

\usepackage{hyperref}
    \hypersetup{colorlinks=true, linkcolor=blue, citecolor=blue, filecolor=blue,
                urlcolor=blue, unicode=false}
    \urlstyle{same}
                                


\begin{document}

\maketitle

\plt{Reflection is an important component of getting the full benefits from a
learning experience.  Besides the intrinsic benefits of reflection, this
document will be used to help the TAs grade how well your team responded to
feedback.  In addition, several CEAB (Canadian Engineering Accreditation Board)
Learning Outcomes (LOs) will be assessed based on your reflections.}

\section{Changes in Response to Feedback}

% \plt{Summarize the changes made over the course of the project in response to
% feedback from TAs, the instructor, teammates, other teams, the project
% supervisor (if present), and from user testers.}

% \plt{For those teams with an external supervisor, please highlight how the feedback 
% from the supervisor shaped your project.  In particular, you should highlight the 
% supervisor's response to your Rev 0 demonstration to them.}

% \plt{Version control can make the summary relatively easy, if you used issues
% and meaningful commits.  If you feedback is in an issue, and you responded in
% the issue tracker, you can point to the issue as part of explaining your
% changes.  If addressing the issue required changes to code or documentation, you
% can point to the specific commit that made the changes.}

\subsection{SRS and Hazard Analysis}



\subsection{Design and Design Documentation}

For the Module Guide, several changes were made based on TA feedback. One such change was
including traceability in the Module Development timelines. To accomplish this, references
were placed into the timeline tables such that readers can quickly view the contents of each
module. Additionally, there were some implementation details regarding specific technology that
should not have been present in the document. As a result, these were removed. Lastly, we were
informed that we should be more specific in regards to the Hardware Hiding Module. Therefore,
we added more details to that module section including specific hardware devices such as a camera.

In the Module Interface Specification, we were informed via TA feedback that some of our syntax
was incorrect. Specifically, syntax regarding types on declared variables and their definitions.
There was also some improper text formatting problems with multi-character variables, all of which
have since been resolved. Additionally, modules in the MIS were missing their Module Types so a new
section was added for all modules to define their module types.

\subsection{VnV Plan and Report}

Based on TA feedback, we added symbolic constants to our VnV Plan. For instance, rather than
using a fixed number such as 70\% in our test descriptions, we changed it to a constant such
as PASSING\_RESPONSE\_RATE. Additionally, we were asked to provide more details on some tests. Specifically,
what constitutes as a pass and some other ambiguity regarding survey usage. Lastly, the
Implementation Verification Plan was missing some details so those were added as well. Based on peer review feedback,
some tests were updated to include missing information such as an initial state, input, and expected output.
There was also feedback to add some more detail to the Design Verification Plan and alphabetize the Symbols,
Abbreviations, and Acronyms table, both of which were addressed.

For the VnV Report, many of the TA feedback changes that were addressed had to do with formatting and
appearance of the document. This included alphabetizing the Symbols, Abbreviations and Acronyms table,
changing some of the table appearances, and adding table rows with short test descriptions to better show
what each test is validating. There were also some minor changes requested such as GitHub stylizing and
explicitly mentioning table names instead of something like ``the above table''. Some peer review feedback
was also addressed in the VnV Report, such as some ambiguity regarding one of our performance graphs and some
missing details on specific tests. We were also asked to explain who the external users we tested with were which
we resolved by pointing readers to the Stakeholders section of our SRS document.

\section{Design Iteration (LO11)}

% \plt{Explain how you arrived at your final design and implementation.  How did
% the design evolve from the first version to the final version?} 

The biggest design change in the application had to do with the receipt scanning portion of the
application. Initially, the vision for the application was to have our own optical character recognition (OCR)
system as part of the application. As such, the original implementation of the OCR involved the use
of various Python scripts using Tesseract OCR and OpenCV. This was demonstrated as part of the proof of concept demo,
however soon afterwards, we realized we were having issues with robustness, accuracy, and speed. Given that this
was meant to be a consumer application, usability was a very important trait for the app to have and these issues
were significant roadblocks for us. After talking to Dr. Smith, he referred us to Dr. Anand who gave us
feedback regarding our implementation. He suggested us to use an existing OCR implementation given
that there was already existing technology that solved what we were trying to do. Additionally, in the interest of time,
we did not have the resources to create our own, optimized OCR implementation in addition to a full
mobile app. As a result, we decided to take Dr. Anand's advice and use Google MLKit as our final OCR implementation. This
design change provided significant improvements to the app's usability, with robustness, accuracy, and speed all being
positively impacted. This also allowed us to dedicate resources to other parts of the application resulting in a
more complete app than what otherwise would have likely been created. This design change had the biggest impact on
the final outcome of the project and we believe the current implementation is far better than the original implementation
we had originally planned.

The other large design change for the application had to do with the user interface (UI). In our Revision 0 demo,
we had a very preliminary UI that simply served to navigate to the different features we wanted to show. At the time,
we had put very limited thought into the actual user experience and visual appeal of the app due to prioritization of
features, however part of the feedback we got from the demo was finalizing a UI design since we would have to do
usability testing as part of the following deliverable. This resulted in us looking back to the data we had originally
elicited from McMaster students during the SRS creation. The general consensus from the original user feedback was that
they wanted an application that felt familiar. As a result, we referenced several existing applications during this development
period and settled on the final design which takes inspiration from apps such as Instagram, McDonald's, and Apple Health. We ended
up finalizing a colour scheme and polishing some of the components to make everything look more cohesive as well. After
completion, this final UI design was very positively received by the McMaster students we showed it to and during usability tests, 
these same students were able to figure out general functionality within 15 minutes. As a result of putting more emphasis on
user feedback, we were able to create a UI design that was more usable and appealing than the original implementation we had
demonstrated.

Based on feedback during the Revision 0 demo, we also made changes to the budgets such as showing the total budget for each,
rather than only showing a bar of progress as was initially done. Additionally, based on user feedback, tap focus was added
to the receipt capture in order to improve the user experience.

\section{Design Decisions (LO12)}

% \plt{Reflect and justify your design decisions.  How did limitations,
%  assumptions, and constraints influence your decisions?}

Due to limited time, some of the design decisions we made involved scoping down or changing
the project. For instance, as previously mentioned, we used Google MLKit for our OCR implementation
rather than making our own. This was due to Dr. Anand's recommendation but also due to the fact that
there was more to our capstone project than just the OCR portion. As a result, we used an existing implementation
in order to focus efforts on other portions of the app. We also had to scope down some of the features we
had originally planned. We had initially intended to have location support in order to give recommendations
based on user and store location, however, we found that we had overcommitted and did not have enough time to
implement this functionality. As a result, it had to be removed from the current scope of the project.

Another design decision that had to be changed during the development period pertains to the Classification
Engine. Originally, we had planned to construct a database of products in order to classify items from different
grocery stores. This, however, was not feasible as after contacting different stores, we were not able to acquire
a list of product data. As a result, we had to use web scraping and approximate string matching to construct our own database
as users input receipts into the system.

Finally, we had planned for iOS and Android support in the final version of the application however the final version
is only supported by Android devices. This was a result of limited time, but also restrictions placed on developers
by Apple themselves. Apple only allows users with Apple devices to develop iOS applications. Furthermore, only two members of the
team had Mac computers/iPhones while the other two had Android and
Windows devices. This meant all developers could work on Android support but only two developers could handle iOS
support. Since not everyone was able to work on it, we decided that iOS support would be out of scope for the current
implementation. In other words, iOS support had to be pushed due to constraints put on iOS development by Apple.

\section{Economic Considerations (LO23)}

% \plt{Is there a market for your product? What would be involved in marketing your 
% product? What is your estimate of the cost to produce a version that you could 
% sell?  What would you charge for your product?  How many units would you have to 
% sell to make money? If your product isn't something that would be sold, like an 
% open source project, how would you go about attracting users?  How many potential 
% users currently exist?}
Since this product was developed with the low-income public as the primary stakeholders and users, we feel strongly that there is a substantial demand and potential users (essentially the whole low-income population) for the services our product provides. The estimated cost to fully develop and produce a version that we feel has enough functionality to be suitable for the market would likely be around \$100,000, primarily paying for the salaries of the development team for 6 months. Due to the charitable nature and motives for developing this product, we are highly motivated to provide the services of the product for free. Taking this into account, the team will need to consider alternative methods to generate revenue, such as a freemium business model, in-app advertising, or the sale of user data (ethically of course). Marketing the product and attracting users will likely begin with low-cost marketing efforts such as social media accounts, network marketing, and working with non-profit organizations at McMaster University. At this point, the team feels it is not likely that the product will be able to generate significant or break-even revenues unless the product can be bought out by a larger organization that can operate the product at a loss or can find other ways to monetize.

\section{Reflection on Project Management (LO24)}

% \plt{This question focuses on processes and tools used for project management.}

\subsection{How Does Your Project Management Compare to Your Development Plan}

% \plt{Did you follow your Development plan, with respect to the team meeting plan, 
% team communication plan, team member roles and workflow plan.  Did you use the 
% technology you planned on using?}

In terms of the team meeting and team communication plan, both were followed closely as documented.
Meetings took place weekly on Wednesdays and Sundays with certain exceptions such as the exam period. Roles were
also followed closely for each meeting and were consistent throughout the project's duration. Moreover, we used Discord,
SMS, and GitHub for communication as was described.

For the team member roles documented, they were generally followed with one change. Jason Nam and Sawyer Tang
followed their roles as described however Allan Fang's and Ryan Yeh's roles slightly changed due to workload capacity.
Allan Fang led frontend development as written but due to extra capacity, Ryan Yeh took on the lead for OCR development in addition
to his other roles. Besides that, the team roles were followed as written.

Regarding the workflow plan, Git and GitHub were used as described in the document with issue tracking and the
version control process being followed.

Finally, the technologies described were followed throughout the project with the exception of the usage of
node-tesseract-ocr. This library was included due to the assumption we would be implementing our own OCR, however
we ended up using Google MLKit instead to handle the receipt capture process.

\subsection{What Went Well?}

% \plt{What went well for your project management in terms of processes and 
% technology?}

The scheduled meetings were very useful to ensure everyone was on track for deadlines.
It also gave transparency regarding any roadblocks and offered an opportunity for other members
of the team to offer suggestions and problem solve together. Overall, it helped keep everyone accountable
which ensured quality work was delivered on time. The GitHub issue tracking process was also very
useful in tracking what was being worked on and what still needed to be worked on. The issues also helped us to organize
our work and isolate different tasks through the use of labels. This made the division and tracking of work
very easy and more efficient overall. Finally, the CI/CD pipeline via GitHub Actions was very helpful in terms
of development. In particular, having unit tests automatically execute when new code is
pushed helped us to ensure that high quality features were being produced and that there were no new problems being introduced.

\subsection{What Went Wrong?}

% \plt{What went wrong in terms of processes and technology?}

We did not have anything that explicitly went wrong however one problem that was encountered multiple times during
the project's duration had to do with the workflow process and pull requests. Part of the workflow
involved waiting for two reviewers to approve a pull request before a merge could take place. Often,
pull requests would get ``stuck'' because reviews were not occurring as frequently
as they should have. As a result, pull request would pile up and merged in bulk rather than a constant stream
of work being pushed through. This would often mean that before a deadline, everyone would go through
a large number of pull requests to push everything that was required for a deliverable which was not ideal.

\subsection{What Would you Do Differently Next Time?}

% \plt{What will you do differently for your next project?}

For projects in the future, we would like to improve the overall planning. For this project, we had to change scope/design decisions several
times which albeit, is expected, however better planning could have reduced the number of
times we had to do it. Next time, we would like to create more detailed timelines for development with exact
expectations documented for each period. This would help with deciding feasibility for features and allow us
to set a better initial scope. More research would also help during the initial phases of the project.
For instance, by researching more OCR approaches/libraries, we might have been able to settle on an off-the-shelf solution
earlier rather than putting resources into something we would eventually scrap. Lastly, we think that speaking with experts/more
knowledgeable individuals before starting a project would be very important as well. Our meeting with Dr. Anand was very insightful
and if we had the opportunity to speak earlier, we likely would have been able to make better design decisions right away.

We also believe that communication could have been improved, especially in regards to pull requests.
Something as simple as more frequent reminders could have improved the throughput of code merges and produced a more
consistent stream of pull requests. As such, frequent communication and reminders would likely be beneficial for more efficient future
projects.

\section{Reflection on Capstone}

% \plt{This question focuses on what you learned during the course of the capstone project.}

\subsection{Which Courses Were Relevant}

% \plt{Which of the courses you have taken were relevant for the capstone project?}

In terms of documentation, SFWRENG 2AA4 was very helpful for this capstone project. This course was our first
introduction to \LaTeX{} and writing software documentation. Working on documentation in that class greatly helped
with writing and working with the documentation for this project. Similarly, SFWRENG 2DM3, SFWRENG 2FA3, and SFWRENG 3RA3
all helped with documentation as well. SFWRENG 2DM3 and SFWRENG 2FA3 were introductory courses for discrete mathematics
which was needed for formalizing our requirements and design, such as in the MIS. SFWRENG 3RA3 taught about software requirements and was the course
that introduced us to different elicitation techniques. This course was leveraged for the different meetings we had with users to elicit
various requirements. It also taught the different kinds of requirements (function and non-functional) which also used for this project, particularly
in the SRS. Outside of documentation, SFWRENG 3DB3 introduced us to relational databases which we used in our final
implementation. This course built the fundamentals which allowed us to plan and construct our own database for this application.
Furthermore, SFWRENG 3A04 taught the basics of software architecture which greatly helped with the design process of our
application. Referencing the different architectures covered, we were able to build out our own application for this project. Finally,
SFWRENG 3S03 was important because it taught the basics of software testing and the different testing methodologies available. This
course helped us to build out our own testing library with the appropriate code coverage to ensure our application was reliable
and functioning as intended.

\subsection{Knowledge/Skills Outside of Courses}

% \plt{What skills/knowledge did you need to acquire for your capstone project
% that was outside of the courses you took?}

Much of the knowledge/skills related to specific technologies required learning and research outside of our coursework.
In particular, Flutter/Dart was a new programming language for all of us so we had to do research on the documentation as well
as watch some videos to familiarize ourselves with it. Likewise, not all of us had worked with NodeJS before or PostgreSQL
so similar learning had to be done. This involved learning more than just the language as well, we tried to familiarize ourselves
with the best coding practices and syntax too. Another skill we had to learn was using Azure and how to deploy an application on it.
We decided to use Azure to host our backend and database, so there was a learning curve to configure everything properly. There was
also a great deal of troubleshooting required which we had to utilize external resources to solve. CI/CD and specifically,
CI/CD through GitHub Actions was also new for most of us. All of us understood what it was conceptually as it had been briefly
covered in courses, however using it in practice was something we had to teach ourselves. This involved learning the
file structure for workflows, the syntax, and importing of existing actions to create our CI/CD pipeline. Overall, the knowledge/skills
outside of courses were mostly technology specific and although some of the concepts were familiar, we still had to learn how
to actually use them in practice.

\end{document}