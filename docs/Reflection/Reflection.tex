\documentclass{article}

\usepackage{tabularx}
\usepackage{booktabs}

\title{Reflection Report on \progname}

\author{\authname}

\date{}

\input{../Comments}
%% Common Parts

\newcommand{\progname}{Grocery Spending Tracker} % PUT YOUR PROGRAM NAME HERE
\newcommand{\authname}{Team 1, JARS
\\ Jason Nam
\\ Allan Fang
\\ Ryan Yeh
\\ Sawyer Tang} % AUTHOR NAMES                  

\usepackage{hyperref}
    \hypersetup{colorlinks=true, linkcolor=blue, citecolor=blue, filecolor=blue,
                urlcolor=blue, unicode=false}
    \urlstyle{same}
                                


\begin{document}

\maketitle

\plt{Reflection is an important component of getting the full benefits from a
learning experience.  Besides the intrinsic benefits of reflection, this
document will be used to help the TAs grade how well your team responded to
feedback.  In addition, several CEAB (Canadian Engineering Accreditation Board)
Learning Outcomes (LOs) will be assessed based on your reflections.}

\section{Changes in Response to Feedback}

\plt{Summarize the changes made over the course of the project in response to
feedback from TAs, the instructor, teammates, other teams, the project
supervisor (if present), and from user testers.}

\plt{For those teams with an external supervisor, please highlight how the feedback 
from the supervisor shaped your project.  In particular, you should highlight the 
supervisor's response to your Rev 0 demonstration to them.}

\plt{Version control can make the summary relatively easy, if you used issues
and meaningful commits.  If you feedback is in an issue, and you responded in
the issue tracker, you can point to the issue as part of explaining your
changes.  If addressing the issue required changes to code or documentation, you
can point to the specific commit that made the changes.}

\subsection{SRS and Hazard Analysis}

\subsection{Design and Design Documentation}

\subsection{VnV Plan and Report}

\section{Design Iteration (LO11)}

% \plt{Explain how you arrived at your final design and implementation.  How did
% the design evolve from the first version to the final version?} 

The biggest design change in the application had to do with the receipt scanning portion of the
application. Initially, the vision for the application was to have our own optical character recognition (OCR)
system as part of the application. As such, the original implementation of the OCR involved the use
of various Python scripts using Tesseract OCR and OpenCV. This was demonstrated as part of the proof of concept demo,
however soon afterwards, we realized we were having issues with robustness, accuracy, and speed. Given that this
was meant to be a consumer application, usability was a very important trait for the app to have and these issues
were significant roadblocks for us. After talking to Dr. Smith, he referred us to Dr. Anand who gave us
feedback regarding our implementation. He suggested us to use an existing OCR implementation given
that there was already existing technology that solved what we were trying to do. Additionally, in the interest of time,
we did not have the resources to create our own, optimized OCR implementation in addition to a full
mobile app. As a result, we decided to take Dr. Anand's advice and use Google MLKit as our final OCR implementation. This
design change provided significant improvements to the app's usability, with robustness, accuracy, and speed all being
positively impacted. This also allowed us to dedicate resources to other parts of the application resulting in a
more complete app than what otherwise would have likely been created. This design change had the biggest impact on
the final outcome of the project and we believe the current implementation is far better than the original implementation
we had originally planned.

The other large design change for the application had to do with the user interface (UI). In our Revision 0 demo,
we had a very preliminary UI that simply served to navigate to the different features we wanted to show. At the time,
we had put very limited thought into the actual user experience and visual appeal of the app due to prioritization of
features, however part of the feedback we got from the demo was finalizing a UI design since we would have to do
usability testing as part of the following deliverable. This resulted in us looking back to the data we had originally
elicited from McMaster students during the SRS creation. The general consensus from the original user feedback was that
they wanted an application that felt familiar. As a result, we referenced several existing applications during this development
period and settled on the final design which takes inspiration from apps such as Instagram, McDonald's, and Apple Health. We ended
up finalizing a colour scheme and polishing some of the components to make everything look more cohesive as well. After
completion, this final UI design was very positively received by the McMaster students we showed it to and during usability tests, 
these same students were able to figure out general functionality within 15 minutes. As a result of putting more emphasis on
user feedback, we were able to create a UI design that was more usable and appealing than the original implementation we had
demonstrated.

Based on feedback during the Revision 0 demo, we also made changes to the budgets such as showing the total budget for each,
rather than only showing a bar of progress as was initially done. Additionally, based on user feedback, tap focus was added
to the receipt capture in order to improve the user experience.

\section{Design Decisions (LO12)}

% \plt{Reflect and justify your design decisions.  How did limitations,
%  assumptions, and constraints influence your decisions?}

Due to limited time, some of the design decisions we made involved scoping down or changing
the project. For instance, as previously mentioned, we used Google MLKit for our OCR implementation
rather than making our own. This was due to Dr. Anand's recommendation but also due to the fact that
there was more to our capstone project than just the OCR portion. As a result, we used an existing implementation
in order to focus efforts on other portions of the app. We also had to scope down some of the features we
had originally planned. We had initially intended to have location support in order to give recommendations
based on user and store location, however, we found that we had overcommitted and did not have enough time to
implement this functionality. As a result, it had to be removed from the current scope of the project.

Another design decision that had to be changed during the development period pertains to the Classification
Engine. Originally, we had planned to construct a database of products in order to classify items from different
grocery stores. This, however, was not feasible as after contacting different stores, we were not able to acquire
a list of product data. As a result, we had to use web scraping and approximate string matching to construct our own database
as users input receipts into the system.

Finally, we had planned for iOS and Android support in the final version of the application however the final version
is only supported by Android devices. This was a result of limited time, but also restrictions placed on developers
by Apple themselves. Apple only allows users with Apple devices to develop iOS applications. Furthermore, only two members of the
team had Mac computers/iPhones while the other two had Android and
Windows devices. This meant all developers could work on Android support but only two developers could handle iOS
support. Since not everyone was able to work on it, we decided that iOS support would be out of scope for the current
implementation. In other words, iOS support had to be pushed due to constraints put on iOS development by Apple.

\section{Economic Considerations (LO23)}

\plt{Is there a market for your product? What would be involved in marketing your 
product? What is your estimate of the cost to produce a version that you could 
sell?  What would you charge for your product?  How many units would you have to 
sell to make money? If your product isn't something that would be sold, like an 
open source project, how would you go about attracting users?  How many potential 
users currently exist?}

\section{Reflection on Project Management (LO24)}

\plt{This question focuses on processes and tools used for project management.}

\subsection{How Does Your Project Management Compare to Your Development Plan}

\plt{Did you follow your Development plan, with respect to the team meeting plan, 
team communication plan, team member roles and workflow plan.  Did you use the 
technology you planned on using?}

\subsection{What Went Well?}

\plt{What went well for your project management in terms of processes and 
technology?}

\subsection{What Went Wrong?}

\plt{What went wrong in terms of processes and technology?}

\subsection{What Would you Do Differently Next Time?}

\plt{What will you do differently for your next project?}

\section{Reflection on Capstone}

\plt{This question focuses on what you learned during the course of the capstone project.}

\subsection{Which Courses Were Relevant}

\plt{Which of the courses you have taken were relevant for the capstone project?}

\subsection{Knowledge/Skills Outside of Courses}

\plt{What skills/knowledge did you need to acquire for your capstone project
that was outside of the courses you took?}

\end{document}